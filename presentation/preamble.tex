%%%%%%%%%%%%%%%%%%% Packages %%%%%%%%%%%%%%%%%%%
\usepackage{amsmath}
\usepackage{amsfonts}
\usepackage{amssymb}
\usepackage{pifont}
\usepackage{hyperref}
\usepackage{svg}
\usepackage{framed}
\ifcsname tikzexternalrealjob\endcsname
    \usepackage[draft, outputdir=build]{minted}
\else
    \usepackage[cache, outputdir=build]{minted}
\fi
\usepackage[export]{adjustbox}
\usepackage{xspace}
\usepackage{array}
\usepackage{float}
\usepackage[autostyle]{csquotes}
\usepackage[backend=biber, style=ieee, urldate=long]{biblatex} % biblatex - bibliography tool %
\renewcommand*{\bibfont}{\tiny}
\bibliography{../00-preliminary/bib}
\usepackage{pgfplots}
\usepackage{pgfplotstable}
\pgfplotsset{compat=1.15}
\usepackage{tikz}
\usetikzlibrary{arrows,arrows.meta, fit, backgrounds, positioning, automata, shapes, external}

% Table macro.
% 1: Table Latex
% 2: Caption
% 3: Label
\definecolor{rowgray}{HTML}{E0E0E0}
\newcommand{\makeTable}[3]{
    \rowcolors{1}{}{rowgray}
    \begin{table}[H]
        \sisetup{round-mode=places}
        \alignCenter{
        \begin{tabular}
            #1
        \end{tabular}}
        \caption{#2}
        \label{tab:#3}
    \end{table}
    \rowcolors{0}{}{}
}

% Table macro.
% 1: Table Latex
% 2: Caption
% 3: Label
\newcommand{\makeTablePB}[3]{
    \centering
    \rowcolors{1}{}{rowgray}
    \begin{longtable}
        #1
        \caption{#2}
        \label{table:#3}
    \end{longtable}
    \rowcolors{0}{}{}
}

\newcommand{\tableref}[1]{Table \ref{tab:#1}}
\newcommand{\tabref}[1]{\tableref{#1}}
\newcommand{\figureref}[1]{Figure \ref{fig:#1}}
\newcommand{\figref}[1]{Figure \ref{fig:#1}}
\newcommand{\coderef}[1]{Code \ref{code:#1}}
\newcommand{\lstref}[1]{Listing \ref{lst:#1}}
\newcommand{\secref}[1]{Section \ref{sec:#1}}
\newcommand{\chapref}[1]{Chapter \ref{chap:#1}}
\newcommand{\charef}[1]{\chapref{#1}}
\newcommand{\appendixref}[1]{Appendix \ref{app:#1}}
\newcommand{\apxref}[1]{\appendixref{#1}}
\newcommand{\appref}[1]{\appendixref{#1}}
%\newcommand{\lineref}[1]{Line \ref{line:#1}}

%\newcommand{\acrfull}[1]{\acrlong{#1} (\acrshort{#1})}

\newcommand{\dquote}[1]{``{#1}''}
\newcommand{\dquotes}[1]{\dquote{#1}}
\newcommand{\squote}[1]{`{#1}'}
\newcommand{\squotes}[1]{\squote{#1}}

\newcommand{\quoteWithCite}[3]{
  \begin{quotation}
  \noindent \textit{\dquote{#1}}
  \begin{flushright}
    -#2\cite{#3}
  \end{flushright}
  \end{quotation}
}

\newcommand{\quoteParticipant}[3]{
  \begin{quotation}
  \noindent \textit{\dquote{#1}}
  \begin{flushright}
    -#2, \appref{#3}
  \end{flushright}
  \end{quotation}
}

\newcommand{\quoteWithoutCite}[2]{
  \begin{quotation}
  \noindent \textit{\dquote{#1}}
  \begin{flushright}
  -#2
  \end{flushright}
  \end{quotation}
}

\definecolor{morell}{rgb}{213,0,96}
\newcommand{\tmc}[1]{\todo[color=morell]{#1}\xspace}
\newcommand{\btc}[1]{\todo[color=red!75]{#1}\xspace}
\newcommand{\tmcc}[1]{\todo[color=cyan]{#1}\xspace}
\newcommand{\needcite}{\todo{Citation needed}\xspace}
\newcommand{\inline}[1]{\todo[inline]{#1}}
\newcommand{\ttt}[1]{\texttt{#1}}
\newcommand{\m}[1]{\ttt{#1}}
\newcommand{\unity}{Unity\xspace}
\newcommand{\unreal}{Unreal Engine\xspace}
\newcommand{\cryengine}{CryEngine\xspace}

%%%%%%%% Column types
\newcolumntype{L}[1]{>{\raggedright\let\newline\\\arraybackslash\hspace{0pt}}m{#1}}
\newcolumntype{C}[1]{>{\centering\let\newline\\\arraybackslash\hspace{0pt}}m{#1}}
\newcolumntype{R}[1]{>{\raggedleft\let\newline\\\arraybackslash\hspace{0pt}}m{#1}}
\newcolumntype{P}[1]{>{\raggedright\arraybackslash}p{#1}}

\newcommand\rowEnd{\\\hline}

\newcommand{\metasheep}{
    \begin{figure}[H]
        \begin{center}
            {\LARGE \setstretch{0.75}
                \hspace{5em}        \ttt{,ww}\\
                \hspace{1.5em}      \ttt{wWWWWWWW\textbackslash\_)}\\
                \hspace{1em}        \ttt{`WWWWWW'}\\
                \hspace{1em}        \ttt{II}
                \hspace{1ex}        \ttt{II}\\
            }
            \caption{{\color{red} \ttt{"Beep beep, I'm a meta sheep. Replace me with meta text!"}}}
        \end{center}
    \end{figure}
}

\newcommand\alignCenter[1]{\makebox[\textwidth][c]{#1}}
\newcommand\alignRight[1]{\makebox[\textwidth][r]{#1}}
\newcommand\alignLeft[1]{\makebox[\textwidth][l]{#1}}

\newcommand\monitor[1]{\hspace*{-2.6em}\ttt{Monitor: }#1\\}
\newcommand\participant[1]{\hspace*{-5em}\ttt{Participant: }#1\\}
\newcommand\partic[1]{\participant{#1}}

\newcommand{\cpp}{C++\xspace}
\newcommand{\csh}{C\#\xspace}
\newcommand{\cs}{\csh}
\newcommand{\fsh}{F\#\xspace}
\newcommand{\fs}{\fsh}

\newcommand{\fsinline}[1]{\mintinline{fsharp}{#1}}
\newcommand{\fsline}[1]{\fsinline{#1}}
\newcommand{\csinline}[1]{\mintinline{csharp}{#1}}
\newcommand{\csline}[1]{\csinline{#1}}

%%% Method Name Macros

\newcommand{\champagne}{Champagne Prototyping\xspace}
\newcommand{\discount}{Discount Method for Language Evaluation\xspace}
\newcommand{\ida}{Instant Data Analysis\xspace}
\newcommand{\cognitive}{Cognitive Dimensions\xspace}
\newcommand{\attention}{Attention Investment Model\xspace}
\newcommand{\attentions}{Attention Investment Models\xspace}
\newcommand{\expert}{Expert Review Method\xspace}


\newcommand{\diagram}[2]{
    \makebox[\textwidth][c]{
        \begin{tikzpicture}
        #2
        \end{tikzpicture}
    }\label{fig:#1}
}

%%%%%%%%%%%%%%%%%%% Packages %%%%%%%%%%%%%%%%%%%
\usepackage{xifthen}
\usepackage{xparse}


%%%%%%%%%%%%%%%%%%% PGF Styles%%%%%%%%%%%%%%%%%%%
\pgfplotsset{
  abs log x ticks/.style={
      xticklabel={
        \pgfkeys{/pgf/fpu=true}
        \pgfmathparse{exp(\tick)}%
        \pgfmathprintnumber[fixed relative, precision=3]{\pgfmathresult}
        \pgfkeys{/pgf/fpu=false}
      }
  },
  abs log y ticks/.style={
      yticklabel={
        \pgfkeys{/pgf/fpu=true}
        \pgfmathparse{exp(\tick)}%
        \pgfmathprintnumber[fixed relative, precision=3]{\pgfmathresult}
        \pgfkeys{/pgf/fpu=false}
      }
  }
}

%%%%%%%%%%%%%%%%%%% Commands %%%%%%%%%%%%%%%%%%%
\newcommand{\ifequals}[3]{\ifthenelse{\equal{#1}{#2}}{#3}{}}
\newcommand{\case}[2]{#1 #2} % Dummy, so \renewcommand has something to overwrite...
\newenvironment{switch}[1]{\renewcommand{\case}{\ifequals{#1}}}{}

\newcommand{\symbolicsStrat}{%
symbolic x coords={,
    Problem,
    10,
    100,
    1000,
    10000,
    100000,
}
}

\newcommand{\symbolic}[1]{symbolic x coords={#1}}
\NewDocumentCommand\lnm{O{y}O{2}}{%
\begin{switch}{#1}%
    \case{x}{xmode=log, log basis x={#2}}%
    \case{y}{ymode=log, log basis y={#2}}%
\end{switch}%
}
\newcommand{\plotData}[2]{
    \addplot table [y={#1}] {#2};
    \addlegendentry{#1};
}

\newcommand{\plotUnmarkedData}[2]{
  \addplot+[mark=none] table [y={#1}] {#2};
  \addlegendentry{#1};
}

\NewDocumentCommand\optPlotData{O{}mm}{
  \addplot+[#1] table [y={#2}] {#3};
  \addlegendentry{#2};
}

\newcommand{\plotDataWithError}[2]{
    \addplot+[error bars/.cd, y dir=both,y explicit] table [y={#1}, y error={#1 Error}] {#2};
    \addlegendentry{#1};
}

\newcommand{\plotDataWithErrorAndLegend}[3]{
    \addplot+[error bars/.cd, y dir=both,y explicit] table [y={#1}, y error={#1 Error}] {#2};
    \addlegendentry{#3};
}

\newcommand{\logmode}[1]{\IfBooleanTF{#1}{log}{normal}}
\newcommand{\logbase}[1]{\IfBooleanTF{#1}{2}{1}}

% O/o for optional, M/m for mandatory, s for star
% s #1 BooleanTrue if star, BooleanFalse otherwise
% O #2 PGFPlot options
% O #3 Y label
% O #4 X label
% m #5 Caption
% m #6 Label
% m #7 Plots
\NewDocumentCommand\lineChart{sO{}O{Logarithmic Run Time (ns)}O{Lower is better}mmm}{%

\begin{figure}[H]
\makebox[\textwidth][c]{
\tikzsetnextfilename{#6}
\begin{tikzpicture}
    \begin{axis}[
        ylabel={#3},
        xlabel={#4},
        width=1.2\textwidth,
        height=8cm,
        ymajorgrids,
        xtick=data,
        #2,
        ymode/.expand once=\logmode{#1},
        log basis y/.expand once=\logbase{#1},
        xticklabel style={rotate=45, anchor=east},
        legend columns = -1,
        legend style={
            draw=none,
            at={(0.5,1.05)},
            anchor=south,
            column sep=1ex,
            font=\tiny
        }
    ]
    #7
    \end{axis}
\end{tikzpicture}}
\caption{#5}
\label{fig:#6}
\end{figure}
}

% O/o for optional, M/m for mandatory, s for star
% s #1 BooleanTrue if star, BooleanFalse otherwise
% O #2 bar width
% O #3 PGFPlot options
% O #4 Y label
% O #5 X label
% m #6 Caption
% m #7 Label
% m #8 Plots
\NewDocumentCommand\barChart{sO{12}O{}O{Run Time (ns)}O{Lower is better}mmm}{%
    \begin{figure}[H]
    \makebox[\textwidth][c]{
    \tikzsetnextfilename{#7}
    \begin{tikzpicture}
        \begin{axis}[
            ylabel={#4},
            xlabel={#5},
            width=1.2\textwidth,
            height=8cm,
            ymajorgrids,
            xtick=data,
            ybar=0.4pt,
            bar width=#2pt,
            xticklabel style={rotate=45, anchor=east},
            #3,
            ymode/.expand once=\logmode{#1},
            log basis y/.expand once=\logbase{#1},
            legend columns = -1,
            legend style={
                draw=none,
                at={(0.5,1.05)},
                anchor=south,
                column sep=1ex,
                font=\tiny
            }
        ]
        #8
        \end{axis}
    \end{tikzpicture}}
    \caption{#6}
    \label{fig:#7}
\end{figure}
}


%\pgfplotstableread[col sep = comma]{data/sequential.csv}\sequentialData
\pgfplotstableread[col sep = comma]{data/concurrent.csv}\concurrentData
\pgfplotstableread[col sep = comma]{data/sequential.csv}\sequentialData
\pgfplotstableread[col sep = comma]{data/seq-avg.csv}\sequentialAverageData
\pgfplotstableread[col sep = comma]{data/ai-avg.csv}\aiAverageData

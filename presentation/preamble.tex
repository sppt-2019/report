%%%%%%%%%%%%%%%%%%% Packages %%%%%%%%%%%%%%%%%%%
\usepackage{amsmath}
\usepackage{amsfonts}
\usepackage{amssymb}
\usepackage{listings}
\lstset{basicstyle=\ttfamily}
\usepackage{syntax}
\usepackage{tikz}
\usetikzlibrary{arrows,arrows.meta, fit, backgrounds, positioning, automata, shapes}
\usepackage{pifont}
\usepackage{svg}
\usepackage{framed}
\usepackage{nameref}
\usepackage{syntax}
\usepackage[cache, outputdir=build]{minted}
\usepackage[export]{adjustbox}
\usepackage{xifthen}
\usepackage{xparse}
\usepackage{xspace}
\usepackage{array}
\usepackage[autostyle]{csquotes}
\usepackage[backend=biber, style=ieee, urldate=long]{biblatex} % biblatex - bibliography tool %
\bibliography{../../00-preliminary/bib}

% Table macro.
% 1: Table Latex
% 2: Caption
% 3: Label
\definecolor{rowgray}{HTML}{E0E0E0}
\newcommand{\makeTable}[3]{
    \rowcolors{1}{}{rowgray}
    \begin{table}[H]
        \sisetup{round-mode=places}
        \alignCenter{
        \begin{tabular}
            #1
        \end{tabular}}
        \caption{#2}
        \label{tab:#3}
    \end{table}
    \rowcolors{0}{}{}
}

% Table macro.
% 1: Table Latex
% 2: Caption
% 3: Label
\newcommand{\makeTablePB}[3]{
    \centering
    \rowcolors{1}{}{rowgray}
    \begin{longtable}
        #1
        \caption{#2}
        \label{table:#3}
    \end{longtable}
    \rowcolors{0}{}{}
}

\newcommand{\tableref}[1]{Table \ref{tab:#1}}
\newcommand{\tabref}[1]{\tableref{#1}}
\newcommand{\figureref}[1]{Figure \ref{fig:#1}}
\newcommand{\figref}[1]{Figure \ref{fig:#1}}
\newcommand{\coderef}[1]{Code \ref{code:#1}}
\newcommand{\lstref}[1]{Listing \ref{lst:#1}}
\newcommand{\secref}[1]{Section \ref{sec:#1}}
\newcommand{\chapref}[1]{Chapter \ref{chap:#1}}
\newcommand{\charef}[1]{\chapref{#1}}
\newcommand{\appendixref}[1]{Appendix \ref{app:#1}}
\newcommand{\apxref}[1]{\appendixref{#1}}
\newcommand{\appref}[1]{\appendixref{#1}}
%\newcommand{\lineref}[1]{Line \ref{line:#1}}

%\newcommand{\acrfull}[1]{\acrlong{#1} (\acrshort{#1})}

\newcommand{\dquote}[1]{``{#1}''}
\newcommand{\dquotes}[1]{\dquote{#1}}
\newcommand{\squote}[1]{`{#1}'}
\newcommand{\squotes}[1]{\squote{#1}}

\newcommand{\quoteWithCite}[3]{
  \begin{quotation}
  \noindent \textit{\dquote{#1}}
  \begin{flushright}
    -#2\cite{#3}
  \end{flushright}
  \end{quotation}
}

\newcommand{\quoteParticipant}[3]{
  \begin{quotation}
  \noindent \textit{\dquote{#1}}
  \begin{flushright}
    -#2, \appref{#3}
  \end{flushright}
  \end{quotation}
}

\newcommand{\quoteWithoutCite}[2]{
  \begin{quotation}
  \noindent \textit{\dquote{#1}}
  \begin{flushright}
  -#2
  \end{flushright}
  \end{quotation}
}

\definecolor{morell}{rgb}{213,0,96}
\newcommand{\tmc}[1]{\todo[color=morell]{#1}\xspace}
\newcommand{\btc}[1]{\todo[color=red!75]{#1}\xspace}
\newcommand{\tmcc}[1]{\todo[color=cyan]{#1}\xspace}
\newcommand{\needcite}{\todo{Citation needed}\xspace}
\newcommand{\inline}[1]{\todo[inline]{#1}}
\newcommand{\ttt}[1]{\texttt{#1}}
\newcommand{\m}[1]{\ttt{#1}}
\newcommand{\unity}{Unity\xspace}
\newcommand{\unreal}{Unreal Engine\xspace}
\newcommand{\cryengine}{CryEngine\xspace}

%%%%%%%% Column types
\newcolumntype{L}[1]{>{\raggedright\let\newline\\\arraybackslash\hspace{0pt}}m{#1}}
\newcolumntype{C}[1]{>{\centering\let\newline\\\arraybackslash\hspace{0pt}}m{#1}}
\newcolumntype{R}[1]{>{\raggedleft\let\newline\\\arraybackslash\hspace{0pt}}m{#1}}
\newcolumntype{P}[1]{>{\raggedright\arraybackslash}p{#1}}

\newcommand\rowEnd{\\\hline}

\newcommand{\metasheep}{
    \begin{figure}[H]
        \begin{center}
            {\LARGE \setstretch{0.75}
                \hspace{5em}        \ttt{,ww}\\
                \hspace{1.5em}      \ttt{wWWWWWWW\textbackslash\_)}\\
                \hspace{1em}        \ttt{`WWWWWW'}\\
                \hspace{1em}        \ttt{II}
                \hspace{1ex}        \ttt{II}\\
            }
            \caption{{\color{red} \ttt{"Beep beep, I'm a meta sheep. Replace me with meta text!"}}}
        \end{center}
    \end{figure}
}

\newcommand\alignCenter[1]{\makebox[\textwidth][c]{#1}}
\newcommand\alignRight[1]{\makebox[\textwidth][r]{#1}}
\newcommand\alignLeft[1]{\makebox[\textwidth][l]{#1}}

\newcommand\monitor[1]{\hspace*{-2.6em}\ttt{Monitor: }#1\\}
\newcommand\participant[1]{\hspace*{-5em}\ttt{Participant: }#1\\}
\newcommand\partic[1]{\participant{#1}}

\newcommand{\cpp}{C++\xspace}
\newcommand{\csh}{C\#\xspace}
\newcommand{\cs}{\csh}
\newcommand{\fsh}{F\#\xspace}
\newcommand{\fs}{\fsh}

\newcommand{\fsinline}[1]{\mintinline{fsharp}{#1}}
\newcommand{\fsline}[1]{\fsinline{#1}}
\newcommand{\csinline}[1]{\mintinline{csharp}{#1}}
\newcommand{\csline}[1]{\csinline{#1}}

%%% Method Name Macros

\newcommand{\champagne}{Champagne Prototyping\xspace}
\newcommand{\discount}{Discount Method for Language Evaluation\xspace}
\newcommand{\ida}{Instant Data Analysis\xspace}
\newcommand{\cognitive}{Cognitive Dimensions\xspace}
\newcommand{\attention}{Attention Investment Model\xspace}
\newcommand{\attentions}{Attention Investment Models\xspace}
\newcommand{\expert}{Expert Review Method\xspace}


\newcommand{\diagram}[2]{
    \makebox[\textwidth][c]{
        \begin{tikzpicture}
        #2
        \end{tikzpicture}
    }\label{fig:#1}
}

\chapter*{Preface\markboth{Acknowledgements}{Acknowledgements}}\label{ch:preface}
\addcontentsline{toc}{chapter}{Preface}
The field of game development is subject to many opinions from the industry and is often backed by little scientific evidence. We will therefore need to cite pages that are not \dquote{traditionally scientific} in this report, examples of which are Quora and StackOverflow. These sources are of questionable quality and need not reflect the broad opinion on the game development industry. However, we include them here to indicate that \dquote{some} game developers share the expressed belief.

\section*{Resume}
This project examines the claims of John Carmack and Tim Sweeney. These two notable game development figures claims than increased used of functional programming in game development would be beneficial. The two gurus suggest using functional programming in different ways; Carmack argues that developers should adhere to a functional style in any language, whereas Sweeney suggests the introduction of a new pure functional language with explicit effects typing and lenient evaluation.

We decided to examine why functional programming is not yet used by AAA game engines, such as Unity and Unreal Engine. This was first examined by conducting a usability evaluation. The participants in the evaluation were profession game developers from Aalborg, who had experience in Unity and C\#. During the test sessions the participants would implement one or more task, which was designed to resemble an aspect of gameplay programming. Half of the tasks were to be implemented in F\# and the rest in C\#. A total of eight tasks were designed for the experiment, which were sorted into four categories. The purpose of the categories was to allow a side-by-side comparison of F\# and C\# under similar conditions. After the session a short interview was conducted to allow the participants to express whether or not they found the use of functional programming beneficial in game development. The participants generally agreed that the use of functional programming would prove many benefits, such as more modular code and immutability, but still were reluctant to use it in practice. Their primary arguments against F\# was that the cost of learning the language would be too high, compared to the provided benefits.

After learning that game developers were not keen on switching, we decided to examine the performance impact of using F\# instead of C\#. We did so with a particular focus on concurrency, to learn whether concurrent code F\# was more performant than in C\#. In most cases, F\# runs slightly slower than C\#. Furthermore, we found that F\#'s Async Workflows parallelisation strategy was fragile, as using \m{Async.Parallel} instead of \m{Async.StartChild} can result in a massive performance degradation. In the best case circumstances, the \m{Task} parallelisation of C\# and Async Workflows of F\# seems to fare equally well. We also conducted a test in Unity, which measures the framerate given an increasing number of \m{MonoBehaviour}s. We found that \m{MonoBehaviour}s written in F\# was a little less performant compared to C\#. Finally, the performance of the \gls{FRP} system, which was written as part of this project, was much slower still. The reason for this was that the current implementation is simple and suboptimal, meaning that each \m{FRPBehaviour} is a full-blown \gls{FRP} system with conditions-checking and event-dispatching.

Finally, we put out findings into the perspective of modern game development and discuss why the benefits provided by F\# might be less important to the developers than productivity.

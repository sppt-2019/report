\cleardoublepage
{\selectlanguage{english}
\pdfbookmark[0]{English title page}{label:titlepage_en}
\aautitlepage{%
  \englishprojectinfo{
    \projectFullTitle %title
  }{%
    Scientific Theme: Language evaluation, game development, functional programming, interpreted languages, evaluation strategey %theme
  }{%
    Spring Semester 2019 %project period
  }{%
    \group % project group
  }{%
    %list of group members
    Thomas Gwynfryn McCollin \smallbreak
    Tobias Morell
  }{%
    %list of supervisors
    Bent Thomsen
  }{%
     Digital distribution only % number of printed copies
  }{%
    \today % date of completion
  }%
}{%department and address
  \textbf{Software Engineering}\smallbreak
  Aalborg University\smallbreak
  \href{Department of Computer Science}{http://www.cs.aau.dk/}
}{% the abstract
This project examines the use of functional programming in gameplay programming. The claims of two notable game development gurus, John Carmack and Tim Sweeney, form the foundation of this research.% Carmack argues that functional programming should be used whenever convenient. Sweeney argues that a new language should be introduced which is pure and has effects typing constructs.
We use the game engine Unity as a host for the research. Carmack's claim is represented by C\# and Sweeney's by F\#.\newline

We first examine experienced game developers attitude towards the claims via a usability evaluation. We found that the participants were able to write more concise and modular code in F\#, but still were reluctant to use it in practise. In need of a stronger incentive we turned to a performance study, intended to measure if concurrent code in F\# is more performant than concurrent code in C\#. We found that F\# is slightly slower than C\# in most cases. Finally we put the observed benefits of F\# into the context of modern game development practices to examine why those benefits are not appealing to experienced gameplay programmers. 
}}
\clearpage

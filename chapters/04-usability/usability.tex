\chapter{Usability}
In this chapter we research usability of functional-style programming in gameplay programming. We first use the cognitive dimensions framework to discuss our transition from C\# to F\# and compare the two languages.

Afterwards we conduct a usability test of F\# in Unity. The tests are formulated based on the Champagne Prototyping methodology (see \secref{champagne}). For this test the Unity game engine was extended, via a plugin to support programming in F\#\cite{fsharp2019plugin}. In addition, a \gls{FRP} module was implemented in F\# for Unity. These two extensions served as the prototype under test. The test cases in the following section were formulated as the experimental tasks. Finally a number of domain experts, in this case experienced Unity developers, were selected as participants.

The usability test was intended to examine the viability of using functional programming in commercial game engines. Again, we drew on the approaches presented by the game development gurus; Carmack and Sweeney. Carmack's opinion of using functional-style programming whenever convenient is represented by C\#, which has many functional-style constructs. Sweeney's opinion of a pure functional language with explicit effects typing constructs is (somewhat) represented by F\#. F\# is pure by default, but it has the \ttt{mutable} keyword, which allows the programmer to declare mutable variables. These variables are assignable using the \ttt{\textless-} operator, which requires the programmer to explicitly indicate a non-pure action.

\section{Cognitive Dimensions of F\# and C\#}
In order to ascertain the usability of \fs in comparison to \cs, we have conducted an analysis based on the cognitive dimensions framework\cite{green1996usability}. \cs is the gameplay programming language available in the Unity game engine, therefore, if \fs is to usurp this position the advantages and disadvantages should be made clear. We conduct this analysis on the basis of our own experience with \fs. Both authors of this report have prior experience with \cs programming in Unity and no experience with \fs.

\subsubsection{Abstract Gradient}
The abstract gradient is measured from abstraction hating, through abstraction tolerant, to abstraction loving. The abstractions measured are the notations' ability to group elements and refer to them as a single entity. Most modern textual-programming languages make extensive use of abstraction and functional languages even more so\cite{hudak1989conception}. \fs is a functional-programming language with object-oriented features allowing for extensive abstractions.

On the other hand \cs is an object-oriented language which supports functional features. This means that \cs also supports extensive abstraction. The main difference lies in the fact that \cs is object-oriented programming first and \fs is functional programming first. Functional programming tends to make use of abstraction more frequently, however that does not mean that other paradigms do not use abstraction\needcite.

Considering the high level of abstraction in both languages they will both be considered abstraction loving in this report. \cite{bendersky:abstraction} presents a discussion, underlining that functional languages generally rely on function abstract, treating types as thin data container, whereas object oriented languages rely on data abstraction, where functionality is associated with the classes. The author notes that this may surface as making it harder add new types in functional languages and adding functionality in object oriented languages.\btc{Tennet}

\subsubsection{Closeness of Mapping}
The measure of how close to the problem domain a language can get is called the closeness of mapping. In order to solve a real problem, the problem must be expressible in the language and the closer the language is to the real world, the easier it is to express\cite{green1996usability}. Textual programming languages are abstractions over the real problem domain and therefore often does not map directly to the domain.

Both languages have mechanisms to model the problem domain. In object-oriented programming, the world is represented as objects and the objects are abstracted over via classes\cite{kindler2011object}. Functional programming models the problem as behaviour (functions) which are applied to data\cite{hughes1989functional}. The advantage of the object-oriented approach is that the object abstraction comes quite close the problem domain.

\quoteWithCite{The object expresses the user's view of reality [...]}{Object Oriented Analysis \& Design}{mathiassen2000object}

This approach is in contrast to the functional paradigm which models reality mathematically. This approach is not as close as the object model, however mathematical modelling of the world is widespread in many different fields of study.

\quoteWithCite{Typically the main function is defined in terms of other functions, which in turn are defined in terms of still more functions, until at the bottom level the functions are language primitives. These functions are much like ordinary mathematical functions [...]}{John Hughes}{hughes1989functional}

The abstraction models of the languages are the tools used by the programmers to model the world. \cs uses a model, which lends itself more to closeness of mapping, but both languages make use of custom types and naming which allow programmers to mold their programs in accordance with the problem domain. Additionally both languages support each other's modelling approach. An example of this can be seen in \lstref{tree-imps}, where a recursive class in \cs can be implemented via custom types in \fs.

\begin{listing}[H]
  \begin{minted}{fsharp}
type Talent(strength, intellect, agility) =
  member val Strength = strength with get, set
  member val Agility = agility with get, set
  member val Intellect = intellect with get, set

type Tree =
| Node of TalentValue:Talent * Children:Tree list
| Leaf of TalentValue:Talent

let rec foldTree folding init tree =
  match tree with
  | Node (t, _, c) ->
    let f = folding t init
    let cf =
      c
      |> List.map (foldTree folding init)
      |> List.fold folding init
    folding f cf
  | Leaf (t,_) ->
    folding t init

let sumPickedNodes root =
  foldTree (fun t1 t2 ->
    Talent(t1.Strength + t2.Strength, t1.Intellect + t2.Intellect, t1.Agility + t2.Agility)) (Talent(0,0,0)) root
  \end{minted}
  \begin{minted}{csharp}
public class MyTalent
{
  public int Strength;
  public int Agility;
  public int Intelligence;

  public List<MyTalent> SubTalents = new List<MyTalent>();

  public MyTalent SumTalents() {
    var t = new MyTalent(Strength, Agility, Intelligence);
    if(SubTalents?.Count == 0)
      return t;

    foreach (var child in SubTalents) {
      var childTalentValues = child.SumTalents();
      t.Strength += childTalentValues.Strength;
      t.Agility += childTalentValues.Agility;
      t.Intelligence += childTalentValues.Intelligence;
    }
    return t;
  }
}
  \end{minted}
  \caption{Talent tree data structure and walker implementations (F\# on top, C\# below).}
  \label{lst:tree-imps}
\end{listing}

The example in \lstref{tree-imps} implements a class and a discriminated union in \fs, which together implement the behaviour of the \cs class. The \fs approach has separated the tree from the talent, where in the  \cs solution the tree emerges from the recursive nature of the type. The \cs solution is closer to the problem domain, but the \fs solution is closer to the mathematical concept of trees.\tmc{Skal vi skrive noget om at man også godt kunne lave en generisk træ-struktur i C\#?}

\subsubsection{Consistency}
% par 2 43:0 32:10, pipe operator
In the cognitive dimensions framework consistency is the coherence between the language designer's understanding and the language user's intuition of the language\cite{green1996usability}. This does not mean that consistency is the difference in language knowledge, but rather the difficulty of extrapolating behaviour and syntax of language features based on knowledge of a subset of the language or other language features.

\fs uses a strict type system which infers types. This feature allows the programmer to omit explicit typing while still gaining the benefits of it. In some cases the type inference can cause confusion or act in an unexpected way, as when a \ttt{int16} value is used in the declaration of a \ttt{int} value. In \lstref{type-incompat} an example of this can be seen. Line 2 gives an error because an \ttt{int} literal and an \ttt{int16} name binding are multiplied. This behaviour is consistent with \fs's rules, but is surprising for programmers who are versed in C-style languages.

\begin{listing}[H]
\begin{minted}{fsharp}
let x = 10s
let y = 2 * x
\end{minted}
\caption{Type Incompatibility}
\label{lst:type-incompat}
\end{listing}

Naming conventions can present consistency difficulties for some languages. An example of this are the type modules in F\#, such as \ttt{List}. These modules supply helper functions for working with a particular type, an example of which is \ttt{List.append}. In C\#, equivalent functionality would have been placed as instance methods on said types. This may cause some confusion for C-family programmers, as they will not find the functionality they seek, where they're used to.

In addition to naming conventions causing confusion, \ttt{list}s have another problem. \fs and \cs share the .NET runtime and can therefore use each other's language features. While this is an advantage, it also presents some disadvantages, most notably that \fs lists and \cs lists are not the same type. This clashes with programmer expectations and converting to the correct list type can be surprisingly difficult, which we have demonstrated in \lstref{list-conv}.

\begin{listing}[H]
\begin{minted}{csharp}
private static List<T> GetParams<T>(Microsoft.FSharp.Collections.List<T> parameters)
{
    return new List<T>(parameters);
}
\end{minted}
\caption{Conversion from \fs List to \cs List}
\label{lst:list-conv}
\end{listing}

In functional programming languages function signatures can often be specified by the programmer, to help the compiler catch unexpected behaviour. This is also possible in \fs, however in an unexpected manner. Function signatures in \fs are reported using the Hindly-Milner type system's syntax\cite{fsharp:type:inference}. However, when the programmer attempts to declare the function signature manually, they cannot use the same syntax. Instead a Python-like syntax is used. An example can be seen in \lstref{fun-sig}.

\begin{listing}[H]
\begin{minted}{fsharp}
// reported function signature
val add: int -> int -> int

// function definition and signature
let add x y = x + y
// or alternatively with explicit types
let add (x:int) (y:int) : int = x + y
\end{minted}
\caption{Function Signatures}
\label{lst:fun-sig}
\end{listing}

In \fs lambda expressions are denoted using the \ttt{fun} keyword and \ttt{-\textgreater} operator. The use of \ttt{fun} vs. the use of \ttt{func} may initially be confusing for programmers, but is quickly learned. After the initial confusion, the feature is consistent with the rest of \fs, as lambda expressions are defined like functions. This is not the case in \cs, where lambda expressions are defined using the \ttt{=\textgreater} operator. This is not consistent with the rest of \cs, because \cs does not use a function signature similar to the Hindly-Milner type system.

\fs has two primary collection types: lists and arrays. The array-collection type can provide some benefits when looking up elements by index.  However, when looking up an element by index, dot notation is used to call the \ttt{[]} function, thus a lookup becomes \fsinline{array.[0]}. This clashes with expectations of a C-family programmer where \csinline{array[0]} is the norm.

While \fs presents some consistency problems, they're are consistent within the language. This indicates that the problems may be experienced more by novice programmers and that they dissipate with experience.

\subsubsection{Diffuseness/Terseness} %Antal af paranteser: Participant 6, 1:01:00 cirka
This dimension measures the conciseness of a notation system on a scale from terse, meaning too brief, to diffuse, meaning not brief enough. The golden middle ground is referred to as concise. This measure is affected by the symbols used for operators as well as the notation system's naming conventions. If the notation is too brief, understanding its meaning may be quite difficult and small changes can have large consequences. Whereas notations that are too verbose, cannot be viewed on a single screen and therefore are more difficult to overview.

To compare the conciseness of both languages, two solutions to a problem will be examined. The problem consists of calculating three sums based on properties of a list of objects. The objects are given and the sums must be printed to the console. In \lstref{cs-armour} the \cs solution can be seen. The method takes a collection of \m{Item}s and iterating over them, keeping a running tally of three sums. Once all objects have been summed, the results are printed to the console.

\begin{listing}[H]
\begin{minted}{csharp}
public void Solution1(IEnumerable<Item> Armour)
{
  int totalAgi = 0;
  int totalStr = 0;
  int totalInt = 0;

  foreach (var item in Armour)
  {
    totalAgi += item.Agility;
    totalStr += item.Strength;
    totalInt += item.Intellect;
  }
  Debug.Log($"Exercise 1\n\t" +
            $"Agility: {totalAgi}\n\t" +
            $"Strength: {totalStr}\n\t" +
            $"Intellect: {totalInt}");
}
\end{minted}
\caption{First Person Movement Controller \cs}
\label{lst:cs-armour}
\end{listing}

The approach taken in \fs is somewhat different, the solution can be seen in \lstref{fs-armour}. Instead of iterating over the given array, a map reduce approach is used. On line 8 the array is piped into a map which deconstructs each \m{Item} into a triple. The triples are then piped into a reduce function which calls the sum function defined on line 1. In the start function on line 16, the sums are computed and printed to the console.

The \fs is slightly longer then the \cs implementation, this is due to dividing the functionality into several smaller functions. This lengthened the implementation of the first solution, but reduced the overall length of the code. The \cs code ended up being 35 lines longer than the \fs code. The full examples can be seen in \appref{terse-diff-comp}.

The most prominent syntactic differences between \cs and \fs are the operators, scope delimiters and line end delimiters. In \cs blocks are denoted using the \m{\{} and \m{\}} symbols, where in \fs indents are used. In addition statements are terminated using a \m{;} in \cs, where a newline character is used in \fs. These differences mean that \fs uses fewer symbols in general than \cs, however some programmers find the "light" syntax more difficult to read \needcite.

Another difference that greatly affects terseness/diffuseness is approach to code reuse the languages use. In \cs inheritance and the Composite Pattern \needcite are often used \needcite. In \fs function composition or chaining, is often used to increase code reuse. Both approaches reduce the codebase, but do so in different ways.

\begin{listing}[H]
\begin{minted}{fsharp}
let sum (triplet1:int*int*int) (triplet2:int*int*int) =
  let (a1, b1, c1) = triplet1.Deconstruct()
  let (a2, b2, c2) = triplet2.Deconstruct()
  (a1+a2,b1+b2,c1+c2)

[...]

let totalStats (armour:Item[]) =
  armour
  |> Array.map (fun a -> (a.Agility, a.Intellect, a.Strength))
  |> Array.reduce (fun acc elm ->
    sum acc elm)

[...]

member this.Start() =
  let i = ItemStore.AllItems()
  let (agi, int, str) = totalStats(i)
  Debug.Log("Agility: " + agi.ToString())
  Debug.Log("Intellect: " + int.ToString())
  Debug.Log("Strength: " + str.ToString())

 [...]
\end{minted}
\caption{First Person Movement Controller \fs}
\label{lst:fs-armour}
\end{listing}

\subsubsection{Error-proneness}
Errors produced by the programmer fall into one of two categories, either the error is a slip: instances where the programmer knows what to do, but did something else by accident, or a mistake: when a programmer makes a logical error. These errors can be exacerbated by language features.

According to \cite{green1996usability}, textual programming languages are inherently more error-prone. The given examples are implicit declaration, line-ends and delimiters. Implicit are not applicable to either \cs or \fs, however line-ends are used in both languages. \cs uses \m{;} to denote line-ends where \fs uses newlines. A C-family programmer may find it difficult to overview \fs code for this reason. However, traditionally in \cs a \m{;} isn't a line-end rather a statement-end and is followed by a newline anyway \needcite. The \fs approach using newlines and indentation to delimit scopes and lines uses less characters and promotes more readable code \needcite. An example of this can be seen in \lstref{con-comp}.

\begin{listing}[H]
\begin{minted}{csharp}
class Person(string name, DateTime, birthday)
{
  public string Name { get; } = name;
  public DateTime Birthday { get; } = birthday;
  public int Age() =>
    DateTime.Today.Subtract(birthday).Days / 365;
}
\end{minted}
\begin{minted}{fsharp}
type Person(name: string, birthday: DateTime ) =
  member this.Name =
    name
  member this.Birthday =
    birthday
  member this.Age() =
    let daysDiff = DateTime.Today.Subtract(birthday).Days
    daysDiff / 365
\end{minted}
\caption{Conciseness Comparison \cite{wlaschin2017FsharpForCsharpProgrammers}}
\label{lst:con-comp}
\end{listing}

The strong type system used in \fs can also cause unexpected errors. While the system prevents some errors down the line, C-family programmers would expect type coercion to assist with operations on integer values of different sizes. An example of this can be seen in \lstref{coer-lack}. The function on line 1 takes a 16-bit integer and returns its square. The programmer then attempts to calculate the square of the integer literal 10, however, 10 is not a 16-bit integer and an error is thrown. This causes initial errors, but may prevent type errors later in development \needcite.

\begin{listing}[H]
\begin{minted}{fsharp}
let sqr (x: int16) = x * x

let value = sqr 10
\end{minted}
\caption{Lack of Type Coercion}
\label{lst:coer-lack}
\end{listing}

Programmers may inadvertently change the parameters of a function by separating parameters using commas, see \lstref{para-err}. Often this is still valid \fs and compiles, however the function now takes a single tuple parameter. This mistake can occur without the programmer noticing any difference, before they have to invoke the function. At this point the invocation has changed from \fsline{add 2 4}(two parameters) to \fsline{add (2, 4)}(tuple parameter). Initially, programmers may have no issue with this behaviour, until the function is on lists via list operations and each element needs to be converted to a tuple.

\begin{listing}[H]
\begin{minted}{fsharp}
// Multiple parameters
let add x y = x + y

// Single parameter
let add (x, y) = x + y
\end{minted}
\caption{Parameter Error}
\label{lst:para-err}
\end{listing}

\subsubsection{Hard Mental Operations}
The hard mental operations dimension defines how often incomprehensible expressions occur in the code. Hard mental operations often occur in conjunction with boolean expressions\cite{green1996usability}.

Boolean expressions are expressed similarly in C\# and F\#, with the only exception that C\# uses \ttt{!} to negate expressions, whereas F\# uses the \ttt{not} function. We argue that the use of the \ttt{not}-function reduces the perceived ambiguity, because it forces the programmer to parenthesise the expression. We have illustrated an example in \lstref{hard:mental:operations}. In this example the reader may incorrectly assume that the \ttt{\&\&} operator is evaluated before \ttt{!}, and that the expression is evaluated as  \csinline{!(expr1 && expr2)}. That is not the case in F\#, where the parentheses indicate the order of evaluation.

\begin{listing}[H]
    \begin{minipage}{.45\textwidth}
        \begin{minted}{fsharp}
if(!expr1 && expr2) {
    //[...]
}
        \end{minted}
    \end{minipage}
    \hfill
    \begin{minipage}{.45\textwidth}
        \begin{minted}{csharp}
if (not expr1) && expr2 then
    //[...]
        \end{minted}
    \end{minipage}
\caption{Hard mental operations illustrated using boolean expressions in C\# and F\#.}
\label{lst:hard:mental:operations}
\end{listing}

In F\# it is optional for the programmer to indicate types when he is writing functions. Sometimes it may be necessary to verify that the compilers inference is correct by looking at the deduced function signature. We argue that this may present hard mental operations in both languages, as the function signatures will quickly get incomprehensible as the number of arguments grow. Take for example the function signature of the \ttt{ReactTo} function that we wrote as part of the \gls{FRP} plugin for Unity: 
\newline
\fsinline{member FRPBehaviour.ReactTo : event:FRPEvent * condition:('T0 -> bool) * handler:('T0 -> unit) -> unit}.\newline
We should underline that this function uses tupled arguments, because we wanted to overload it with a function to unconditionally react to events. But even without tupled arguments the definition would have been:\newline
\fsinline{member FRPBehaviour.ReactTo : event:FRPEvent -> condition:('T0 -> bool) -> handler:('T0 -> unit) -> unit}.\newline
In C\# the equivalent would have been:\newline
\csinline{public void ReactTo(FRPEvent event, Func<T, bool> condition, Action<T> handler)}\newline
Whether one or the other is easier to comprehend than the other is a matter of opinion. The problem is more prominent in F\#, however, due to the type inference in the compiler.
\subsubsection{Hidden Dependencies}
Hidden dependencies discuss how many relationships there are between components, that are not visible from atleast one of the components. We discuss several problems in this section, but we must underline that some of these problems can be mitigated by using an \gls{IDE}, as they often allow programmers to trace dependencies.

The problem of hidden dependencies is prominent in both C\# and F\#, though in two different flavours. We have listed an example in \lstref{hidden:dependencies} that showcase the problems. In F\# the problem is present on the function level. This is because F\# programmers are allowed to write functions that are directly nested in a module. These functions may be imported into another module or namespace with the \ttt{open} keyword. Given that a name of a function does not collide, the function may be refered to without the use of its fully qualified name. In C\# the problem occurs because programmers are allowed to reference code in base classes without using their fully qualified name. One example of this in Unity is that any \ttt{MonoBehaviour} may directly call \ttt{Destroy}, which is actually a static method on the \ttt{GameObject}-class. This may make it more difficult to distinct between static and non-static methods and \dquote{hide} the base class from the reader. In F\# programmers are required to give fully qualified names when dealing with classes.

\begin{listing}[H]
    \begin{minipage}{.50\textwidth}
        \begin{minted}{csharp}
class Base {
    protected static string Method() {
        return "Method";
    }
}

class Inherited : Base {
    private string GetString() {
        return Method() + " is the string";
    }
}
        \end{minted}
    \end{minipage}
    \hfill
    \begin{minipage}{.40\textwidth}
        \begin{minted}{fsharp}
//Imagine this is in another file
module X =
    let function () =
        "function"

open X
module Y =
    let function2 () =
        function() + "2"
        \end{minted}
    \end{minipage}
\caption{Hidden dependencies in function/method calls in C\# and F\#.}
\label{lst:hidden:dependencies}
\end{listing}

In both languages the problem of function/method hiding may occur. An example of this is if a base class defines a \ttt{virtual} method in C\# or an \ttt{abstract} method with default implementation in F\#. This method may be hidden further down the inheritance tree by implementing a function with the same name without using the \ttt{override} keyword. This will give a warning at compile-time, which can be removed by supplying the \ttt{new} keyword in front of the method that hides the existing implementation.

% goto statements in C#
C\# has a modified version of the \ttt{goto}-statement, which traditionally allow programmers to jump to labels anywhere in the source code. In C\#, however, the jumps are restricted to either a label in the same scope or in an enclosing scope\cite{csharp:goto}. Nevertheless, it may be hard to comprehend the exact target of a \ttt{goto} if it is deeply nested in multiple loops. According to a StackOverflow discussion\cite{goto:stack:overflow}, it seems that the \ttt{goto} statement sees limited use in practise.

% mutability
% Unity Editor reference assignment
The problem of hidden dependencies may also occur if a component is dependant on a global variable. In games it is common to see such types of dependency\cite{blow2004game, guana2015building, nystrom2014game}. This problem is easier to mitigate in F\# because the global variable is per default immutable and the developer has to explicitly indicate if he wants a mutable variable. In C\# it's easier to make slips, as everything is mutable per default. In Unity this problem is also present in-between components, as it's a common pattern to declare a field on a class that references some component and thereby assign that field from Unity's Inspector\cite{unity:inspector:assignment}. This has the consequence that there is no way of knowing which components depend on each other by inspecting the source code. Furthermore, it is not possible to trace dependencies backwards in Unity without writing custom Editor scripts\cite{unity:dependencies:backwards}.
\subsubsection{Premature Commitment}
Premature commitment describes how much guess-ahead the programmers has to make when he is programming in a given language. 

In F\# there is a fixed ordering when defining types, that enforces all name bindings (\ttt{let}) to be declared before members. This is similar to the problem of \textit{Commitment to layout} presented in \cite{green1996usability}. Luckily, these declarations may quickly be moved around in F\# source code by cutting and pasting. This problem is not present in C\#, where the programmer is free to choose any ordering he likes when declaring methods, fields and properties on classes.

In C\# the problem of premature commitment may surface when dealing with class hierarchies. The problem is also present in F\#, as it is also object-oriented, but we argue that C\# is more prone to the class-related problems as it is object-oriented first. The problem of premature commitment arises when a programmer has to implement a base class, without being certain which other classes might inherit therefrom. This introduces guess-work and might result in missing functionality, unneeded class members and potentially the requirement to re-implement the base class. This problem is even more prominent when inheriting from third party code that contains multiple classes, as the programmer might choose to inherit from one class and later discover that he made a wrong decision and need to re-implement the entire class. The problem is more prominent in Unreal than Unity, as Unreal has multiple different base classes for components\cite{unreal:components}, where Unity has one.

Another small-scale issue of premature commitment arises when using F\#'s collection functions (such as \ttt{List.map} or \ttt{Array.reduce}). These functions take as first argument a function and as second the collection to operate on. If they are used without the pipe operator (\ttt{|\textgreater}), the \gls{IDE} will be unable to aid the programmer when he is implementing the mapping function until the second argument has been given. The correct order is thus to write the name of the function, add empty brackets, add the name of the collection and finally implement the function.

Another problem of premature commitment arises in F\#, because circular dependencies between classes are not allowed. The dependencies between classes are visible in \glspl{IDE}, where classes that are further down the source file list may depend on classes that are further up. If circular dependencies are needed in a program architecture, the programmer must define and implement an interface on one of the classes and reorder the source files. This constraint might seem annoying at first, but has the benefit of yielding class architectures with looser coupling\cite{interfaces:and:coupling}. 
\subsubsection{Progressive Evaluation}
Progressive evaluation defines how well a partially finished program can be executed and evaluated. Higher progressive evaluation means that more incomplete programs can be executed. C\# and F\# programs can only be executed if the source code is syntactically correct. We therefore deem that their progressive evaluation is equivalent. Both languages are supported by \gls{REPL}-tools called called C\# Interactive\cite{csharp:interactive} and F\# Interactive\cite{fsharp:interactive} respectively. These tools enable programmers to run and evaluate anything from single lines of code to whole files. We used this tool frequently in the beginning of the project, when we were learning F\# to experiment with different functions and constructs, before implementing them in the source code.
\subsubsection{Role-expressiveness}
Role-expressiveness defines how easily a program can be read and comprehended. The dimension is easily confused with hard mental operations or secondary notation, from which it should be kept apart\cite{green1996usability}. Role-expressiveness thus defines how self-explanatory a program is.

Being languages that run in the same platform, C\# and F\# share many constructs and all libraries. This means that a discussion of the standard library is of little interest. There are, however, some differences in the syntax that we will highlight. 

%let vs. var
First and foremost C\# uses either \ttt{var} or type names in variable declarations. In F\# the equivalent is \ttt{let}, possibly followed by \ttt{mutable} to indicate mutability. Depending on the programmer's background, these keywords may be more or less expressive. From a mathematical background it makes sense to create name bindings by using \ttt{let}, as that's common in proofs and similar mathematical lingo. The type of a name binding may be inferred by how it's used. Furthermore, \ttt{let} indicates a name binding and not a variable, which further underlines that F\# is pure until otherwise is expressed. This contrasts with the classic C-style way of defining variables; by using their type name. Some programmers that are less versed in mathematical notation may prefer this way, as it is more expressive of the variable's type. Finally, the \ttt{var} keyword is simply an abbreviation for \squote{variable}, which goes well hand-in-hand with its purpose; a variable which the compiler should determine the type of. Common for the last two types of declarations are that they do not indicate anything about mutability and thus require knowledge about the language at hand.

%F# fun-keyword
Lambda functions are available in both languages. In C\# they're expressed as \mintinline{csharp}|(a,b) => a + b|, where as in F\# they're expressed as \mintinline{fsharp}|fun a b -> a + b|. In this case F\# uses the keyword \ttt{fun} to indicate a lambda, which is quite literally an abbreviation of \squote{function}. One thing worth noting is that the \ttt{fun} keyword may easily be confused with the English word fun. Another abbreviation such as \ttt{fn} or \ttt{func}, would probably have been better. C\# uses the \ttt{=\textgreater}, which is of limited expressiveness, especially because C\# does not use arrow-style function signatures anywhere else.

%Type vs. class
In order to declare custom data structures in C\#, one can use \ttt{class}, \ttt{struct} or \ttt{enum}, depending on the purpose of the structure. In F\# all data structures are constructed using the \ttt{type} keyword. Depending on the symbols used in and around the definition, the outcome will change (see \lstref{fsharp:type}). Consequently this means that the \ttt{type} keyword in F\# has very limited role-expressiveness, compared to those of C\#.

\begin{listing}[H]
    \begin{minipage}{.45\textwidth}
        \begin{minted}{fsharp}
type Enum =
| B = 0
| C = 1

type Union =
| B
| C

type DiscriminatingUnion =
| B of bool
| C of char
        \end{minted}
    \end{minipage}
    \hfill
    \begin{minipage}{.45\textwidth}
        \begin{minted}{fsharp}
type Record = { b: bool, c: char }

type Class() =
    let b = true
    let c = 'x'

[<Struct>]
type Struct(b:bool, c:char) =
    member this.B = b
    member this.C = c
        \end{minted}
    \end{minipage}
\caption{Different kinds of data structures defined using the \ttt{type}-keyword in F\#.}
\label{lst:fsharp:type}
\end{listing}
\subsubsection{Secondary Notation and Escape from Formalism}
Secondary notation and escapte from formalism defines how well a programming environment supports conyeing information that is not part of the source code. Typical examples of such are comments, indentation and grouping code into paragraphs in textual languages\cite{green1996usability}.

C\# and F\# are very alike in this dimension. They both support the \ttt{//}-operator, which indicates that the rest of the line should be commented out and matching pairs of \ttt{/*} and \ttt{*/}, which comments everything out between them. Furthermore functions, methods, classes and more or less any program construct may be annotated with \ttt{///}-comments, which allow the programmer to add \gls{XML}-documentation\cite{fsharp:xml:doc}. The programmer may use this to describe the intend of the construct along with its arguments and, if needed, link to other constructs in the program. Other developers may open this documentation in a pop-up box elsewhere in the program, whenever such construct is encountered.
\subsubsection{Viscosity}
Viscosity defines how much effort a developer has to put in to make a small change. \cite{green1996usability} notes that textual languages are less viscous than visual programming languages.

C\# and F\# are both textual languages and thus have relatively low viscosity. We argue that the primary difference between C\# and F\# is scope delimitation. C\# scopes are delimited by pairs of curly brackets, whereas in F\# they are delimited by indentation. If a programmer is to move code from one scope to another in C\#, he would have to either insert or delete pairs of curly brackets, whereas in F\# he would select the code that needs to be moved and press TAB or Shift+TAB.
\subsubsection{Visibility and Juxtaposability}
Visibility and Juxtaposability determines whether required material is accessible without cognitive work\cite{green1996usability}. In textual languages this dimension is not necessarily determined by the language, but more so by the environment (\gls{IDE}).

There are two prominent \glspl{IDE} for the .NET platform: Visual Studio and Rider. Both \glspl{IDE} support numerous ways of making source code available. Examples of such are:
\begin{itemize}
    \item Splitting the text editor both horizontally and vertically, such that two files can be open side-by-side
    \item Jumping to implementations by control-clicking.
    \item Hovering over function or type names to read a description of what they're meant for.
    \item Opening documentation pop-ups that explain how to use of classes and methods.
\end{itemize}

\section{Usability Evaluation}
In this section we present the usability evaluation that was conducted as part of the project. We first go over the setup of the test, presenting participant selection criteria and describe the tasks that were given to the participants. We then turn to the results of the test, where we adopt the analysis method described in \champagne\cite{blackwell2004champagne}. The \champagne method gives a rather shallow overview of how the language scores and hence we present a more in-depth analysis of the problems using the \cognitive. Finally we discuss potential sources of errors of the experiment.

\subsection{Setup}
The test setup draws inspiration from the Champagne Prototyping method and Discount Method for Language Evaluation. The tasks, prototype and participants were selected according to the former, whereas the use of a cheat-sheet for the participant was inspired by the latter. Contrary to the suggestion of using a text editor in Discount Method for Language Evaluation, we chose to use \glspl{IDE}, as we're testing well established languages. We allowed the users to choose between JetBrain's Rider and Microsoft's Visual Studio, depending on what they were used to.

For each participant a one and a half hour session was planned. We expected that actual coding time would be roughly one hour, as we conducted a questionnaire before the test to learn about the test participants' experience with Unity and an interview after the test to allow the participant to share their opinion on F\# in Unity, C\# and functional programming. Out of the one hour coding time, we intended to use 20 minutes on C\# and 40 minutes on F\#, as the participants were required to have experience with C\# in Unity and therefore likely would complete the C\# tasks faster.

The cheat-sheet served a two-fold purpose and was made available online\footnote{\url{https://sppt-2019.github.io/unity-fsharp-introduction/}. Please note that the document is in Danish, as all participants were Danes.} using Github pages. The first purpose of the document was to give the test participants an introduction to F\# in Unity prior to the test and second to act as a cheat-sheet during the test. Similarly, the tasks were also made available online during the test\footnote{\url{https://sppt-2019.github.io/unity-fsharp-introduction/tasks/}. Please note that the document is in Danish, as all participants were Danes.}. We also created a Github repository, in which the test-setups were stored\footnote{\url{https://github.com/sppt-2019/Unity-FSharp}}. The master branch of said repository holds a Unity project with eight scenes, one for each of the test cases. The purpose of this setup was to remove Unity as a factor in the experiment and avoid having the participants spend time setting up scenes. For each of the test participants we created a new branch in the repository, which would allow us to view each of the participants' code in isolation.

During the tests we recorded the screen and audio on the test computer. The files were not transcribed, but whenever we quote one of the participants we refer to transcriptions of larger pieces of dialogue that are listed in \appendixref{user:quotes}. We do so in order to give the reader more context on the quotes and avoid \dquote{plucking} sentences out of their context.

\subsubsection{Test Cases} \label{sec:usability:test:cases}
We created a total of eight test cases, which fall into four categories; player controller, generalised walkers, map-reduce and \dquote{concurrent} update. We list concurrent with double ticks here, as the tasks were designed to be implemented with \ttt{async}/\ttt{await} in C\# and asynchronous workflows in F\#. The reason we chose this was that it would be possible for the participants to implement a concurrent version with minor rework, given the right setup. Unity's concurrency model requires more boilerplate code, which we deemed would not provide any useful information.

\begin{labeling}{\quad\quad}
    \item[FPS Controller] The participant is to implement a \gls{FiPS} controller, i.e. a component which can be added to a player character to move it around the world with the WASD-keys and rotate the camera with the mouse.
    \item[3rd Person Controller] This test case is similar to the previous test case, only that the camera must rotate around the player character.
    \item[Talent Tree-Walker] The participant is first asked to implement a data structure that can hold a talent tree. Afterwards the player is given a pre-made talent tree from which the participant must calculate two things. At first a character's bonuses in three attributes should be calculated by summing all talents that are \ttt{Picked} and afterwards the maximum achievable bonus of all attributes by summing all talents.
    \item[Armour Graph] In this test case, the participant is given a list of armour equipped on a character. The participant is to implement code that sums the character's bonus in three attributes from the armour. In the second half of the test, we assume that certain pieces of armour can scale the attributes from all other pieces of armour and ask the participant to calculate the scaled attribute bonuses.
    \item[Dialogue Tree] The participant is to first implement a data structure that can hold a dialogue tree. Afterwards the participant must parse a list of dialogue options into a tree using his own data structure. Finally the participant is to find all unique dialogue paths that has a certain outcome.
    \item[Currency] The participant is presented with three different types of coin. He must first implement code that exchanges a given number of said coins into the minimum number of total coins. Afterwards he should add a function to calculate whether a player can buy a certain item from a vendor and finally implement code that buys the item and updates the player's wallet.
    \item[Unit Management (RTS)] The participant is to implement an inverse state machine. By inverse we mean that the state machine should hold collections of entities for each state in the state machine. At each \ttt{Update} the state machine should map the corresponding state's update-function over each collection to create new collections of entities. Finally the participant is asked to implement a \dquote{concurrent} mapping of the update-function.
    \item[Magnetic objects] The participant is to simulate magnetism. He is presented with a list of objects, some of them which are magnetic. All magnetic objects should be attracted to a common center-point at a given speed. In the second half of this task the participant is asked to implement a \dquote{concurrent} version of the simulation.
\end{labeling}

%In \tableref{task:categories} we list the categories and their associated tasks. We had initially intended that each test case should be implemented by two different participants in C\# and F\#, but after the first session we could conclude that each participant would only be capable of solving one task. We therefore to a fixed ordering.

 \makeTable{
     {| l | c | c |}
     \hline
     \textbf{Category} & \textbf{F\# Task} & \textbf{C\# Task} \rowEnd
     Player Controller & FPS Controller & 3rd Person Controller \rowEnd
     Generalised Walkers & Talent Tree & Armour Graph \rowEnd
     Map-Reduce & Dialogue & Currency \rowEnd
     \dquote{Concurrent} Update & Magnetism & Unit Management \rowEnd
 }{Categories and their associated tasks}{task:categories}

 In \tableref{task:categories} we list the categories and their associated tasks. As we wanted to explore how suitable Carmack's and Sweeney's approaches to game development are, we decided to use a fixed ordering of the test categories for each participant. We had initially planned that the participants would be given one of the tasks from the controller category as the first task and distribute the remaining semi-randomly, but discovered after the first actual test that the participants would only have time to solve a single task. We therefore decided that the participants would be given starting tasks from different categories. This ordering allow us to compare each of the participants' the solutions to tasks of the same category.

\subsubsection{FRP System}\label{sec:frp-sys}
We chose to add support for F\# via the Unity F\# Integration plugin\cite{fsharp2019plugin}. The test cases are to be implemented in a \gls{FRP} system that we developed in F\#. The \gsl{FRP} system introduces a new class called \m{FRPBehaviour}. Classes that inherit from \m{FRPBehaviour} inherit a method called \m{ReactTo}, which exists in two variations. The first variations accepts an event type (such as Update, Keyboard or MouseMove) and a handler. This method will unconditionally invoke the handler whenever an event of the given type occurs. The second variation accepts an event type, a filtering function and a handler. This variation will only invoke the handler when the filtering function returns true. \lstref{frp:example} shows an example of the magnetism task implemented in F\#. This code rotates all balls to look at the venter and thereby moves them forward.

\begin{listing}[H]
    \begin{minted}{fsharp}
member this.Start() = 
    let balls = GameObject.FindGameObjectsWithTag("Magnetic")

    this.ReactTo (FRPEvent.Update,
        (fun () -> 
            let center:Vector3 = getCenter(balls)
            let updatedBalls = 
                balls
                |> Array.map (fun b -> lookAt(b, center))
                |> Array.map step
        )
    )
    \end{minted}
    \caption{Implementation of the magnetism task in F\#. The \m{getCenter} and \m{lookAt} functions are excluded for brevity.}
    \label{lst:frp:example}
\end{listing}

The F\# plugin also allows programmers to implement gameplay code using the standard Unity strategy (i.e. a chain of if-else statements to check for input in the \m{Update}-method). We decided that we wanted to remove this option entirely and therefore implemented an \m{Update} method in the \m{FRPBehaviour} that was reserved for condition checking. This method cannot be overridden, which entirely prevents the user from using \m{Update}-based programming. We took this decision partly because scientific research underlines that \gls{FRP} is well suited for game development\cite{courtney2003yampa,cheong2005functional,maraffi:frp} and partly because we argue that the temptation of writing C\# code in F\# syntax would be too large.

\subsubsection{Participants} \label{sec:par-crit}
The participants for this experiment were recruited by sending emails to game studios and other game-related companies in Denmark, asking if their employees were interested in participating in the experiment. The participants were required to have experience with C\# and Unity in the game development industry. We gathered a total of six participants; two participants with Indie game-development experience, three from a company that creates \gls{AR}/\gls{VR} applications and one who teaches children game development.

For selection criteria we looked for participants that had developed games or had significant experience with C\# in Unity. In addition the participants need not have experience with F\#. The criteria were ranked as follows:
\begin{enumerate}
	\item Game development experience.
	\item Unity \& C\# experience.
	\item No F\# experience.
\end{enumerate}

\subsubsection{Pilot Test}
We conducted a pilot test with a participant from the Programming Technology specialisation course at Aalborg University in order to get some feedback on the setup. The test participant has experience with Unity from Indie game development and teaching Unity programming to children in the Danish secondary school/advanced level (not to be confused with our test participant, who has a similar job). Contrary to the requirements for the actual test, the pilot test participant had experience with F\# and functional programming in general.

Prior to the pilot test we had assumed that a 50/50 distribution between C\# and F\# would give the most valuable results, as it would not skew the results time-wise. During the pilot test the participant, who had prior experience with F\#, was able to complete three tasks in C\# and two in F\#. We therefore decided to change the distribution to 20 minutes of C\# and 40 minutes of F\#.

Apart from that, the participant noted that some of the tasks were hard to understand, mainly because they required the participant to utilise code that was pre-implemented. We decided to rewrite the tasks and insert code snippets with existing classes where-ever the participant would have to use them. Finally, the participant noted that the function-name \ttt{ReactTo}, which is used to bind an event handler to a given event, was odd. He suggested renaming it to \ttt{Bind}. We chose to ignore this suggestion, with the argument that the sound of \ttt{ReactTo} has a strong connection with \gls{FRP}, whereas \ttt{Bind} has a stronger connection with conventional functional programming.

\section{Results Analysis} \label{sec:test-results}
The result of the tests was six git branches with source code written by the participants along with five video-files. Sadly the video-recording program broke down during one of the tests and we were unable to recover the file. We took notes during the tests and will refer to those instead. Whenever we quote the notes rather than the participant, we will clearly indicate that.

In general the participants were able to complete roughly one test case, some didn't and some started the second as well. This was the case for both \fsh and \csh.

\subsection{Questionnaire}
All participants were asked to estimate their own skill levels in these categories and give a ballpark estimate of how many Unity applications they had developed. The results can be seen in \tableref{participant-scores}.

\begin{table}[H]
\begin{tabular}{| c | r | r | r | r | r |}
	\hline
	\textbf{Participant}&\textbf{\csh}&\textbf{Unity}&\textbf{Game Dev}&\textbf{Functional}&\textbf{Unity Apps} \\ \hline
	1 & 9 & 9 & 5 & 3 & 25 \\ \hline
	2 & 8 & 8 & 7 & 2 & 10 \\ \hline
	3 & 8 & 8 & 2 & 1 & 12 \\ \hline
	4 & 10 & 10 & 8 & 1 & 10 \\ \hline
	5 & 9 & 9 & 9 & 2 & 10 \\ \hline
	6 & 8 & 8 & 9 & 6 & 5 \\ \hline
\end{tabular}
\caption{Participants Self Evaluations}
\label{tab:participant-scores}
\end{table}

\newcommand{\mn}{\newmoon}
\newcommand{\mns}{\fullmoon}

\subsection{Data Processing}
In the following section we will process the data from the tests, using methodology presented in \secref{prog-usability}. The tests here constructed using the Champagne Prototyping methodology and the analysis of the data will also follow this approach. Champagne Prototyping uses two methodologies for data processing; Attention Investment Model and Cognitive Dimensions.

\subsection{Syntax Comparison}
Test participants encountered problems with various aspects of F\#'s syntax. Such issues are to be expected when developing in a new programming language, furthermore, syntactical errors are not the main focus of this inquiry. Therefore these issues are discussed in this section.

\subsubsection{Bindings and Operators}
In F\# the \ttt{=} symbol is used to denote two different operators; the value-binding operator, and the equality operator. Notably the symbol is not used for rebinding, F\# equivalent of assignments which it is in C\#. A comparison of the two different rebinding/assignment styles can be seen in \lstref{ass-comp}. Rebindings are only allowed on \ttt{mutable} variables, which should be limited to a scope.

\begin{listing}[H]
\begin{minipage}{.45\textwidth}
\begin{minted}{fsharp}
let mutable x = 0
x <- 1
\end{minted}
\end{minipage}
\hfill
\begin{minipage}{.45\textwidth}
\begin{minted}{csharp}
int x = 0;
x = 1;
\end{minted}
\end{minipage}
\caption{Assignment Comparison in F\# (left) and C\# (right).}
\label{lst:ass-comp}
\end{listing}

Furthermore, since \ttt{x = 1} is valid F\#, no errors where encountered immediately. Instead the program behaved unexpectedly and they received a warning, that the value of a boolean expression was being ignored. None of the participants where able to deduce the source of the error without assistance from the monitor. The degree of assistance required ranged from a hint to the monitor explaining the problem so that the test could continue.

Some participants encountered minor errors when declaring variables, because variables are implicitly immutable in F\#. This is the opposite of C\#, where the \ttt{const} keyword is used to explicitly declare a variable immutable. However, once the participants realised this, they had no issue using it.

\subsubsection{Scopes and Indentation}
Several of the participants did not connect indentation with scope. This meant that the scoping of variables at times presented a challenge. In most test cases the solutions are implemented in a class, therefore when test participants indented their function bodies incorrectly the code would work initially. This was because the function body would be part of the class instead of the function and could therefore still be called within the scope, this can be seen in \lstref{scope-prob}.

\begin{listing}[H]
\begin{minted}{fsharp}
type FRP_FPSController() =
    inherit FRPBehaviour()

    [...]

    member this.HandleMoveForward() =
    let newPosition = this.transform.position + new Vector3(0.0f, 0.0f, _velocity)
    this.transform.position <- newPosition
\end{minted}
\caption{Incorrect Indentation}
\label{lst:scope-prob}
\end{listing}

Lines 7 and 8 are indented incorrectly, but the code compiles and behaves as expected. The \gls{IDE} gives a warning on both lines, but no errors are thrown until new \ttt{member} functions are defined.

\subsubsection{Types and Type Inference}
Many participants had problems with the type system and type inference. The functions provided in the sample sheets had explicitly typed parameters, which the participants partially mimicked when declaring their own functions, however participants did not specify function return types nor did they utilise typing of their variables. An example of such a function can be seen in \lstref{part-func}.

\begin{listing}[H]
\begin{minted}{fsharp}
[...]
let getTotalWithMod (items:Item list) (attribute:Item->int) (attributeMod:Item->float32) =
[...]
\end{minted}
\caption{Participant Function with Type Annotations}
\label{lst:part-func}
\end{listing}

In addition several participants remarked that they preferred strictly typed languages, even though they had encountered several typing errors. Some of the participants had Python experience which may explain this assumption, because Python 3.x annotates function parameters similarly to F\# and Python is dynamically typed.

\subsubsection{Function Definitions}
In F\# all \ttt{let} functions must be declared before the first \ttt{member} function. An example of this can be seen in \lstref{let-mem-incor}. This caused some initial confusion for participants, but it was quickly overcome. The reason for the strict order was not intuitive for the participants. The difference between the use of \ttt{let} and \ttt{member} is the accessibility of the method or function in question. The \ttt{let} keyword denotes a function in the class instance's scope, effectively a private function. The \ttt{member} keyword denotes a method which is similar to methods in C\#.

\begin{listing}[H]
\begin{minted}{fsharp}
type Candy(price:int) =
  member val Price = price with get
  let discountTuesday = this.Price / 2
\end{minted}
\caption{Incorrect Order}
\label{lst:let-mem-incor}
\end{listing}

In addition to the confusion surrounding type definitions, participants also expressed frustration with how functions are defined. The issue stems from a problem with our sample sheet which did not contain an example of a simple function definition, plenty of functions were defined, but only ever as part of examples of other language features. Furthermore, functions share the \ttt{let} keyword with namebindings, which contributed to the confusion during the test.
%Some participants also had some problems with lambda expressions. Some users where initially thrown off by the \ttt{fun} keyword, while others where new to the concept.

Several participants struggled with lambda expressions in \fs. Some of these participants expressed that they where not familiar with lambda expressions in \cs either, but even the participants with lambda expression experience from \cs had issues with them in \fs. The difference in syntax can be see in \lstref{lam-exp-syn}, with \cs on the left and \fs on the right.

\begin{listing}[H]
\begin{minipage}{.5\textwidth}
\begin{minted}{csharp}
str => Console.WriteLine(str);
\end{minted}
\end{minipage}
\hfill
\begin{minipage}{.4\textwidth}
\begin{minted}{fsharp}
fun str -> printfn str
\end{minted}
\end{minipage}
\caption{Lambda Expression Syntax, C\# on the left and F\# on the right.}
\label{lst:lam-exp-syn}
\end{listing}

\subsection{Champagne Prototyping}
In this section we analyse the video files from the usability experiment using the champagne prototyping methodology. This method uses two different analysis strategies; attention investment and cognitive dimensions. In both of these analyses we count mentions of each feature or dimension. We present these using bullets in tables, which is recommended by \cite{blackwell2004champagne}. A closed bullet (\mn) means that a participant mentioned an aspect of the feature, when the participant mentioned it multiple times an open bullet is used (\mns). The categories, \textit{Correct \& Unprompted} and \textit{Correct \& Prompted} are instances where the participants mention or explain a feature correctly and/or positively. The last category are instances where features are mentioned negatively, incorrectly or is a cause for confusion.

\subsection{Attention Investment Model}\label{sec:att-inv-app}
The attention investment model, presented in \secref{attention-investment}, can be used to map out the programming steps undertaken by a user. This section will endeavor to do so, using the user test video material. Firstly, user comprehension of the feature being evaluated is measured. This feature is the use of \fsh and \gls{FRP} in gameplay programming. In order to measure comprehension, the instances where participants mention aspects of the feature were recorded and categorised, as can be seen in \tableref{comp-matrix}.

\makeTable{
	{| p {.20\textwidth} | p {.20\textwidth} | p {.20\textwidth} | p {.20\textwidth} |}
	\hline
	\textbf{Recognised Aspects} & \textbf{Correct \& Unprompted} & \textbf{Correct \& Prompted} & \textbf{Incorrect} \rowEnd
	Modularity 			& \mn\mn & & \mn \rowEnd
	ReactTo 				& \mns & \mn\mn\mns & \mn\mn\mn \rowEnd
	Types 					& \mn\mn & & \mn\mns\mns\mns\mns\mns \rowEnd
	List Operations	& \mn & \mn\mn\mns & \mn\mns \rowEnd
}{User comprehension of the \gls{FRP} features.}{comp-matrix}

In addition to a measure of the participants' comprehension, the attention investment model also provides a quantification of their efforts in the programming activity. This consists of four metrics, mentioned in \secref{attention-investment}. This can be seen in \tableref{att-inv-findings}. The risk metric measures the amount of times participants mentioned or discussed things that could go wrong, which includes increased difficulty of some tasks using \fsh. The cost metric is the attention and time required to switch to \fsh. Any musing over the difficulties of switching over is included. The payoff is the reduced cost of gameplay programming after switching to \fsh. The imperative alternative metric is the number of times participants mentioned the problems in \csh, or other imperative languages, that form the basis for switching to \fsh.

\makeTable{
	{| l | c |}
	\hline
	\multicolumn{2}{|c|}{\textbf{Attention Investment}} \rowEnd
	Investment Risk & \mn\mn\mn\mn\mns  \rowEnd
	Investment Cost & \mn\mns\mns\mns\mns \rowEnd
	Investment Payoff & \mn\mns\mns\mns\mns \rowEnd
	Imperative Alternative & \mn\mn\mn \rowEnd
}{Attention investment findings.}{att-inv-findings}

The participants were able to correctly use and describe \fsh and \gls{FRP} behaviour in some instances and struggled in other instances. As can be seen in \tableref{comp-matrix}, not all participants were cognisant of all feature aspects, e.g. all participants misunderstood or struggled with the type system. Functional programming claims a greater degree of modularity than imperative languages\cite{hughes1989functional}, however, less than half of the participants expressed cognisance of this. Some participants wrote much more modular code in \fsh than in \csh, but did not mention it.

As can be seen in \tableref{att-inv-findings} the participants noted a high cost with a high payoff. Participants could see the usefulness of \gls{FRP}, but several participants expressed uncertainty of any benefit provided by \fsh. Another point is that very few participants noted the problems with existing solutions, even when compared to \fsh. The high cost is attributed to loss of productivity while a developer learns to use \fsh.


\subsection{Cognitive Dimensions}
In this section we use the cognitive dimensions framework to aid the analysis of the video from the usability test. In this analysis we count the number of times the participants experience or mention problems with each dimension. We use the same notation that was presented in the previous section. The frequency of mentions can be seen in \tableref{cog-dim-findings}. The mentions counted are any instances where the participants said or did some thing that fit under a cognitive dimension. Many of the counted instances are therefore not explicit statements made by the participants, but rather instances where their actions fell under one of the categories. Furthermore, the mentions are not necessarily negative, but might also reflect instances where the user made use of the given dimension.

\begin{table}[H]
	\alignCenter{
	\begin{tabular}{| l | l | l |}\hline
		& \textbf{F\#} & \textbf{C\#} \rowEnd
	 	Abstract Gradient & \mn & \mn \rowEnd
		Closeness of Mapping & \mn\mn &  \rowEnd
		Consistency & \mn\mn\mns\mns\mns & \mn\mn \rowEnd
		Diffuseness/Terseness & \mn\mn\mns & \mn \rowEnd
		Error-proneness & \mn\mns\mns\mns\mns & \mn\mn \rowEnd
		Hard Mental Operations & \mn\mn\mns & \mns \rowEnd
		Hidden Dependencies & \mn\mn & \mn \rowEnd
		Premature Commitment & \mn\mn\mn & \rowEnd
		Progressive Evaluation & \mn\mn\mns & \mn\mn \rowEnd
		Role-expressiveness & \mn\mn\mns\mn & \mn\mn \rowEnd
		Secondary Notation and Escape from Formalism & & \mn\mn \rowEnd
		Viscosity & \mn & \mn \rowEnd
		Visibility and Juxtaposability & \mn\mns & \mn \rowEnd
	\end{tabular}}
	\caption{Cognitive Dimensions Findings}
	\label{tab:cog-dim-findings}
\end{table}

It is not surprising that participants encountered more problems with \fs than they did with \cs. Participants were selected based on the criteria stated in \secref{par-crit}, these included requirements for \cs experience and limited to no  \fs experience. This allowed us to study the potential adoption difficulties of \fs. The higher frequency of mentions when using \fs is inline with this.

While an overview can be gained from \tableref{cog-dim-findings}, it does not provide adequate information of \textit{why} a participant would undertake a certain strategy or action. Therefore, selected instances among the mentions, are analysed further in the following sections.

\subsection{Instance Analysis}
Some instances during the test made the difficulties of using \fs and the functional paradigm more clear than others. In this section, these instances are explored and examined in order to discover the problems facing developer adopting \fs in a game development setting. Each instance is discussed and explored under the dimension it belongs to.

\subsubsection{Consistency}\label{sec:part-cons} % p6 20:15
All participants had prior \cs experience which affected their expectations. This was apparent when participants applied \cs methodology in the \fs code. An example of this is the confusion of types some participants experienced. Several participants noted that they preferred strict typing or specifying types manually. An example of this can be seen in \lstref{type-conf}.

\begin{listing}[H]
\begin{minted}{fsharp}
[<SerializeField>]
let mutable _velocity = 5.0f
[...]
member this.HandleMoveForward() =
  this.transform.position+=new Vector3(0,0,this._velocity)
\end{minted}
\caption{Problem experienced with types in F\#. The \m{Vector3} constructor accepts \m{float}s and are invoked with \m{int}-parameters.}
\label{lst:type-conf}
\end{listing}

In \lstref{type-conf} the participant correctly types the \ttt{\_velocity} variable on line 2. However, when attempting to set the object's position on line 5, the participant first uses \ttt{0} instead of \ttt{0.0f}. This valid in \cs, but not in \fs. This is an instance of confusion surrounding the automatic type inference of variables and the typing of literals.

\begin{listing}[H]
\begin{minted}{fsharp}
let moveMagneticBalls (objs:GameObject[]) (center:GameObject) =
  objs center |> Array.map (fun i ->
    i.transform.LookAt(center.transform)
    i.transform.Translate(i.transform.forward * Time.deltaTime * speed))
\end{minted}
\caption{Closure misunderstanding. The user attempts to catch \m{center} in the closure by piping it into the map-function.}
\label{lst:clos-mis}
\end{listing}

Some participants also experienced issues with closures. In \lstref{clos-mis} a participant has defined a function to move a number of objects towards a center point. The center point, \ttt{center}, is passed as a parameter, but the participant became confused as to how to pass it to the lambda expression. Therefore center was added on line 2 after \ttt{objs} and piped into the \ttt{map} function. This causes an error because \ttt{objs center} is now a function invocation to a non-existent function, \ttt{objs}, with the argument \ttt{center}. Passing \ttt{center} to the lambda function is unnecessary, because it is captured in the lambda's closure.

The behaviour described is consistent within \fs, but not with the expectations of the participant. Arguably, this instance is a product of the participant's inexperience with functional programming, however it is an example of the disharmony between the largely consistent rules of \fs and the expectations of programmers. These consistency issues were primarily present in the \fs code, which is not surprising considering the participants' experience.

\subsubsection{Role-expressiveness}
In F\# the last expression in a function body is implicitly its return. In some cases that should not be the case and the function should return \ttt{Unit} (equivalent to C\#'s \ttt{void}). This can be done by adding \ttt{()} as the last line of the function body. This confused some of the participants, as they felt that it was unnecessary to explicitly indicate a non-existing return type. One of the participants even asked directly what \ttt{Unit} was and the monitor answered with an explanation. Approximately fifteen minutes later the participant encountered another \ttt{Unit}-type problem and was unable to recover without an additional explanation.

Another role-expressiveness problem we encountered was that some participants attempted to \ttt{let} declare multiple values without initialising them. Some participants assumed that \ttt{let} declarations without assignment would result in default values (such as \ttt{null} for classes). Another problem faced by the participants were related to class type declarations, where the participants did not realise that the brackets after the type's name could be used to pass constructor arguments.

A single participant also noted that he found F\# pipe operations \textit{\dquote{not nice to read}}. He stated that he would rather prefer the SQL-like variation of \gls{LINQ} in C\# because it is closer to plain English. The monitor asked if it was related to the names of the functions (\ttt{map} and \ttt{reduce} or \ttt{Select} and \ttt{Aggregate}), to which the participant stated that it was solely related to the way pipe operations were structured.
% Participant 6: Hvad betyder () - 4:30 + Unit!? - 20:30
% Default værdier på variable using let
% Participant 6: Type constructors
% Select vs. map: Participant 3, 06.30

\subsubsection{Secondary Notation}
In the test we saw very limited use of secondary notation. This is likely caused by the relatively small tasks, along with the fact that the code was not meant to be used in production nor expanded upon. The programmers, however, would often open documentation pop-up boxes to read about functions and methods before putting them to use. Generally, our intuition was that the programmers found it more easy to understand the C\# documentation than the F\# documentation. We suspect that there two reasons:
\begin{enumerate}
    \item The programmers were more experienced in C\#.
    \item F\# uses the arrow function signatures (e.g. \fsinline{val map : mapping : ('T -> 'U) -> list:'T list -> 'U list}, which is the signature of \ttt{List.map}) in its documentation. This notation indicates that a function of multiple parameters can be curried. Currying is not supported in C\# and was thus a concept that the programmers were unfamiliar with.
\end{enumerate}
\subsubsection{Viscosity}
% Participant 2, 40:40 samt 1:02:30
In our test cases, viscosity is particularly visible in the \dquote{concurrent}-update category. The reason for this is that the participants are asked to develop a sequential solution first, followed by a parallel implementation. Generally viscosity is low in both languages. In F\# we saw the magnetism task implemented using the pipe operator. Such an implementation can be extended to a parallel solution by piping into \ttt{Async.Parallel} and then \ttt{Async.RunSynchronously} (see \lstref{fsharp:pipe:async}). A similar solution can be achieved in C\# using \gls{LINQ}, albeit the change requires the programmer to delete a semicolon.

\begin{listing}[H]
    \begin{minted}{fsharp}
let speed = 3f;

let moveBallForward (ball:GameObject) =
    ball.transform.Translate(ball.transform.forward * Time.deltaTime * speed)

let Update () =
    let balls = GameObject.FindGameObjectsWithTag("Magnetic")
    balls
    |> Array.map moveBallForward
    ()

let UpdateAsync () =
    let balls = GameObject.FindGameObjectsWithTag("Magnetic")
    balls
    |> Array.map (fun b -> async {moveBallForward})
    |> Async.Parallel
    |> Async.RunSynchronously
    ()
    \end{minted}
    \caption{Transforming from sequential to concurrent list operations in F\#.}
    \label{lst:fsharp:pipe:async}
\end{listing}

As with many other dimensions, viscosity can also be affected by the programmer's style of programming. This is explified in \lstref{csharp:viscous}, which is taken from one of the solutions in C\#. In order to make implement \dquote{concurrent} update, the participant had to construct a new list in the \ttt{Update}-method and wrap the calls to the state methods in \ttt{Task.Run} (e.g. \mintinline{csharp}|updateTasks.Add(Task.Run(() => Flee(fleeingShooter)))|). This change is manageable, but imagine how much effort it would take if we wanted to add an additional state to the shooter units. If such change was to be implemented, the programmer would have to:
\begin{itemize}[H]
    \item Add an additional list and \ttt{foreach}-statement in \lstref{csharp:viscous}.
    \item Add an additional case to the switch-statement in \ttt{JoinState} in \lstref{csharp:viscous}.
    \item Change the signature of the \ttt{TransferState}-method in \lstref{csharp:viscous:transfer} to accept an additional list and add an addtitional \ttt{else if} for the third list.
    \item Add an additional case to the switch in \lstref{csharp:viscous:transfer}, and update the call to \ttt{RemoveFromList} in all other cases.
\end{itemize}
An alternative and less viscous solution was implemented by another participant in the test. We have listed that in \apxref{csharp:non:viscous}.

\begin{listing}[H]
    \begin{minted}{csharp}
class StateMachine : MonoBehaviour
{
    [...] //Pre-implemented code

    private List<Shooter> fleeingShooters;
    private List<Shooter> movingShooters;
    private List<Shooter> attackingShooters;

    public void JoinState(Shooter shooter, State state)
    {
        switch(state)
        {
            case State.Fleeing:
                fleeingShooters.Add(shooter);
                break;
            case State.Moving:
                movingShooters.Add(shooter);
                break;
            case State.Attacking:
                attackingShooters.Add(shooter);
                break;
            default:
                break;
        }
    }

    private void Update()
    {
        foreach(var fleeingShooter in fleeingShooters)
        {
            Flee(fleeingShooter);
        }
        foreach (var movingShooter in movingShooters)
        {
            Move(movingShooter);
        }
        foreach (var attackingShooter in attackingShooters)
        {
            Attack(attackingShooter);
        }
    }

    [...] //TransferState
    [...] //RemoveFromList

    [...] //methods for each unit state
}
    \end{minted}
    \caption{Example of viscous C\# implementation of the Unit Management Test.}
    \label{lst:csharp:viscous}
\end{listing}

\begin{listing}
    \begin{minted}{csharp}
public void TransferState(Shooter shooter, State state)
{
    switch (state)
    {
        case State.Fleeing:
            fleeingShooters.Add(shooter);
            RemoveFromList(shooter, ref movingShooters, ref attackingShooters);
            break;
        case State.Moving:
            movingShooters.Add(shooter);
            RemoveFromList(shooter, ref fleeingShooters, ref attackingShooters);
            break;
        case State.Attacking:
            attackingShooters.Add(shooter);
            RemoveFromList(shooter, ref movingShooters, ref fleeingShooters);
            break;
        default:
            break;
    }
}
    \end{minted}
    \caption{TransferState-method, which is part of the viscous Unit Management implementation from \lstref{csharp:viscous}.}
    \label{lst:csharp:viscous:transfer}
\end{listing}

\begin{listing}
    \begin{minted}{csharp}
private void RemoveFromList(Shooter shooter, ref List<Shooter> list1, ref List<Shooter> list2)
{
    if (list1.IndexOf(shooter) != -1) {
        list1.Remove(shooter);
        return;
    }
    if(list2.IndexOf(shooter) != -1) {
        list2.Remove(shooter);
        return;
    }
}
    \end{minted}
    \caption{RemoveFromList-method, which is part of the viscous Unit Management implementation from \lstref{csharp:viscous}.}
    \label{lst:csharp:viscous:remove}
\end{listing}



%Asger: "Nu copy-paster jeg kode. Det er ikke godt, men det er spiludvikling." - 56:15
%Asger: "Det er det jeg godt kan lide ved funktionelt: De tvinger dig til at gøre det pænt, men her har jeg lov til at svine." -59:10
%Participant 3: "Altså hvis det er noget jeg skal bruge tre gange, så er det måske... altså så er det fint nok bare lige at kopiere det, men eller kunne man godt lige lave en funktion." - 13:45
\discount\subsection{Threats to Validity} \label{sec:validity}
In this section we will examine potential sources or errors and other threats to validity. Such threats are categorised as either an internal threat or an external threat. An internal threat occurs when data is mishandled, misinterpreted or in some other way skewed to such a degree that the results are untrustworthy. The second category consists of errors caused by the data being inapplicable to other cases. This inline with terminology outlined in \cite{mcleod:validity}.


\subsubsection{Internal Validity}
In this section the internal validity is explored. The goal of this section is map out the possible  shortcomings originating from data handling and interpretation. Each potential threat is explained in turn and actions undertaken to mediate the threat are outlined. In some cases a threat cannot be sufficiently mediated, in which case it is simply listed.

\paragraph{Evaluation Parameters}
The user tests where evaluated in accordance with the \champagne method (see \secref{champagne}). This methodology is intended to be used to measure the usability of a single feature, not an entire programming language. Therefore it was modified to fit our case better. The methodology consists of two other usability techniques: \cognitive and \attentions (see \secref{cog-dim} and \secref{attention-investment}). We modified the \cognitive aspect to focus on the languages as a whole and kept the feature focus of \attention. Thus the evaluation parameters where centered on the users experience with \fs and their understanding of the \gls{FRP} system.

The \cognitive framework is originally intended to  estimate the usability, or provide vocabulary to such an estimation, of a notational system such as a programming language. Therefore our modification of \champagne is to use \cognitive for its original purpose. Furthermore, a single aspect of \discount was used, namely the sample sheet (see \secref{discount-method}). This was employed to assist test participants with \fs.

\paragraph{Task Difficulty}
The user test consisted of a number tasks that participants where asked to complete. These tasks where inspired by game development scenarios and applicability of functional idioms. Before the test we where worried that some tests where more difficult than others. This could skew the results somewhat since, participants completing the more difficult tasks would struggle more then participants completing the easier tasks. The dialogue tree task (see \secref{usability:test:cases}) presented a much more difficult problem than expected. The task relied on recursion and tree structures which where unfamiliar to participants.

After the user test is became apparent that some tasks where indeed more difficult than others, significantly so in some cases. The difficulty of the tasks themselves did not affect the results to the degree we had expected, instead the difficult tasks brought conflicting idioms and faulty problem solving approaches to light that may not have been as apparent in the easier tasks.

\paragraph{Task Presentation}
During the tests it became apparent that some tasks where not formulated clearly enough. This meant that participants misunderstood the task and therefore did not sufficiently complete the task. These misunderstandings came from unclear task descriptions and lacking context. To mediate this we conducted a pilot test with another software student. Using this test, clarified several tasks before the user test. Unfortunately this test did not catch all the obscurities in the tasks.

Furthermore, some tasks could have been constructed better. An example of this is the dialogue tree task, instead of constructing a class that participants would use, an interface could have been formulated. This would have implicitly clarified the purpose of the class. \tmcc{This section is a bit sketchy}

\paragraph{Sample Size \& Test Duration}
The user test was conducted with six participants outlined in \secref{par-crit}, this was inline with the \champagne methodology. In order to analyse the test results qualitative methods where employed, therefore the number of participants is sufficient to find 80\% of usability issues\cite{virzi1992refining}. However, there is no consensus on the optimum number of participants for such a usability study\cite{hwang2010number}. However, the participants only had an hour to complete exercises in both \fs and \cs. This meant that most participants only managed to complete a single test case in each language, which provides limited insight into the programming ability of the participant. Some test cases focus on certain aspects of the programming language's idiom, such as recursion, which some participants where unfamiliar or inexperienced with.

These problems could be mediated by allocating more time to the tests, however taking up an hour of the participants day, presents scheduling issues already. Therefore extending the tests where not an option. In order to mediate this issue somewhat the test restructured so that participants worked with \fs in 40 minutes and \cs in 20 minutes, based on the pilot test. The assumption was that the participants where experienced in \cs and new to \fs. \tmcc{Weak arguments, reevaluate later}


\subsubsection{External Validity}
This section explores the potential threats originating from outside the test itself. This means any threat that affect or skew the results that is not directly related to the test conduction. An example of this is whether the test tests the right things and whether the findings are broadly applicable or not.

\paragraph{Applicability to Game Development}
The test cases, the tasks participants where asked complete, where designed to mimic game development scenarios. These cases where constructed to showcase situations where conventional and functional solution strategies could compete. However, the realism of the these cases is questionable. We do not have industrial game development experience and therefore our experiences may not represent the reality.

Some participants commented on the realism of the test cases. One participant noted that the character controller tasks where not representative of real development since they are freely available and even if they weren't the developer would buy one from the asset store. The other participants indicated that they had spent a lot of time writing character controllers.

\paragraph{Previous Experience}


\subsection{Champagne Prototyping}\label{sec:cham-app}
In this section we analyse the video files from the usability experiment using the champagne prototyping methodology. This method uses two different analysis strategies; attention investment and cognitive dimensions. In both of these analyses we count mentions of each feature or dimension. We present these using bullets in tables, which is recommended by \cite{blackwell2004champagne}. A closed bullet (\mn) means that a participant mentioned an aspect of the feature, when the participant mentioned it multiple times an open bullet is used (\mns). The categories, \textit{Correct \& Unprompted} and \textit{Correct \& Prompted} are instances where the participants mention or explain a feature correctly and/or positively. The last category are instances where features are mentioned negatively, incorrectly or is a cause for confusion.

\subsection{Attention Investment Model}\label{sec:att-inv-app}
The attention investment model, presented in \secref{attention-investment}, can be used to map out the programming steps undertaken by a user. This section will endeavor to do so, using the user test video material. Firstly, user comprehension of the feature being evaluated is measured. This feature is the use of \fsh and \gls{FRP} in gameplay programming. In order to measure comprehension, the instances where participants mention aspects of the feature were recorded and categorised, as can be seen in \tableref{comp-matrix}.

\makeTable{
	{| p {.20\textwidth} | p {.20\textwidth} | p {.20\textwidth} | p {.20\textwidth} |}
	\hline
	\textbf{Recognised Aspects} & \textbf{Correct \& Unprompted} & \textbf{Correct \& Prompted} & \textbf{Incorrect} \rowEnd
	Modularity 			& \mn\mn & & \mn \rowEnd
	ReactTo 				& \mns & \mn\mn\mns & \mn\mn\mn \rowEnd
	Types 					& \mn\mn & & \mn\mns\mns\mns\mns\mns \rowEnd
	List Operations	& \mn & \mn\mn\mns & \mn\mns \rowEnd
}{User comprehension of the \gls{FRP} features.}{comp-matrix}

In addition to a measure of the participants' comprehension, the attention investment model also provides a quantification of their efforts in the programming activity. This consists of four metrics, mentioned in \secref{attention-investment}. This can be seen in \tableref{att-inv-findings}. The risk metric measures the amount of times participants mentioned or discussed things that could go wrong, which includes increased difficulty of some tasks using \fsh. The cost metric is the attention and time required to switch to \fsh. Any musing over the difficulties of switching over is included. The payoff is the reduced cost of gameplay programming after switching to \fsh. The imperative alternative metric is the number of times participants mentioned the problems in \csh, or other imperative languages, that form the basis for switching to \fsh.

\makeTable{
	{| l | c |}
	\hline
	\multicolumn{2}{|c|}{\textbf{Attention Investment}} \rowEnd
	Investment Risk & \mn\mn\mn\mn\mns  \rowEnd
	Investment Cost & \mn\mns\mns\mns\mns \rowEnd
	Investment Payoff & \mn\mns\mns\mns\mns \rowEnd
	Imperative Alternative & \mn\mn\mn \rowEnd
}{Attention investment findings.}{att-inv-findings}

The participants were able to correctly use and describe \fsh and \gls{FRP} behaviour in some instances and struggled in other instances. As can be seen in \tableref{comp-matrix}, not all participants were cognisant of all feature aspects, e.g. all participants misunderstood or struggled with the type system. Functional programming claims a greater degree of modularity than imperative languages\cite{hughes1989functional}, however, less than half of the participants expressed cognisance of this. Some participants wrote much more modular code in \fsh than in \csh, but did not mention it.

As can be seen in \tableref{att-inv-findings} the participants noted a high cost with a high payoff. Participants could see the usefulness of \gls{FRP}, but several participants expressed uncertainty of any benefit provided by \fsh. Another point is that very few participants noted the problems with existing solutions, even when compared to \fsh. The high cost is attributed to loss of productivity while a developer learns to use \fsh.


\subsection{Cognitive Dimensions}\label{sec:cog-dim-app}
In this section we use the cognitive dimensions framework to aid the analysis of the video from the usability test. In this analysis we count the number of times the participants experience or mention problems with each dimension. We use the same notation that was presented in the previous section. The frequency of mentions can be seen in \tableref{cog-dim-findings}. The mentions counted are any instances where the participants said or did some thing that fit under a cognitive dimension. Many of the counted instances are therefore not explicit statements made by the participants, but rather instances where their actions fell under one of the categories.%Furthermore, the mentions are not necessarily negative, but might also reflect instances where the user made use of the given dimension.

\makeTable{
	{| l | l | l |}
	\hline
	\textbf{Dimension} & \textbf{F\#} & \textbf{C\#} \rowEnd
	Abstract Gradient & \mn & \mn \rowEnd
	Closeness of Mapping & \mn\mn &  \rowEnd
	Consistency & \mn\mn\mns\mns\mns & \mn\mn \rowEnd
	Diffuseness/Terseness & \mn\mn\mns & \mn \rowEnd
	Error-proneness & \mn\mns\mns\mns\mns & \mn\mn \rowEnd
	Hard Mental Operations & \mn\mn\mns & \mns \rowEnd
	Hidden Dependencies & \mn\mn & \mn \rowEnd
	Premature Commitment & \mn\mn\mn & \rowEnd
	Progressive Evaluation & \mn\mn\mns & \mn\mn \rowEnd
	Role-expressiveness & \mn\mn\mns\mn & \mn\mn \rowEnd
	Secondary Notation and Escape from Formalism & & \rowEnd
	Viscosity & \mn & \mn \rowEnd
	Visibility and Juxtaposability & \mn\mns & \mn \rowEnd
}{Cognitive Dimensions Findings}{cog-dim-findings}

It is not surprising that participants encountered more problems with \fs than they did with \cs. Participants were selected based on the criteria stated in \secref{par-crit}, these included requirements for \cs experience and limited to no  \fs experience. This allowed us to study the potential adoption difficulties of \fs. The higher frequency of mentions when using \fs is inline with this.

While an overview can be gained from \tableref{cog-dim-findings}, it does not provide adequate information of \textit{why} a participant would undertake a certain strategy or action. Therefore, selected instances among the mentions, are analysed further in the following sections.

\subsection{Instance Analysis}
Some instances during the test made the difficulties of using \fs and the functional paradigm more clear than others. In this section, these instances are explored and examined in order to discover the problems faced by developers adopting \fs in a gameplay programming setting.% Each instance is discussed and explored under the dimension it belongs to.

\subsubsection{Consistency}\label{sec:part-cons} % p6 20:15
All participants had prior \cs experience which affected their expectations. This was apparent when participants applied \cs methodology in the \fs code. An example of this is the confusion of types some participants experienced. Several participants noted that they preferred strict typing or specifying types manually. An example of this can be seen in \lstref{type-conf}.

\begin{listing}[H]
\begin{minted}{fsharp}
[<SerializeField>]
let mutable _velocity = 5.0f
[...]
member this.HandleMoveForward() =
  this.transform.position+=new Vector3(0,0,this._velocity)
\end{minted}
\caption{Problem experienced with types in F\#. The \m{Vector3} constructor accepts \m{float}s and are invoked with \m{int}-parameters.}
\label{lst:type-conf}
\end{listing}

In \lstref{type-conf} the participant correctly types the \ttt{\_velocity} variable on line 2. However, when attempting to set the object's position on line 5, the participant first uses \ttt{0} instead of \ttt{0.0f}. This valid in \cs, but not in \fs. This is an instance of confusion surrounding the automatic type inference of variables and the typing of literals.

\begin{listing}[H]
\begin{minted}{fsharp}
let moveMagneticBalls (objs:GameObject[]) (center:GameObject) =
  objs center |> Array.map (fun i ->
    i.transform.LookAt(center.transform)
    i.transform.Translate(i.transform.forward * Time.deltaTime * speed))
\end{minted}
\caption{Closure misunderstanding. The user attempts to catch \m{center} in the closure by piping it into the map-function.}
\label{lst:clos-mis}
\end{listing}

Some participants also experienced issues with closures. In \lstref{clos-mis} a participant has defined a function to move a number of objects towards a center point. The center point, \ttt{center}, is passed as a parameter, but the participant became confused as to how to pass it to the lambda expression. Therefore center was added on line 2 after \ttt{objs} and piped into the \ttt{map} function. This causes an error because \ttt{objs center} is now a function invocation to a non-existent function, \ttt{objs}, with the argument \ttt{center}. Passing \ttt{center} to the lambda function is unnecessary, because it is captured in the lambda's closure.

The behaviour described is consistent within \fs, but not with the expectations of the participant. Arguably, this instance is a product of the participant's inexperience with functional programming, however it is an example of the disharmony between the largely consistent rules of \fs and the expectations of programmers. These consistency issues were primarily present in the \fs code, which is not surprising considering the participants' experience.

\subsubsection{Role-expressiveness}
In F\# the last expression in a function body is implicitly its return. In some cases that should not be the case and the function should return \ttt{Unit} (equivalent to C\#'s \ttt{void}). This can be done by adding \ttt{()} as the last line of the function body. This confused some of the participants, as they felt that it was unnecessary to explicitly indicate a non-existing return type. One of the participants even asked directly what \ttt{Unit} was and the monitor answered with an explanation. Approximately fifteen minutes later the participant encountered another \ttt{Unit}-type problem and was unable to recover without an additional explanation.

Another role-expressiveness problem we encountered was that some participants attempted to \ttt{let} declare multiple values without initialising them. Some participants assumed that \ttt{let} declarations without assignment would result in default values (such as \ttt{null} for classes). Another problem faced by the participants were related to class type declarations, where the participants did not realise that the brackets after the type's name could be used to pass constructor arguments.

A single participant also noted that he found F\# pipe operations \textit{\dquote{not nice to read}}. He stated that he would rather prefer the SQL-like variation of \gls{LINQ} in C\# because it is closer to plain English. The monitor asked if it was related to the names of the functions (\ttt{map} and \ttt{reduce} or \ttt{Select} and \ttt{Aggregate}), to which the participant stated that it was solely related to the way pipe operations were structured.
% Participant 6: Hvad betyder () - 4:30 + Unit!? - 20:30
% Default værdier på variable using let
% Participant 6: Type constructors
% Select vs. map: Participant 3, 06.30

\subsubsection{Secondary Notation}
In the test we saw very limited use of secondary notation. This is likely caused by the relatively small tasks, along with the fact that the code was not meant to be used in production nor expanded upon. The programmers, however, would often open documentation pop-up boxes to read about functions and methods before putting them to use. Generally, our intuition was that the programmers found it more easy to understand the C\# documentation than the F\# documentation. We suspect that there two reasons:
\begin{enumerate}
    \item The programmers were more experienced in C\#.
    \item F\# uses the arrow function signatures (e.g. \fsinline{val map : mapping : ('T -> 'U) -> list:'T list -> 'U list}, which is the signature of \ttt{List.map}) in its documentation. This notation indicates that a function of multiple parameters can be curried. Currying is not supported in C\# and was thus a concept that the programmers were unfamiliar with.
\end{enumerate}
\subsubsection{Viscosity}
% Participant 2, 40:40 samt 1:02:30
In our test cases, viscosity is particularly visible in the \dquote{concurrent}-update category. The reason for this is that the participants are asked to develop a sequential solution first, followed by a parallel implementation. Generally viscosity is low in both languages. In F\# we saw the magnetism task implemented using the pipe operator. Such an implementation can be extended to a parallel solution by piping into \ttt{Async.Parallel} and then \ttt{Async.RunSynchronously} (see \lstref{fsharp:pipe:async}). A similar solution can be achieved in C\# using \gls{LINQ}, albeit the change requires the programmer to delete a semicolon.

\begin{listing}[H]
    \begin{minted}{fsharp}
let speed = 3f;

let moveBallForward (ball:GameObject) =
    ball.transform.Translate(ball.transform.forward * Time.deltaTime * speed)

let Update () =
    let balls = GameObject.FindGameObjectsWithTag("Magnetic")
    balls
    |> Array.map moveBallForward
    ()

let UpdateAsync () =
    let balls = GameObject.FindGameObjectsWithTag("Magnetic")
    balls
    |> Array.map (fun b -> async {moveBallForward})
    |> Async.Parallel
    |> Async.RunSynchronously
    ()
    \end{minted}
    \caption{Transforming from sequential to concurrent list operations in F\#.}
    \label{lst:fsharp:pipe:async}
\end{listing}

As with many other dimensions, viscosity can also be affected by the programmer's style of programming. This is explified in \lstref{csharp:viscous}, which is taken from one of the solutions in C\#. In order to make implement \dquote{concurrent} update, the participant had to construct a new list in the \ttt{Update}-method and wrap the calls to the state methods in \ttt{Task.Run} (e.g. \mintinline{csharp}|updateTasks.Add(Task.Run(() => Flee(fleeingShooter)))|). This change is manageable, but imagine how much effort it would take if we wanted to add an additional state to the shooter units. If such change was to be implemented, the programmer would have to:
\begin{itemize}[H]
    \item Add an additional list and \ttt{foreach}-statement in \lstref{csharp:viscous}.
    \item Add an additional case to the switch-statement in \ttt{JoinState} in \lstref{csharp:viscous}.
    \item Change the signature of the \ttt{TransferState}-method in \lstref{csharp:viscous:transfer} to accept an additional list and add an addtitional \ttt{else if} for the third list.
    \item Add an additional case to the switch in \lstref{csharp:viscous:transfer}, and update the call to \ttt{RemoveFromList} in all other cases.
\end{itemize}
An alternative and less viscous solution was implemented by another participant in the test. We have listed that in \apxref{csharp:non:viscous}.

\begin{listing}[H]
    \begin{minted}{csharp}
class StateMachine : MonoBehaviour
{
    [...] //Pre-implemented code

    private List<Shooter> fleeingShooters;
    private List<Shooter> movingShooters;
    private List<Shooter> attackingShooters;

    public void JoinState(Shooter shooter, State state)
    {
        switch(state)
        {
            case State.Fleeing:
                fleeingShooters.Add(shooter);
                break;
            case State.Moving:
                movingShooters.Add(shooter);
                break;
            case State.Attacking:
                attackingShooters.Add(shooter);
                break;
            default:
                break;
        }
    }

    private void Update()
    {
        foreach(var fleeingShooter in fleeingShooters)
        {
            Flee(fleeingShooter);
        }
        foreach (var movingShooter in movingShooters)
        {
            Move(movingShooter);
        }
        foreach (var attackingShooter in attackingShooters)
        {
            Attack(attackingShooter);
        }
    }

    [...] //TransferState
    [...] //RemoveFromList

    [...] //methods for each unit state
}
    \end{minted}
    \caption{Example of viscous C\# implementation of the Unit Management Test.}
    \label{lst:csharp:viscous}
\end{listing}

\begin{listing}
    \begin{minted}{csharp}
public void TransferState(Shooter shooter, State state)
{
    switch (state)
    {
        case State.Fleeing:
            fleeingShooters.Add(shooter);
            RemoveFromList(shooter, ref movingShooters, ref attackingShooters);
            break;
        case State.Moving:
            movingShooters.Add(shooter);
            RemoveFromList(shooter, ref fleeingShooters, ref attackingShooters);
            break;
        case State.Attacking:
            attackingShooters.Add(shooter);
            RemoveFromList(shooter, ref movingShooters, ref fleeingShooters);
            break;
        default:
            break;
    }
}
    \end{minted}
    \caption{TransferState-method, which is part of the viscous Unit Management implementation from \lstref{csharp:viscous}.}
    \label{lst:csharp:viscous:transfer}
\end{listing}

\begin{listing}
    \begin{minted}{csharp}
private void RemoveFromList(Shooter shooter, ref List<Shooter> list1, ref List<Shooter> list2)
{
    if (list1.IndexOf(shooter) != -1) {
        list1.Remove(shooter);
        return;
    }
    if(list2.IndexOf(shooter) != -1) {
        list2.Remove(shooter);
        return;
    }
}
    \end{minted}
    \caption{RemoveFromList-method, which is part of the viscous Unit Management implementation from \lstref{csharp:viscous}.}
    \label{lst:csharp:viscous:remove}
\end{listing}


\subsection{Threats to Validity} \label{sec:validity}
In this section we will examine potential sources or errors and other threats to validity. Such threats are categorised as either an internal threat or an external threat. An internal threat occurs when data is mishandled, misinterpreted or in some other way skewed to such a degree that the results are untrustworthy. The second category consists of errors caused by the data being inapplicable to other cases. This inline with terminology outlined in \cite{mcleod:validity}.


\subsubsection{Internal Validity}
In this section the internal validity is explored. The goal of this section is map out the possible  shortcomings originating from data handling and interpretation. Each potential threat is explained in turn and actions undertaken to mediate the threat are outlined. In some cases a threat cannot be sufficiently mediated, in which case it is simply listed.

\paragraph{Evaluation Parameters}
The user tests where evaluated in accordance with the champagne prototyping method (see \secref{champagne}). This methodology is intended to be used to measure the usability of a single feature, not an entire programming language. Therefore it was modified to fit our case better. The methodology consists of two other usability techniques: cognitive dimensions and attention investment models (see \secref{cog-dim} and \secref{attention-investment}). We modified the cognitive dimensions aspect to focus on the languages as a whole and kept the feature focus of attention investment. Thus the evaluation parameters where centered on the users experience with \fs and their understanding of the \gls{FRP} system.


The cognitive dimensions framework is originally intended to  estimate the usability, or provide vocabulary to such an estimation, of a notational system such as a programming language. Therefore our modification of champagne prototyping is to use cognitive dimensions for its original purpose. Furthermore, a single aspect of discount method for language evaluation was used, namely the sample sheet (see \secref{discount-method}). This was employed to assist test participants with \fs.

\paragraph{Task Difficulty}
% Er opgaverne i kategorierne lige svære. Jeg tror foreksempel at Unit Management har været sværere at overskue for forsøgskaninerne end Magnetism.

\paragraph{Task Presentation}
% I samme boldgade: Dialog tree opgaven er dårligt formuleret. Vi kunne have lavet et interface der hedder INode, som skulle implementeres på en ny type i F#.

\paragraph{Sample Size}
A test setup with six participants and eight test cases does provide limitations.

\subsubsection{External Validity}
\metasheep

\paragraph{Applicability to Game Development}

\paragraph{Previous Experience}

%Generalisability

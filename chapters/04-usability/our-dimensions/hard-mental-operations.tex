\subsubsection{Hard Mental Operations}
The hard mental operations dimension defines how often incomprehensible expressions occur in the code. Hard mental operations often occur in conjunction with boolean expressions\cite{green1996usability}.

Boolean expressions are expressed similarly in C\# and F\#, with the only exception that C\# uses \ttt{!} to negate expressions, whereas F\# uses the \ttt{not} function. Boolean expressions in both languages are equally hard to read. We have illustrated examples in \lstref{hard:mental:operations}. In this example the reader may incorrectly assume that the \ttt{\&\&} operator is evaluated before \ttt{!}, and that the expression is evaluated as \csinline{!(expr1 && expr2)}. The exact same case is present in F\#. Programmers may reduce perceived ambiguity by inserting parentheses, but may risk changing the order of evaluation by doing so.

\begin{listing}[H]
    \begin{minipage}{.45\textwidth}
        \begin{minted}{fsharp}
if(!expr1 && expr2) {
    //[...]
}
        \end{minted}
    \end{minipage}
    \hfill
    \begin{minipage}{.45\textwidth}
        \begin{minted}{csharp}
if not expr1 && expr2 then
    //[...]
        \end{minted}
    \end{minipage}
\caption{Hard mental operations illustrated using boolean expressions in C\# and F\#.}
\label{lst:hard:mental:operations}
\end{listing}

In F\# it is optional for the programmer to indicate types when he is writing functions. Sometimes it may be necessary to verify that the compiler's inference is correct by looking at the deduced function signature. We argue that this may present hard mental operations in both languages, as the function signatures will quickly get incomprehensible as the number of arguments grow. Take for example the function signature of the \ttt{ReactTo} function that we wrote as part of the \gls{FRP} plugin for Unity: 
\newline
\fsinline{member FRPBehaviour.ReactTo : event:FRPEvent * condition:('T0 -> bool) * handler:('T0 -> unit) -> unit}.\newline
We should underline that this function uses tupled arguments, because we wanted to overload it with a function to unconditionally react to events. But even without tupled arguments the definition would have been:\newline
\fsinline{member FRPBehaviour.ReactTo : event:FRPEvent -> condition:('T0 -> bool) -> handler:('T0 -> unit) -> unit}.\newline
In C\# the equivalent would have been:\newline
\csinline{public void ReactTo(FRPEvent event, Func<T, bool> condition, Action<T> handler)}\newline
Whether one or the other is easier to comprehend than the other is a matter of opinion. The problem is more prominent in F\#, however, due to the type inference in the compiler.
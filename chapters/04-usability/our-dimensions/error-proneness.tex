\subsubsection{Error-proneness}
Errors produced by the programmer fall into one of two categories, either the error is a slip or a mistake. Slips are instances where the programmer knows what to do, but did something else by accident and mistakes are instances when a programmer makes a logical error. These errors can be exacerbated by language features.

According to \cite{green1996usability}, textual programming languages are inherently more error-prone than visual languages. The given examples are implicit declaration, line-ends and delimiters. Implicit declarations are not applicable in either \cs or \fs, however line-ends are used in both languages. \cs uses \m{;} to denote line-ends whereas \fs uses newlines. A C-family programmer may find it difficult to overview \fs code for this reason. In \cs a \m{;} need not be followed by a line-end and depending on the type of statement, practices may differ (it is not typical to break lines in \ttt{for}-loop declarations, but it is after variable declarations).
%The \fs approach of using newlines and indentation to delimit scopes and lines uses fewer characters and promotes more readable code \needcite\tmc{Jeg er ikke enig i denne observation, men hvis der findes en kilde som siger det kan vi godt lade den stå.}. An example of this can be seen in \lstref{con-comp}.

%\begin{listing}[H]
%\begin{minted}{csharp}
%class Person(string name, DateTime, birthday)
%{
%  public string Name { get; } = name;
%  public DateTime Birthday { get; } = birthday;
%  public int Age() =>
%    DateTime.Today.Subtract(birthday).Days / 365;
%}
%\end{minted}
%\begin{minted}{fsharp}
%type Person(name: string, birthday: DateTime ) =
%  member this.Name =
%    name
%  member this.Birthday =
%    birthday
%  member this.Age() =
%    let daysDiff = DateTime.Today.Subtract(birthday).Days
%    daysDiff / 365
%\end{minted}
%\caption{Conciseness Comparison, examples taken from \cite{wlaschin2017FsharpForCsharpProgrammers}.}
%\label{lst:con-comp}
%\end{listing}\tmc{Er vi sikker på at den her skal være under Error-proneness?}

The strong type system used in \fs can also cause unexpected errors. While the system prevents some errors down the line, C-family programmers would expect type coercion to assist with operations on integer values of different sizes. We gave an example of this in \lstref{type-incompat}, which yields an error because an \ttt{int} and \m{int16} are multiplied. This causes initial errors, but may prevent type errors later in development\cite{ray2014large}.

Programmers may inadvertently change the parameters of a function by separating parameters using commas (see \lstref{para-err}). In most cases this is valid \fs and compiles, however the function now takes a single tuple parameter. This mistake can occur without the programmer noticing any difference, until they have to invoke the function. At this point the invocation has changed from \fsline{add 2 4}(two parameters) to \fsline{addTupled (2, 4)}(tuple parameter). Both declarations are valid, but the latter may cause confusion when the programmer attempts to use the functions as higher-order, as he would have to pack all arguments in tuples. We have listed examples in \lstref{para-err}.

\begin{listing}[H]
\begin{minted}{fsharp}
// Multiple parameters
let add x y = x + y

// Single parameter
let addTupled (x, y) = x + y

//Sum list without tupled parameters
let sum = [1..10] |> List.reduce add

//Sum list with tupled parameters
let sum = [1..10] 
|> List.reduce (fun acc elm -> addTupled (acc, elm)) 
\end{minted}
\caption{Examples of functions with and without tupled parameters and it's influence on their applications as higher-order.}
\label{lst:para-err}
\end{listing}

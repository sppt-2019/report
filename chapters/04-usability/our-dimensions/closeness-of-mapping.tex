\subsubsection{Closeness of Mapping}
The measure of how close to the problem domain a language can get is called the closeness of mapping. In order to solve a real problem, the problem must be expressible in the language and the closer the language is to the real world, the easier it is to express\cite{green1996usability}. Textual programming languages are abstractions over the real problem domain and therefore often do not map directly to the domain.

Both languages have mechanisms to model the problem domain. In object-oriented programming, the world is represented as objects and the objects are abstracted over via classes\cite{kindler2011object}. Functional programming models the problem as behaviour (functions) which are applied to data\cite{hughes1989functional}. The advantage of the object-oriented approach is that the object abstraction comes quite close the problem domain.

\quoteWithCite{The object expresses the user's view of reality [...]}{Object Oriented Analysis \& Design}{mathiassen2000object}

This approach is in contrast to the functional paradigm which models reality mathematically. This approach is not as close as the object model, however mathematical modelling of the world is widespread in many different fields of study.

\quoteWithCite{Typically the main function is defined in terms of other functions, which in turn are defined in terms of still more functions, until at the bottom level the functions are language primitives. These functions are much like ordinary mathematical functions [...]}{John Hughes}{hughes1989functional}

The abstraction models of the languages are the tools used by the programmers to model the world. \cs uses a model, which lends itself more to closeness of mapping, but both languages make use of custom types and naming which allow programmers to mold their programs in accordance with the problem domain. Additionally both languages support each other's modelling approach. An example of this can be seen in \lstref{tree-imps}, where a recursive class in \cs can be implemented via custom types in \fs.

\begin{listing}[H]
  \begin{minted}{fsharp}
type Talent(strength, intellect, agility) =
  member val Strength = strength with get, set
  member val Agility = agility with get, set
  member val Intellect = intellect with get, set

type Tree =
| Node of TalentValue:Talent * Children:Tree list
| Leaf of TalentValue:Talent

let rec foldTree folding init tree =
  match tree with
  | Node (t, _, c) ->
    let f = folding t init
    let cf =
      c
      |> List.map (foldTree folding init)
      |> List.fold folding init
    folding f cf
  | Leaf (t,_) ->
    folding t init

let sumPickedNodes root =
  foldTree (fun t1 t2 ->
    Talent(t1.Strength + t2.Strength, t1.Intellect + t2.Intellect, t1.Agility + t2.Agility)) (Talent(0,0,0)) root
  \end{minted}
  \begin{minted}{csharp}
public class MyTalent
{
  public int Strength;
  public int Agility;
  public int Intelligence;

  public List<MyTalent> SubTalents = new List<MyTalent>();

  public MyTalent SumTalents() {
    var t = new MyTalent(Strength, Agility, Intelligence);
    if(SubTalents?.Count == 0)
      return t;

    foreach (var child in SubTalents) {
      var childTalentValues = child.SumTalents();
      t.Strength += childTalentValues.Strength;
      t.Agility += childTalentValues.Agility;
      t.Intelligence += childTalentValues.Intelligence;
    }
    return t;
  }
}
  \end{minted}
  \caption{Talent tree data structure and walker implementations (F\# on top, C\# below).}
  \label{lst:tree-imps}
\end{listing}

The example in \lstref{tree-imps} implements a class and a discriminated union in \fs, which together implement the behaviour of the \cs class. The \fs approach has separated the tree from the talent, where in the  \cs solution the tree emerges from the recursive nature of the class. The \cs solution is closer to the problem domain, but the \fs solution is closer to the mathematical concept of trees.\tmc{Skal vi skrive noget om at man også godt kunne lave en generisk træ-struktur i C\#?}

\subsubsection{Viscosity}
% Participant 2, 40:40 samt 1:02:30
In our test cases, viscosity is particularly visible in the \dquote{concurrent}-update category. The reason for this is that the participants were asked to develop a sequential solution first, followed by a concurrent implementation. Generally viscosity is low in both languages. In F\# we saw the magnetism task implemented using the pipe operator. Such an implementation can be extended to a concurrent solution by piping into \ttt{Async.Parallel} and then \ttt{Async.RunSynchronously} (see \lstref{fsharp:pipe:async}). A similar solution can be achieved in C\# using \gls{LINQ}.

\begin{listing}[H]
    \begin{minted}{fsharp}
let speed = 3f;

let moveBallForward (ball:GameObject) =
    ball.transform.Translate(ball.transform.forward * Time.deltaTime * speed)

let Update () =
    let balls = GameObject.FindGameObjectsWithTag("Magnetic")
    balls
    |> Array.map moveBallForward
    ()

let UpdateAsync () =
    let balls = GameObject.FindGameObjectsWithTag("Magnetic")
    balls
    |> Array.map (fun b -> async {moveBallForward})
    |> Async.Parallel
    |> Async.RunSynchronously
    ()
    \end{minted}
    \caption{Transforming from sequential to concurrent list operations in F\#.}
    \label{lst:fsharp:pipe:async}
\end{listing}

As with many other dimensions, viscosity can also be affected by the programmer's style of programming. This is exemplified in \lstref{csharp:viscous} through \lstref{csharp:viscous:remove}, which is taken from one of the solutions in C\#. In order to implement \dquote{concurrent} update, the participant had to construct a new list in the \ttt{Update}-method and wrap the calls to the state methods in \ttt{Task.Run} (e.g. \mintinline{csharp}|updateTasks.Add(Task.Run(() => Flee(fleeingShooter)))|). This change is manageable, but imagine how much effort it would take if we wanted to add an additional state to the units. If such change was to be implemented, the programmer would have to:
\begin{itemize}
    \item Add an additional list and \ttt{foreach}-statement in \lstref{csharp:viscous}.
    \item Add an additional case to the switch-statement in \ttt{JoinState} in \lstref{csharp:viscous}.
    \item Change the signature of the \m{RemoveFromList} in \lstref{csharp:viscous:remove} to accept another list and add an additional \m{else if} for the third list.
    \item Add an additional case to the switch in \lstref{csharp:viscous:transfer}, and update the call to \ttt{RemoveFromList} in all other cases.
\end{itemize}
An alternative and less viscous solution was implemented by another participant in the test. We have listed that in \apxref{csharp:non:viscous}.

\begin{listing}[H]
    \begin{minted}{csharp}
class StateMachine : MonoBehaviour
{
    [...] //Pre-implemented code

    private List<Shooter> fleeingShooters;
    private List<Shooter> movingShooters;
    private List<Shooter> attackingShooters;

    public void JoinState(Shooter shooter, State state)
    {
        switch(state)
        {
            case State.Fleeing:
                fleeingShooters.Add(shooter);
                break;
            case State.Moving:
                movingShooters.Add(shooter);
                break;
            case State.Attacking:
                attackingShooters.Add(shooter);
                break;
            default:
                break;
        }
    }

    private void Update()
    {
        foreach(var fleeingShooter in fleeingShooters)
        {
            Flee(fleeingShooter);
        }
        foreach (var movingShooter in movingShooters)
        {
            Move(movingShooter);
        }
        foreach (var attackingShooter in attackingShooters)
        {
            Attack(attackingShooter);
        }
    }

    [...] //TransferState
    [...] //RemoveFromList

    [...] //methods for each unit state
}
    \end{minted}
    \caption{Example of viscous C\# implementation of the Unit Management Test.}
    \label{lst:csharp:viscous}
\end{listing}

\begin{listing}
    \begin{minted}{csharp}
public void TransferState(Shooter shooter, State state)
{
    switch (state)
    {
        case State.Fleeing:
            fleeingShooters.Add(shooter);
            RemoveFromList(shooter, ref movingShooters, ref attackingShooters);
            break;
        case State.Moving:
            movingShooters.Add(shooter);
            RemoveFromList(shooter, ref fleeingShooters, ref attackingShooters);
            break;
        case State.Attacking:
            attackingShooters.Add(shooter);
            RemoveFromList(shooter, ref movingShooters, ref fleeingShooters);
            break;
        default:
            break;
    }
}
    \end{minted}
    \caption{TransferState-method, which is part of the viscous Unit Management implementation from \lstref{csharp:viscous}.}
    \label{lst:csharp:viscous:transfer}
\end{listing}

\begin{listing}
    \begin{minted}{csharp}
private void RemoveFromList(Shooter shooter, ref List<Shooter> list1, ref List<Shooter> list2)
{
    if (list1.IndexOf(shooter) != -1) {
        list1.Remove(shooter);
        return;
    }
    if(list2.IndexOf(shooter) != -1) {
        list2.Remove(shooter);
        return;
    }
}
    \end{minted}
    \caption{RemoveFromList-method, which is part of the viscous Unit Management implementation from \lstref{csharp:viscous}.}
    \label{lst:csharp:viscous:remove}
\end{listing}

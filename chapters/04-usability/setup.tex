\subsection{Setup}
The test setup draws inspiration from the Champagne Prototyping method and Discount Method for Language Evaluation. The tasks, prototype and participants were selected according to the former, whereas the use of a cheat-sheet for the participant was inspired by the latter. Contrary to the suggestion of using a text editor in Discount Method for Language Evaluation, we chose to use \glspl{IDE}, as we're testing well established languages. We allowed the users to choose between JetBrain's Rider and Microsoft's Visual Studio, depending on what they were used to.

For each participant a one and a half hour session was planned. We expected that actual coding time would be roughly one hour, as we conducted a questionnaire before the test to learn about the test participants' experience with Unity and an interview after the test to allow the participant to share their opinion on F\# in Unity, C\# and functional programming. Out of the one hour coding time, we intended to use 20 minutes on C\# and 40 minutes on F\#, as the participants were required to have experience with C\# in Unity and therefore likely would complete the C\# tasks faster.

The cheat-sheet served a two-fold purpose and was made available online\footnote{\url{https://sppt-2019.github.io/unity-fsharp-introduction/}. Please note that the document is in Danish, as all participants were Danes.} using Github pages. The first purpose of the document was to give the test participants an introduction to F\# in Unity prior to the test and second to act as a cheat-sheet during the test. Similarly, the tasks were also made available online during the test\footnote{\url{https://sppt-2019.github.io/unity-fsharp-introduction/tasks/}. Please note that the document is in Danish, as all participants were Danes.}. We also created a Github repository, in which the test-setups were stored\footnote{\url{https://github.com/sppt-2019/Unity-FSharp}}. The master branch of said repository holds a Unity project with eight scenes, one for each of the test cases. The purpose of this setup was to remove Unity as a factor in the experiment and avoid having the participants spend time setting up scenes. For each of the test participants we created a new branch in the repository, which would allow us to view each of the participants' code in isolation.

During the tests we recorded the screen and audio on the test computer. The files were not transcribed, but whenever we quote one of the participants we refer to transcriptions of larger pieces of dialogue that are listed in \appendixref{user:quotes}. We do so in order to give the reader more context on the quotes and avoid \dquote{plucking} sentences out of their context.

\subsubsection{Test Cases} \label{sec:usability:test:cases}
We created a total of eight test cases, which fall into four categories; player controller, generalised walkers, map-reduce and \dquote{concurrent} update. We list concurrent with double ticks here, as the tasks were designed to be implemented with \ttt{async}/\ttt{await} in C\# and asynchronous workflows in F\#. The reason we chose this was that it would be possible for the participants to implement a concurrent version with minor rework, given the right setup. Unity's concurrency model requires more boilerplate code, which we deemed would not provide any useful information.

\begin{labeling}{\quad\quad}
    \item[FPS Controller] The participant is to implement a \gls{FiPS} controller, i.e. a component which can be added to a player character to move it around the world with the WASD-keys and rotate the camera with the mouse.
    \item[3rd Person Controller] This test case is similar to the previous test case, only that the camera must rotate around the player character.
    \item[Talent Tree-Walker] The participant is first asked to implement a data structure that can hold a talent tree. Afterwards the player is given a pre-made talent tree from which the participant must calculate two things. At first a character's bonuses in three attributes should be calculated by summing all talents that are \ttt{Picked} and afterwards the maximum achievable bonus of all attributes by summing all talents.
    \item[Armour Graph] In this test case, the participant is given a list of armour equipped on a character. The participant is to implement code that sums the character's bonus in three attributes from the armour. In the second half of the test, we assume that certain pieces of armour can scale the attributes from all other pieces of armour and ask the participant to calculate the scaled attribute bonuses.
    \item[Dialogue Tree] The participant is to first implement a data structure that can hold a dialogue tree. Afterwards the participant must parse a list of dialogue options into a tree using his own data structure. Finally the participant is to find all unique dialogue paths that has a certain outcome.
    \item[Currency] The participant is presented with three different types of coin. He must first implement code that exchanges a given number of said coins into the minimum number of total coins. Afterwards he should add a function to calculate whether a player can buy a certain item from a vendor and finally implement code that buys the item and updates the player's wallet.
    \item[Unit Management (RTS)] The participant is to implement an inverse state machine. By inverse we mean that the state machine should hold collections of entities for each state in the state machine. At each \ttt{Update} the state machine should map the corresponding state's update-function over each collection to create new collections of entities. Finally the participant is asked to implement a \dquote{concurrent} mapping of the update-function.
    \item[Magnetic objects] The participant is to simulate magnetism. He is presented with a list of objects, some of them which are magnetic. All magnetic objects should be attracted to a common center-point at a given speed. In the second half of this task the participant is asked to implement a \dquote{concurrent} version of the simulation.
\end{labeling}

%In \tableref{task:categories} we list the categories and their associated tasks. We had initially intended that each test case should be implemented by two different participants in C\# and F\#, but after the first session we could conclude that each participant would only be capable of solving one task. We therefore to a fixed ordering.

 \makeTable{
     {| l | c | c |}
     \hline
     \textbf{Category} & \textbf{F\# Task} & \textbf{C\# Task} \rowEnd
     Player Controller & FPS Controller & 3rd Person Controller \rowEnd
     Generalised Walkers & Talent Tree & Armour Graph \rowEnd
     Map-Reduce & Dialogue & Currency \rowEnd
     \dquote{Concurrent} Update & Magnetism & Unit Management \rowEnd
 }{Categories and their associated tasks}{task:categories}

 In \tableref{task:categories} we list the categories and their associated tasks. As we wanted to explore how suitable Carmack's and Sweeney's approaches to game development are, we decided to use a fixed ordering of the test categories for each participant. We had initially planned that the participants would be given one of the tasks from the controller category as the first task and distribute the remaining semi-randomly, but discovered after the first actual test that the participants would only have time to solve a single task. We therefore decided that the participants would be given starting tasks from different categories. This ordering allow us to compare each of the participants' the solutions to tasks of the same category.

\subsubsection{FRP System}\label{sec:frp-sys}
We chose to add support for F\# via the Unity F\# Integration plugin\cite{fsharp2019plugin}. The test cases are to be implemented in a \gls{FRP} system that we developed in F\#. The \gsl{FRP} system introduces a new class called \m{FRPBehaviour}. Classes that inherit from \m{FRPBehaviour} inherit a method called \m{ReactTo}, which exists in two variations. The first variations accepts an event type (such as Update, Keyboard or MouseMove) and a handler. This method will unconditionally invoke the handler whenever an event of the given type occurs. The second variation accepts an event type, a filtering function and a handler. This variation will only invoke the handler when the filtering function returns true. \lstref{frp:example} shows an example of the magnetism task implemented in F\#. This code rotates all balls to look at the venter and thereby moves them forward.

\begin{listing}[H]
    \begin{minted}{fsharp}
member this.Start() = 
    let balls = GameObject.FindGameObjectsWithTag("Magnetic")

    this.ReactTo (FRPEvent.Update,
        (fun () -> 
            let center:Vector3 = getCenter(balls)
            let updatedBalls = 
                balls
                |> Array.map (fun b -> lookAt(b, center))
                |> Array.map step
        )
    )
    \end{minted}
    \caption{Implementation of the magnetism task in F\#. The \m{getCenter} and \m{lookAt} functions are excluded for brevity.}
    \label{lst:frp:example}
\end{listing}

The F\# plugin also allows programmers to implement gameplay code using the standard Unity strategy (i.e. a chain of if-else statements to check for input in the \m{Update}-method). We decided that we wanted to remove this option entirely and therefore implemented an \m{Update} method in the \m{FRPBehaviour} that was reserved for condition checking. This method cannot be overridden, which entirely prevents the user from using \m{Update}-based programming. We took this decision partly because scientific research underlines that \gls{FRP} is well suited for game development\cite{courtney2003yampa,cheong2005functional,maraffi:frp} and partly because we argue that the temptation of writing C\# code in F\# syntax would be too large.

\subsubsection{Participants} \label{sec:par-crit}
The participants for this experiment were recruited by sending emails to game studios and other game-related companies in Denmark, asking if their employees were interested in participating in the experiment. The participants were required to have experience with C\# and Unity in the game development industry. We gathered a total of six participants; two participants with Indie game-development experience, three from a company that creates \gls{AR}/\gls{VR} applications and one who teaches children game development.

For selection criteria we looked for participants that had developed games or had significant experience with C\# in Unity. In addition the participants need not have experience with F\#. The criteria were ranked as follows:
\begin{enumerate}
	\item Game development experience.
	\item Unity \& C\# experience.
	\item No F\# experience.
\end{enumerate}

\subsubsection{Pilot Test}
We conducted a pilot test with a participant from the Programming Technology specialisation course at Aalborg University in order to get some feedback on the setup. The test participant has experience with Unity from Indie game development and teaching Unity programming to children in the Danish secondary school/advanced level (not to be confused with our test participant, who has a similar job). Contrary to the requirements for the actual test, the pilot test participant had experience with F\# and functional programming in general.

Prior to the pilot test we had assumed that a 50/50 distribution between C\# and F\# would give the most valuable results, as it would not skew the results time-wise. During the pilot test the participant, who had prior experience with F\#, was able to complete three tasks in C\# and two in F\#. We therefore decided to change the distribution to 20 minutes of C\# and 40 minutes of F\#.

Apart from that, the participant noted that some of the tasks were hard to understand, mainly because they required the participant to utilise code that was pre-implemented. We decided to rewrite the tasks and insert code snippets with existing classes where-ever the participant would have to use them. Finally, the participant noted that the function-name \ttt{ReactTo}, which is used to bind an event handler to a given event, was odd. He suggested renaming it to \ttt{Bind}. We chose to ignore this suggestion, with the argument that the sound of \ttt{ReactTo} has a strong connection with \gls{FRP}, whereas \ttt{Bind} has a stronger connection with conventional functional programming.

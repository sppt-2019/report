\chapter{Code Examples} \label{app:terse-diff-comp}
\begin{listing}[H]
\begin{minted}{csharp}
using System;
using System.Collections.Generic;
using UnityEngine;

public class ArmourBehaviour : MonoBehaviour
{
    // Start is called before the first frame update
    void Start()
    {
        var Armour = Item.Exercise1();
        var GroupedArmour = Item.Exercise2();

        Solution1(Armour);
        Solution2(GroupedArmour);
    }

    public void Solution1(IEnumerable<Item> Armour)
    {
        int totalAgi = 0;
        int totalStr = 0;
        int totalInt = 0;

        foreach (var item in Armour)
        {
            totalAgi += item.Agility;
            totalStr += item.Strength;
            totalInt += item.Intellect;
        }
        Debug.Log($"Exercise 1\n\t" +
                  $"Agility: {totalAgi}\n\t" +
                  $"Strength: {totalStr}\n\t" +
                  $"Intellect: {totalInt}");
    }
}

\end{minted}
\caption{Armour Graph test case implemented in \cs, Part 1.}
\label{lst:armour-graph-cs-1}
\end{listing}

\begin{listing}[H]
\begin{minted}[firstnumber=35]{csharp}
public void Solution2(List<Group> GroupedArmour)
{
    GroupedArmour = Item.Exercise2();

    var groupTotals = new List<Tuple<ItemGroup, int, int, int>>();

    foreach (var group in GroupedArmour)
    {
        int totalAgi = 0;
        int totalStr = 0;
        int totalInt = 0;

        float agiMod = 0.0f;
        float strMod = 0.0f;
        float intMod = 0.0f;

        foreach (var item in group.Items)
        {
            totalAgi += item.Agility;
            totalStr += item.Strength;
            totalInt += item.Intellect;

            agiMod += item.AgilityMod;
            strMod += item.StrengthMod;
            intMod += item.IntellectMod;
        }

        totalAgi = (int) (totalAgi * agiMod);
        totalStr = (int) (totalStr * strMod);
        totalInt = (int) (totalInt * intMod);

        groupTotals.Add(new Tuple<ItemGroup, int, int, int>(group.GroupName, totalAgi, totalStr, totalInt));[firstline=35]
    }

    string res = "Exercise 2\n";
    foreach (var tuple in groupTotals)
    {
        res += $"{tuple.Item1}\n\t" +
                  $"Agi: {tuple.Item2}\n\t" +
                  $"Str: {tuple.Item3}\n\t" +
                  $"Int: {tuple.Item4}\n";
    }
    Debug.Log(res);
}

// Update is called once per frame
void Update()
{

}
\end{minted}
\caption{Armour Graph test case implemented in \cs, Part 2.}
\label{lst:armour-graph-cs-2}
\end{listing}

\begin{listing}[H]
\begin{minted}{fsharp}
namespace UnityFunctional
open UnityEngine
open System

type FRP_ArmourBehaviour() =
    inherit FRPBehaviour()

    let sum (triplet1:int*int*int) (triplet2:int*int*int) =
        let (a1, b1, c1) = triplet1.Deconstruct()
        let (a2, b2, c2) = triplet2.Deconstruct()
        (a1+a2,b1+b2,c1+c2)

    let fSum (triplet1:float32*float32*float32) (triplet2:float32*float32*float32) =
        let (a1, b1, c1) = triplet1.Deconstruct()
        let (a2, b2, c2) = triplet2.Deconstruct()
        (a1+a2,b1+b2,c1+c2)

    let totalStats (armour:Item[]) =
        armour
        |> Array.map (fun a -> (a.Agility, a.Intellect, a.Strength))
        |> Array.reduce (fun acc elm ->
            sum acc elm)

    let scaledStats (groups:Group list) =
        groups
        |> List.map (fun g ->
                      List.ofSeq(g.Items)
                      |> List.map (fun i -> (i.Group,(i.AgilityMod, i.IntellectMod, i.StrengthMod),(i.Agility, i.Intellect, i.Strength)))
                      |> List.reduce (fun acc elm ->
                          let (gr,aMod,aStats) = acc.Deconstruct()
                          let (gr,md,stats) = elm.Deconstruct()
                          (gr,(fSum aMod md), (sum aStats stats))))
        |> List.map (fun i ->
            let (grp,(agiMod,intMod,strMod),(agi,int,str)) = i.Deconstruct()
            (grp, float32(agi) * agiMod, float32(int)*intMod, float32(str) * strMod))
\end{minted}
\caption{Armour Graph test case implemented \fs, Part 1.}
\label{lst:armour-graph-fs-1}
\end{listing}

\begin{listing}[H]
\begin{minted}[firstnumber=36]{fsharp}
    member this.Start() =
        let i = ItemStore.AllItems()
        let (agi, int, str) = totalStats(i)
        Debug.Log("Agility: " + agi.ToString())
        Debug.Log("Intellect: " + int.ToString())
        Debug.Log("Strength: " + str.ToString())

        let g = ItemStore.GroupedItems()
        let ss = scaledStats(g)

        ss
        |> List.iter (fun gs ->
            let (gr, sAgi,sInt,sStr) = gs.Deconstruct()
            Debug.Log(gr.ToString() + "\n" + String.Join("\n", ["Agility: " + sAgi.ToString(); "Intellect: " + sInt.ToString(); "Strength: "+ sStr.ToString()])))
\end{minted}
\caption{Armour Graph test case implemented in \fs, Part 2.}
\label{lst:armour-graph-fs-2}
\end{listing}

\section{A Less Viscous Implementation of Unit Management} \label{app:csharp:non:viscous}
\begin{listing}
\begin{minted}{csharp}
class StateMachine : MonoBehaviour
{
    [...] //Pre-implemented code, such as JoinState

    public List<Shooter> shooterList = new List<Shooter>();
    public List<State> stateList = new List<State>();

    public void JoinState(Shooter shooter, State state)
    {
        shooterList.Add(shooter);
        stateList.Add(state);
    }

    public void TransferState(Shooter shooter, State state)
    {
        if (shooterList.Contains(shooter))
        {
            stateList[shooterList.IndexOf(shooter)] = state;
        }
    }

    private void Update()
    {
        for (int i = 0; i < shooterList.Count; i++)
        {
            switch (stateList[i])
            {
                case State.Attacking:
                    Attack(shooterList[i]);
                    break;
                case State.Fleeing:
                    Flee(shooterList[i]);
                    break;
                case State.Moving:
                default:
                    Move(shooterList[i]);
                    break;
            }
        }
    }

    [...] //methods for each unit state
}
\end{minted}
\caption{A less viscous implementation of the unit management test case.}
\label{lst:um:less:viscous}
\end{listing}

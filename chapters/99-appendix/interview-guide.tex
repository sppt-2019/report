\chapter{Interview Guide}
The interview structure is based on the \textit{scenario-based interview} technique presented in \cite{blackwell2004champagne}. The scenario in this case is the task being solved by the test participant. The interview is conducted during the test itself. The questions asked are open and may be used to lead the participant towards the relevant problem, but not to lead them to answers. The questions should be designed to clarify that the prototype is being tested and not the participant themselves. With this in mind the following questions are posed.

\begin{enumerate}
  \item Is it clear how \gls{FRP}-reactions work?
  \item Do you consider \gls{FRP} advantageous? Why?
  \item Would you use this tool in a production environment?
  \begin{enumerate}
    \item What if \gls{FRP} had a performance impact?
    \item What if \gls{FRP} had a performance gain?
  \end{enumerate}
\end{enumerate}

All the posed questions are open and participants were encouraged to engage in dialogue as a response. Their comments were recorded and at a later date, codified and analysed. The theoretic justification for each question is summarised in the remainder of this chapter.

Question 1 attempts to query the participant's understanding of \gls{FRP}, without placing the focus on the participant. This can mitigate the pressure experienced by the test participants and can therefore lead to clearer results. Determining how well the \gls{FRP} paradigm is understood is directly related to answering research question 1 and 3. Furthermore the participants may provide insight into how the \gls{FRP} system can be made easier to understand.

Question 2 ascertains whether the participants consider the paradigm shift worthwhile. The participants will be exploring both new programming and game development paradigms. Therefore their perception of benefits affords insight into the usability of the paradigm. This question attempts to answer research question 2.

The third is an inquiry into research question 3. The participant is directly asked about, in their opinion, the viability of such an approach in a production environment. Their response is followed up by either question a or b. If the participant believes that the system can be employed in a production environment, then question a is asked, otherwise they are asked question b.

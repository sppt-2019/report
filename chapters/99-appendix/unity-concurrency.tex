\chapter{Concurrency in Unity}\label{app:unity:conc}
\begin{listing}[H]
    \begin{minted}{csharp}
    public class BulletJobSystem : MonoBehaviour {
        //This array holds all bullets that need to be moved
        private TransformAccessArray _transforms;
        //This job handle may be used to synchronise with previously scheduled move jobs
        private JobHandle _bulletMoveHandle;
        public GameObject BulletPrefab;
    
        //This struct implements the actual job.
        public struct MoveBulletsJob : IJobParallelForTransform {
            public float _moveSpeed;
            public float _deltaTime;
    
            //In the Execute-method we define how we want to apply an update to a single bullet.
            public void Execute(int index, TransformAccess transform) {
                transform.position += _moveSpeed * _deltaTime * (transform.rotation * new Vector3(0,0,1));
            }
        }
    
        void Start() {
            //Create an array with 0 elements and unlimited capacity
            _transforms = new TransformAccessArray(0, -1);
        }
    }
    \end{minted}
    \caption{Implementation of a job that moves bullets forward in Unity. \m{SpawnBullet} and \m{Update} listed in \lstref{unity:job:example:2}.} \label{lst:unity:job:example}
    \end{listing}

\begin{listing}
    \begin{minted}{csharp}
public void SpawnBullet(Transform shooter) {
    //Instantiate the bullet as usual in Unity
    var bullet = Instantiate(BulletPrefab);
    bullet.transform.position = shooter.position;
    bullet.transform.forward = shooter.forward;

    //Synchronise with the job and add the new bullet to the _transforms array.
    _bulletMoveHandle.Complete();
    _transforms.capacity++;
    _transforms.Add(bullet.transform);
}

void Update() {
    //Complete the job, i.e. only apply one update at a time
    _bulletMoveHandle.Complete();

    //And schedule a new bullet update
    var bulletMoveJob = new MoveBulletsJob (5f, Time.deltaTime);
    _bulletMoveHandle = bulletMoveJob.Schedule(_transform);
    JobHandle.ScheduleBatchedJobs();
}
    \end{minted}
    \caption{Spawning bullets and updating when using Unity's C\# Job System, part of \lstref{unity:job:example}.}
    \label{lst:unity:job:example:2}
\end{listing}

    \begin{listing}
        \begin{minted}{csharp}
//MoveForward component. Defines data that is needed to move forward
struct MoveForward {
    public float Speed;
}

//C# Job system that moves forward in parallel
public class MoveForwardSystem : JobComponentSystem {
    //The generic types of the interface implements the filter
    struct MoveForwardJob : IJobForEach<Translation, Rotation, MoveForward> {
        private readonly float _deltaTime;

        //Move the entity forward. We do not have access to "standard" Unity methods, so we use the dedicated math-library
        public void Execute(ref Translation trans, [ReadOnly]ref Rotation rotation, [ReadOnly]ref MoveForward moveForward) {
            var forward = math.forward(rotation);
            var step = math.mul(moveForward.Speed, _deltaTime);
            var moveVec = math.mul(forward, new float3(step));
            trans.Value += moveVec;
        }
    }
}

//On each update schedule the movement
protected void JobHandle OnUpdate(JobHandle inputDeps) {
    var job = new MoveForwardJob{_deltaTime = Time.deltaTime};
    return job.Schedule(this, inputDeps);
}
        \end{minted}
        \caption{Implementation of an \gls{ECS} that moves bullets forward in Unity.} \label{lst:unity:ecs:example}
        \end{listing}
\subsubsection{Abstract Gradient}
The abstract gradient is measured from abstraction hating, through abstraction tolerant, to abstraction loving. The abstractions measured are the notations' ability to group elements and refer to them as a single entity. Most modern textual-programming languages make extensive use of abstraction and functional languages even more so\cite{hudak1989conception}. \fs is a functional-programming language with object-oriented features allowing for extensive abstractions.

On the other hand \cs is an object-oriented language which supports functional features. This means that \cs also supports extensive abstraction. The main difference lies in the fact that \cs is object-oriented programming first and \fs is functional programming first. Functional programming tends to make use of abstraction more frequently, however that does not mean that other paradigms do not use abstraction\needcite.

Considering the high level of abstraction in both languages they will both be considered abstraction loving in this report. An important detail is that while \fs generally relies on finer grain abstraction and \cs on broader abstraction, they each support the other's abstraction level\tmc{Kan vi finde noget om data abstraction vs. functional abstraction som Bent snakkede om?}.

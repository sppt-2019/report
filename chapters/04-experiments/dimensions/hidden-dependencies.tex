\subsubsection{Hidden Dependencies}
Hidden dependencies discuss how many relationships there are between components, that are not visible from atleast one of the components. We discuss several problems in this section, but we must underline that some of these problems can be mitigated by using an \gls{IDE}, as they often allow programmers to trace dependencies.

The problem of hidden dependencies is prominent in both C\# and F\#, though in two different flavours. We have listed an example in \lstref{hidden:dependencies} that showcase the problems. In F\# the problem is present on the function level. This is because F\# programmers are allowed to write functions that are directly nested in a module. These functions may be imported into another module or namespace with the \ttt{open} keyword. Given that a name of a function does not collide, the function may be refered to without the use of its fully qualified name. In C\# the problem occurs because programmers are allowed to reference code in base classes without using their fully qualified name. One example of this in Unity is that any \ttt{MonoBehaviour} may directly call \ttt{Destroy}, which is actually a static method on the \ttt{GameObject}-class. This may wash out the distinction between static and non-static methods and \dquote{hide} the base class from the reader. In F\# programmers are required to give fully qualified names when dealing with classes.

\begin{listing}[H]
    \begin{minipage}{.50\textwidth}
        \begin{minted}{csharp}
class Base {
    protected static string Method() {
        return "Method";
    }
}

class Inherited : Base {
    private string GetString() {
        return Method() + " is the string";
    }
}
        \end{minted}
    \end{minipage}
    \hfill
    \begin{minipage}{.40\textwidth}
        \begin{minted}{fsharp}
//Imagine this is in another file
module X =
    let function () =
        "function"

open X
module Y =
    let function2 () =
        function() + "2"
        \end{minted}
    \end{minipage}
\caption{Hidden dependencies in function/method calls in C\# and F\#.}
\label{lst:hidden:dependencies}
\end{listing}

In both languages the problem of function/method hiding may occur. An example of this is if a base class defines a \ttt{virtual} method in C\# or an \ttt{abstract} method with default implementation in F\#. This method method may be hidden further down the inheritance tree by implementing a function with the same name without using the \ttt{override} keyword. This will give a warning on compile-time, which can be removed by supplying the \ttt{new} keyword in front of the method that hides the existing implementation.

% goto statements in C#
C\# has a modified version of the \ttt{goto}-statement, which traditionally allow programmers to jump to labels anywhere in the source code. In C\#, however, the jumps are restricted to either a label in the same scope or in an enclosing scope\cite{csharp:goto}. Nevertheless, it may be hard to comprehend the exact target of a \ttt{goto} if it is deeply nested in multiple loops. According to a StackOverflow discussion\cite{goto:stack:overflow}, it seems that the \ttt{goto} statement sees limited use in practise.

% mutability
% Unity Editor reference assignment
The problem of hidden dependencies may also occur if a component is dependant on a global variable. In games it is common to see such types of dependency\cite{blow2004game, guana2015building, nystrom2014game}. This problem is easier to mitigate in F\# because the global variable is per default immutable and the developer has to explicitly indicate if he wants a mutable variable. In C\# it's easier to make slips, as everything is mutable per default. In Unity this problem is also present in-between components, as it's a common pattern to declare a field on a class that references some component and thereby assign that field from Unity's Inspector\cite{unity:inspector:assignment}. This has the consequence that there is no way of knowing which components depend on each other by inspecting the source code. Furthermore, it is not possible to trace dependencies backwards in Unity without writing custom Editor scripts\cite{unity:dependencies:backwards}.
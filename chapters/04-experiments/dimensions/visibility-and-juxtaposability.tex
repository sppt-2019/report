\subsubsection{Visibility and Juxtaposability}
Visibility and Juxtaposability determines whether required material is accessible without cognitive work\cite{green1996usability}. In textual languages this dimension is not necessarily determined by the language, but more so by the environment (\gls{IDE}).

During the sessions all the participants chose to use Visual Studio. Visual Studio supports numerous ways of making source code available. Examples of such are:
\begin{itemize}
    \item Splitting the text editor both horizontally and vertically, such that two files can be open side-by-side
    \item Jumping to implementations by control-clicking. 
    \item Hovering function or type names to read a description of what they're meant for. 
\end{itemize}
As Visual Studio was used in both C\# and F\#, the differences are minimal. It is interesting, however, that none of the Visibility and Juxtaposability enchancing features were used during the tests. This may have two possible explanations:
\begin{enumerate}
    \item The tests were relatively small-scale and required minimal interaction with existing code.
    \item When existing code was needed, the participants would refer to the cheat-sheet or task descriptions, where example code was given. Some participants actually chose to open those documents side-by-side with Visual Studio.
\end{enumerate}
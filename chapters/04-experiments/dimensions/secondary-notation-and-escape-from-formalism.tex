\subsubsection{Secondary Notation and Escape from Formalism}
Secondary notation and escapte from formalism defines how well a programming environment supports conyeing information that is not part of the source code. Typical examples of such are comments, indentation and grouping code into paragraphs in textual languages\cite{green1996usability}.

C\# and F\# are very alike in this dimension. They both support the \ttt{//}-operator, which indicates that the rest of the line should be commented out and matching pairs of \ttt{/*} and \ttt{*/} which comments everything out between them. Furthermore functions, methods, classes and more or less any program construct may be annotated with \ttt{///}-comments, which allow the programmer to add \gls{XML}-documentation\cite{fsharp:xml:doc}. The programmer may use this to describe the intend of the construct along with its arguments and, if needed, link to other constructs in the program. Other developers may open this documentation in a pop-up box elsewhere in the program, whenever such construct is encountered.
\chapter{Experimenting with Parallelism}
In this chapter we conduct a series of experiments with parallel programs. First and foremost \cite{DBLP:journals/cl/Tremblay-parallel} claims that the lenient evaluation strategy provides high implicit parallelism by evaluating function arguments in parallel with the function body. This claim is put to the test using two binary tree benchmarks that was presented in the same article.

The results from the benchmarks indicate that there is a certain threshold of task-sizes, which must be exceeded before parallelisation has a positive effect on execution time. We attempt to estimate this threshold in \secref{crit:work} using two tests; a busy-wait delay estimation and matrix summation.

\section{Benchmarks}
In this section we explore the performance of the lenient evaluation strategy and how well it parallelises. The reason for this is that the interest in it has been lacking to say the least. We could not find a reason as to why and decided to explore whether it was because \cite{DBLP:journals/cl/Tremblay-parallel} gave false promises.

\subsection{Test cases}
\cite{DBLP:journals/cl/Tremblay-parallel} presents two test cases that we reuse in this experiment:

\begin{labeling}{\quad\quad}
    \item[Binary Tree Sum] where a tree-walker sums the values of all leaves of a binary tree.
    \item[Binary Tree Accumulation and Sort] where a tree-walker flattens all values of a tree's leaves into a list and sorts it.
\end{labeling}

The difference between the two test cases is that there are no data-dependencies in the summation test, i.e. we can calculate the results of the left subtree and the right subtree in parallel. In the Binary Tree Accumulation and Sort benchmark the tree must be traversed in a left-to-right order and thus there is a data-dependency between the results from the left and right subtree.

\subsection{Implementations}
In this benchmark we decided to use C\#, as we were familiar with the language. Furthermore, we discovered in previous work that C\#'s performance in the dotnet runtime is comparable to that of C++ in the context of microbenchmarks\cite{p92018gameplay}. We are well aware that C\# is not lenient, but we wanted to compare the strategies within the same language runtime and therefore attempted to map the lenient evaluation strategy onto C\#'s task system. \lstref{lenient:to:task} gives a code example how to do so.

\begin{listing}[H]
\begin{minted}{csharp}
//Lenient function, body evaluated asynchronously from caller
//parameters evaluated asynchronously from function body
int CalculateC(int a, int b) {
    int c = a * 50; //Implicit synchronisation with a-thread
    return c + b;   //Implicit synchronisation with b-thread
}

//C# mapping of the CalculateC-function
//async method that runs asyncronously from the caller
async Task<int> CalculateC_Lenient(Task<int> a, Task<int> b) {
    int c = await a * 50;   //Explicit synchronisation with a-task
    return c + await b;     //Explicit synchronisation with b-task
}
\end{minted}
\caption{Lenient evaluation in C\#} \label{lst:lenient:to:task}
\end{listing}
The mapping shown in \lstref{lenient:to:task} involves wrapping all arguments to methods in \ttt{Task}s and explicitly synchronise two \ttt{Task}s whenever there is a data-dependency between them. This means that all values of arguments are calculated asynchronously from the method body. The method is marked as \ttt{async}, meaning that a \ttt{Task} is spawned for every invocation of the method, i.e. it runs asyncronously from the caller. This leaves the majority of the footwork to the dotnet task scheduler, which must figure out in which order to execute them.

We implemented the test cases in three different synchronisation models:
\begin{labeling}{\quad\quad}
    \item[Sequential] which uses divide and conquer on one unit of execution to do all the calculations. This provides the baseline speed for the computation on a single core.
    \item[Fork-Join] which also uses divide and conquer, but in each recursion step, two \ttt{Task}s are spawned to compute the results of the left and right subtree. The \ttt{Task}s are then synchronised at the end of each recursive call. This provides a \textit{\dquote{classic parallel}} baseline speed.
    \item[Lenient] using the mapping displayed in \lstref{lenient:to:task}, i.e. wrapping everything in \ttt{Task}s and have the .NET Core task scheduler figure out the computation order.
\end{labeling}

\subsection{Test Setup}
According to \cite{sestoft2013microbenchmarks} the most reliable results are obtained when the tests are repeated multiple times and the average and standard deviation calculated for the results. We therefore decided to take the average execution speed over 100 repetitions for varying problem sizes, starting with 10 leaf nodes and gradually increasing the number up to a total of 10,000 leaves.

The tests were run on a laptop, of which the specifications are listed in \tableref{sys-specs}.

\makeTable{
{| l | R{6em} | p{3em} |}
\hline
\multicolumn{3}{| c |}{\textbf{Processor}} \\ \hline
Model & \multicolumn{2}{| c |}{Intel Core i7 4702HQ} \\ \hline
Clock Frequency & 2.2 & GHz \\ \hline
Max Turbo & 3.2 & GHz \\ \hline
Physical & 4 & Cores \\ \hline
Logical\footnotemark & 8 & Cores \\ \hline
\multicolumn{3}{|c|}{\textbf{Memory}} \\ \hline
Memory Size & 16 & GiB  \\ \hline
Memory Speed & 1600 & MHz \\ \hline
Memory Type &  \multicolumn{2}{| c |}{DDR3L 1600} \\ \hline
\multicolumn{3}{|c|}{\textbf{Software}} \\ \hline
Operating System & \multicolumn{2}{| c |}{Ubuntu 18.04 64bit}  \\ \hline
C\# runtime & \multicolumn{2}{| c |}{dotnet 2.2.104} \\ \hline
}{System specifications.}{sys-specs}\footnotetext{Logical cores are sometimes called threads. However logical cores is used here to avoid confusion with the software concept; threads, which is distinct from hardware threads.}

\subsection{Results}
The results are plotted in \figref{binary-accumulation} and \figref{binary-summation} (and listed in \tableref{accumulation:res} and \tableref{summation:res} in \apxref{benchmark:data}).

\newcommand{\binAcumSymbolics}{\symbolic{Problem,10,100,1000,10000,100000}}
\barChart*[15][\binAcumSymbolics]{Binary Accumulation}{binary-accumulation}{
    \plotDataWithError{Sequential}{\accumulationData}
    \plotDataWithError{Fork Join}{\accumulationData}
    \plotDataWithError{Lenient}{\accumulationData}
}\btc{?? -in reference to problem size 100}
\barChart*[15][\binAcumSymbolics]{Binary Summation}{binary-summation}{
    \plotDataWithError{Sequential}{\summationData}
    \plotDataWithError{Fork Join}{\summationData}
    \plotDataWithError{Lenient}{\summationData}
}

Much to our surprise, the sequential implementation was actually the fastest in all cases. It seems that the overhead of spawning and synchronising \ttt{Task}s outweighs the performance gain of parallelisation, when the problem sizes are in the magnitude of additions and list appending. Furthermore, the Accumulation test case presented in \cite{DBLP:journals/cl/Tremblay-parallel} is a poor choice when it comes to parallelism, as it must traverse the tree in a left-to-right manner, meaning that the only things that can be executed in parallel is the recursive calls down the tree. Another interesting result is that the execution time of the lenient approach grows much faster with the problem size compared to both Fork Join and Sequential. We suspect that the advantages of parallel programming will be more prominent, as the amount of work in each unit of execution increases. We will research this hypthosis in greater depth in the following section.

\section{Parallel Overhead \& Performance}\label{sec:crit:work}
In this section we present results from an experiment that estimates the critical workload for each task. We also implement a matrix summation benchmark to determine how different parallelisation strategies handle matrices of increasing sizes.

\subsection{Estimating Critical Workload}
In this experiment we estimate the critical workload of C\#'s \ttt{Task}-system. By critical workload we mean the time each task must execute before it is worthwhile to spawn it, compared to a sequential solution. The reason for this exploration is that we found the sequential solution to be faster in the binary tree benchmarks presented in the previous section.

\subsubsection{Test Setup}
We use the Binary Tree Summation benchmark presented in the previous section with a minor modification: Every time the algorithm finds a \ttt{Node}, it busy-waits for a given amount of time to simulate work. The busy-wait was implemented with a loop, whose number of iterations is gradually halved until the sequential solution executes faster than the parallel. The hypothesis here is that the parallel solutions will be faster, because it is capable of busy-waiting multiple tasks at the same time.

We implemented the test in two variations. The first variation emulates a data dependency between the wait and the results the subtrees, i.e. the wait is intended to emulate a computation that must be carried out after the results of both subtrees have been computed. The other variation emulates a situation where the left and right subtree can be computed in parallel with the wait, i.e. no data dependency between the delay and the subtrees. The tree has a total of 60 leaf nodes. In addition to the binary tree summation, we also implemented a N-ary tree summation in the same variations as that of binary.

\subsubsection{Results}
The results are plotted in \figureref{crit-work-dep} and \figureref{crit-work-no-dep} (and listed in \tableref{binary:tree:with:bias:dependency} and \tableref{binary:tree:with:bias:no:dependency} in \apxref{crit:work:data}).

\newcommand{\workBiasSymbolics}{\symbolic{Work Bias (iterations),134217728,67108864,33554432,16777216,8388608,4194304,2097152,1048576,524288,262144,131072,65536,32768,16384,8192,4096,2048,1024}}
\lineChart{Critical workload with data dependency.}{crit-work-dep}{
    \plotData{Sequential}{\workWithDependencyData}
    \plotData{Fork Join}{\workWithDependencyData}
    \plotData{Lenient}{\workWithDependencyData}
}
\lineChart{Critical workload without data dependency.}{crit-work-no-dep}{
    \plotData{Sequential}{\workWithoutDependencyData}
    \plotData{Fork Join}{\workWithoutDependencyData}
    \plotData{Lenient}{\workWithoutDependencyData}
}
\lineChart{Critical workload with data dependency, N-ary tree.}{crit-work-dep-nary}{
    \plotData{Sequential}{\workWithDependencyDataNary}
    \plotData{Fork Join}{\workWithDependencyDataNary}
    \plotData{Lenient}{\workWithDependencyDataNary}
}
\lineChart{Critical workload with data dependency, N-ary tree.}{crit-work-no-dep-nary}{
    \plotData{Sequential}{\workWithoutDependencyDataNary}
    \plotData{Fork Join}{\workWithoutDependencyDataNary}
    \plotData{Lenient}{\workWithoutDependencyDataNary}
}

\makeTable{
    { c | c | c }
    & No data dependency & Data dependency \\\hline
    Binary & 1024 & 4096 \\
    N-ary & 2048 & 2048
}{Iterations of the busy-wait loop before the sequantial solution becomes the fastest.}{crit:work:iterations}

The number of iterations in the busy-wait loop before the sequential solution is faster than the concurrent is listed in \tableref{crit:work:iterations}. The graphs also underline that our hypothesis was correct. The parallel solutions grows slower than the sequential because they are capable of executing multiple busy-wais concurrently. Finally we notice that the lenient and fork join strategy lie very close in execution speed. This is a promising result for the lenient evaluation strategy, as it shows that it may be as fast as a traditional concurrency strategy.

Some conccurency models batch smaller jobs together to form larger jobs\needcite. Such batching may reduce the time spent context switching as thus increase the execution speed of the concurrent solutions. Such strategy is employed by Unity's C\# job system\cite{unity:csharp:job:system}.

\subsection{Matrix Summation}
In this section we execute a matrix summation benchmark. This benchmark measures the time it takes to sum all indices of a random $N x N$ matrix. This benchmark was implemented in different parallelisation strategies to explore how well they scale to increasing sizes of $N$:

\begin{labeling}{\quad\quad}
    \item[Sequential] utilises a double-nested for-loop to iterate over the matrix and sum the values. This benchmark provides a baseline value for running the computation on one thread.
    \item[Map Reduce] maps a function that sums each column over the matrix. The resulting list of column sums is then reduced to the overall sum of the matrix. In C\# we utilise the \gls{LINQ}-methods \ttt{Select}, \ttt{Sum} and \ttt{Aggregate}.
    \item[Parallel Foreach] uses a parallel loop to iterate over the columns of the matrix that may execute the summation of each column in parallel.
    \item[Tasks] is similar to parallel foreach, with the only exception that we manually spawn a \ttt{Task} that calculates the sum of each column.
\end{labeling}

We have not included a lenient-variation in this experiment, as an implementation in our C\# mapping would be largely equivalent to the Tasks-implementation (see \lstref{matrix-sum-csharp}). The most notable difference being that a lenient-evaluation strategy would most likely also construct the matrix in parallel with the summation. As the time it takes to construct a matrix is not included in the results here, this should have no effect on the validity of the results.
\begin{listing}
\begin{minted}{csharp}
public static async Task<long> SumTask(long[,] matrix)
{
    //Create an enumerable over the columns of the matrix
    var columns  = Enumerable.Range(0, matrix.GetLength(0));
    //Sum each column in parallel
    var sums = columns.Select(c => Task.Run(() =>
    {
        var sum = 0L;
        for(var i = 0; i < matrix.GetLength(1); i++)
        {
            sum = unchecked(sum + matrix[c, i]);
        }

        return sum;
    })).ToList();

    //Join the resuls and sum the sums of each column
    await Task.WhenAll(sums);
    return sums.SumUnchecked();
}
\end{minted}
\caption{Tasks implementation of Matrix Sum, largely equal to a lenient C\# mapping.} \label{lst:matrix-sum-csharp}
\end{listing}

As the matrices are of size $N x N$, they contain a total of $N^2$ elements with random values between \ttt{Int64.Minvalue} and \ttt{Int64.MaxValue}. When running the test with large matrices we found that the result would overflow, which throws an exception because C\# is a managed language. In order to avoid this, we used the \ttt{unchecked}-keyword, which disables bounds-checking on an integral arithmetic operation\cite{csharp:unchecked}. \cite{csharp:unchecked} states that using \ttt{unchecked} \textit{\dquote{might improve performance}}, compared to checked integral arithmetic operations.

\subsubsection{Results}
The results are plotted in \figureref{linpack-summation}. The first thing to notice is that Map Reduce seems to be roughly equal to the sequential in running time. This could indicate that the \ttt{Select}-method of C\#'s \gls{LINQ}, which was used to implement Map Reduce, does not parallelise its iterations. We will thus treat Map Reduce as a sequential solution for the rest of this result discussion.

\newcommand{\linpackSymbolics}{\symbolic{Problem Size,2,4,8,16,32,64,128,256,512,1024,2048,4096}}
\barChart*[7][\linpackSymbolics]{Matrix Summation}{linpack-summation}{
    \plotDataWithError{Sequential}{\linpackData}
    \plotDataWithError{Map Reduce}{\linpackData}
    \plotDataWithError{Parallel Foreach}{\linpackData}
    \plotDataWithError{Tasks}{\linpackData}
}
In general, the results from this experiment is in alignment with those of the previous, in that there is an initial overhead associated with parallelisation. In this case, it seems the sequential and parallel solutions evens out at parallel job sizes of around 256 summations, after which point the parallel solutions are faster.

After overcoming the initial overhead, the parallel solutions handle increasing matrix sizes much better than their sequential counterparts. This is even more notable in \figureref{linpack-summation-line}, which plots the same data as a line and without logarithmic y-axis. As the matrix sizes continue to grow, it may be possible to split the columns in multiple separate tasks, possibly making parallel faster yet.

\lineChart{Matrix Summation}{linpack-summation-line}{
    \plotData{Sequential}{\linpackData}
    \plotData{Map Reduce}{\linpackData}
    \plotData{Parallel Foreach}{\linpackData}
    \plotData{Tasks}{\linpackData}
}

\section{Performance}
In this chapter we examine the performance of F\# in Unity, mainly using the \gls{FRP}-extension that was developed as part of this project. We first examine the aspect of \gls{GC} by looking at Unity best-practise guidelines, which suggests that garbage is to be avoided to the extend that it's possible. We were not aware of this when we implemented the reference solutions. We therefore adapt one of the solutions to conform to Unity best-practise guidelines and benchmark that against the more \dquote{careless} implementation. Finally we research concurrency in Unity \tmc{Der skal nok puttes lidt mere ind her.}.

\subsection{Unity Garbage Collection}
In this section we first examine best pratices for developing applications in Unity with a particular focus on garbage. We then list different \gls{GC} algorithms, briefly characterise them and investigate which algorithms are used in Mono, dotnet and Unity. Finally we measure the running times of F\# against C\# in Unity and a functional map-based approach against an imperative one.

\subsubsection{Best Practices}
Unity recommends careful memory management when writing in C\# and avoiding unnecessary heap allocations\cite{unity:optimisation}. The performance optimisation guideline in \cite{unity:optimisation} lists many common performance bottlenecks for Unity developers. The most notable of those are lack of caching and extensive use of boxing. Unity provides many methods and properties that allow developers to access collections of components, such as the \ttt{GameObject.FindObjectsWithTag} method and \ttt{Mesh.vertices} property\cite{unity:optimisation, unity:heap}. The implementation of those methods will allocate a new array for the objects behind the scenes every time they're invocated. We list an example given in the best practise guidelines as \ttt{Wrong} in \lstref{unity:array:prop}\cite{unity:heap}. In the example \ttt{mesh.veritices[i]} might seem like an innocent property access, but each time the property is accessed a new array been allocated. This means that the code allocates four new arrays in every iteration of the loop. This puts a huge burden on the \gls{GC} and will, according to \cite{unity:heap}, result in notisable performance degredation. Instead, developers should use the code listed as \ttt{Correct} in \lstref{unity:array:prop}, which does the exact same, but only allocates one array for all iterations, due to better use of caching\tmc{Skal vi tilføje noget om at de her problemer skyldes at Unity bruger en hacket version af .NET?}.

The problems highlighted in \lstref{unity:array:prop} are an instance of common subexpression elimination, and one could speculate whether or not Unity's C\# compiler should be capable of performing such optimisations. Nevertheless, Unity's best practise guidelines list them as an example and explains how developers should transform their code manually\cite{unity:heap}.

\begin{listing}
\begin{minted}{csharp}
//sample implementation of mesh.vertices
class Mesh {
    public Vector3[] vertices {
        get {
            var verts = new Vector3[/*number of vertices*/]
            //find the vertices and put them into verts
            return verts;
        }
    }
}

//Wrong
for(int i = 0; i < mesh.vertices.Length; i++)
{
    float x, y, z;

    x = mesh.vertices[i].x;
    y = mesh.vertices[i].y;
    z = mesh.vertices[i].z;

    DoSomething(x, y, z);
}

//Correct
var verts = mesh.vertices;
for (var i = 0; i < verts.Length; i++) {
    DoSomething(verts[i].x, verts[i].y, verts[i].z);
}
\end{minted}
\caption{Common performance bottleneck in Unity \cite{unity:heap}. \ttt{mesh.vertices} should be cached.} \label{lst:unity:array:prop}
\end{listing}

The problem of boxing occurs when a value-type should be used by refrence, for instance when constructing a list of integers or appending a float to a string. This generates a small ammount of garbage, which can quickly accumulate, e.g. during list iterations. Furthermore, \cite{unity:optimisation} underlines the importance of avoiding \gls{LINQ}-statements all together, due to the garbage generated under the hood. \cite{unity:heap} recommends avoiding coding styles that requires passing functions as arguments and to completely avoid closures, due to the ammount of garbage generated by said language constructs.

\subsubsection{Garbage Collection Algorithm}
Unity uses the Boehm–Demers–Weiser \gls{GC}, which is a conservative mark-sweep \gls{GC}\cite{unity:heap}, originally created for automatic memory management in C and C++\cite{boehm2007transparent}. Mark-sweep algorithms are the simplest type of \glspl{GC} and has the primary disadvantages that they halt computation while running, increase in execution time as more objects are allocated and may fragment memory\cite{sestoft2017programming}.

The dotnet runtime uses a generational \gls{GC} with three generations for smaller objects and a single generation for large objects\cite{dotnet:gc}. The younger generations are collected more often than the higher and all surviving objects are moved to the older generations. Each time an older generation is collected, all younger generations are also collected. Generational \glspl{GC} have the advantage that short-lived object allocations has a smaller performance penalty, but the disadvantage that they introduce additional overhead if old objects contain references to young objects\cite{sestoft2017programming}. Depending on the system the dotnet runtime may use different \gls{GC} strategies, including concurrent versions\cite{dotnet:gc}. Concurrent \glspl{GC} can collect garbage concurenly with the computation, meaning that \gls{GC} pauses are minimised or entirely removed\cite{dotnet:gc}.

Mono has previously used the Boehm–Demers–Weiser \gls{GC}, but has since moved to a concurrent, generational \gls{GC} called sgen\cite{mono:gc}. We have previously mentioned that Unity uses the Mono runtime, which may cause some confusion, so a clarification is due. Unity supports two different runtimes: Mono and IL2CPP. Unity's Mono runtime is a fork of the official Mono runtime\cite{unity:mono:github}, meaning that updates to the official Mono is not necessarily applied to the Unity's Mono runtime. The IL2CPP runtime \gls{AoT} compiles code in \gls{IL} to C++, which also uses the Boehm–Demers–Weiser \gls{GC}\cite{il2cpp:gc}. However, as part of Unity's 2019.1.0 release an experimental \textit{\dquote{incremental garbage collector, which should reduce stutters and time spikes}} was added\cite{unity:roadmap}.

\subsubsection{Functional Programming and Garbage Collection}
All these recommendations stand in direct contrast to the common practices employed in the pure functional programming paradigm. In functional programming it's typical to map over collections, which has two problems compared to this Unity performance guideline:
\begin{enumerate}
    \item map allocates a new collection instead of mutating the existing collection.
    \item map requires a function as one of the arguments, which defines what should happen to each of the elements in the collection.
\end{enumerate}
This practise also extends to other generalised constructs, such as the tree-walker employed in talents test case. These guidelines explain why Unity Technologies does not want to add F\# support despite over 3500 votes on their feedback forums in April 2018\cite{unity:fsharp}. The vote was later closed by Unity, without any explanation\footnote{In previous work we have cited the Unity forums to support this claim\cite{p92018gameplay}, but as of Feburary 2019 Unity has closed their feedback forums, meaning that this citation is no longer valid.}.

\subsubsection{Investigating Performance}
%Unity's performance guidelines regarding \gls{GC} seems to convey the message that functional-style programming should be avoided in Unity. However,
\gls{GC} is not the only thing that may affect performance in a managed language. There is also the problem of calling from the native (or unmanaged) code to the managed code. An investigation of Unity's integration with the managed runtime shows that a there is a considerable overhead in calling the pre-defined \ttt{MonoBehaviour}-methods (such as \ttt{Update}) in Unity 5.2.2\cite{unity:runtime:calls}. In \cite{unity:runtime:calls} the author sets up two different scenes:
\begin{enumerate}
    \item A scene containing 10,000 separate \ttt{MonoBehaviours} with an \ttt{Update}-method that increments a variable.
    \item A scene containing one \ttt{MonoBehaviour}, which contains an array of 10,000 objects. Each time the \ttt{Update}-method is called, the \ttt{MonoBehaviour} iterates through the 10,000 objects and calls a custom \ttt{MyUpdate}-method.
\end{enumerate}
On an iPhone 6 the first approach took an average of 5.4ms to update the 10,000 objects, whereas the second took 0.22ms\cite{unity:runtime:calls}. In the first approach only 0.4\% of the time is spent actually executing the \ttt{Update}-code, the remaining 99.6\% is spent doing sanity checks, iterating \ttt{MonoBehaviours} and instrumenting calls from the native code into the runtime.

\subsubsection{Test Setup}
The question then arises if the (potentially) increased overhead from \gls{GC} can be outweighed by having a single \ttt{MonoBehaviour}, manage several other behaviours in the same scene? In order to investigate, we reused the implementation of the Unit Management test case from the usability test (see \secref{usability:test:cases}). This solution is listed in \lstref{test:case:ai}. This test case may be solved by creating a collection of tuples: \ttt{(Unit, State)}. The state machine contains a series of unit management methods; one for each state. These methods take a unit as argument and returns a state. At each iteration the corresponding state's methods are mapped over the collection to create a new collection of game objects and their updated state, which is stored for the subsequent update. This approach avoids dealing with the problems of updating the list while iterating and potentially applying two updates to one \ttt{GameObject}. The advantage is that a single \ttt{MonoBehaviour} is in charge of updating all units in the \dquote{Realtime Strategy Game} and the disadvantage is that it generates substantially more garbage, as a new collection is allocated at each \ttt{Update}. We refer to this method as \dquote{Inverse} in the remainder of this section.

The other approach, here referred to as \dquote{Normal}, creates a \ttt{MonoBehaviour} for each unit, which contains it's own state machine. This has the advantage that we can exploit caching and generate less garbage, as suggested by Unity Technologies\cite{unity:optimisation}. It comes at the disadvantage that each unit must have its own \ttt{Update}-method, potentially introducing a large overhead\cite{unity:runtime:calls}.

\begin{listing}
\begin{minted}{csharp}
public void Update()
{
    //Apply updates and store the updated states in a list
    var newStates = _stateList.Select(s =>
    {
        switch (s.state)
        {
            case State.Fleeing:
                return Flee(s.entity);
            case State.Moving:
                return Move(s.entity);
            case State.Attacking:
                return Attack(s.entity);
            default: return (State.Moving, s.entity);
        }
    }).ToList();

    //zip the list with the old states to create tuples: (new state, old state)
    foreach (var statePair in newStates.Zip(_stateList, (sNew, sOld) => (sNew,sOld)))
    {
        //Compare old state and new state, initialise the unit for the new state if changed
        if (statePair.sNew.Item1 != statePair.sOld.state)
        {
            _initialiseState(statePair.sNew.Item1, statePair.sNew.entity);
            //Create a new list containing the updated unit
            _stateLise = _stateList.Select(s => s.entity == shooter ? (state, shooter) ? s);
        }
    }
}
\end{minted}
\caption{Possible solution for the Unit Management test cases.}
\label{lst:test:case:ai}
\end{listing}

\subsubsection{Methodology}
We decided to implement the two approaches in both C\# and F\#. We run the test using the il2cpp runtime under Unity 2019.1.0f2. In all test cases we used a \ttt{MonoBehaviour} written in C\# to measure the time between each \ttt{Update}-call, i.e. the time it takes to generate a frame. We decided to run the test in five setups with 500, 1000, 1500, 2000 and 2500 units. For each setup we generated 900 frames, as that corresponds to 15 seconds of gameplay at 60 \glspl{FPS}. Each measurement was added to a \ttt{HashSet}, which was written to a CSV file after the test. This means that the measurements include all game-related code, both including rendering, physics and so alike. However, as this system is ultimately going to be used to develop games, we conclude that delta time (or equivalently \gls{FPS}) is a good metric, as that is of utmost importance to the player.

The following research questions outlines the intend of the experiment:
\begin{itemize}
    \item How does the performance penalty from extensive garbage generation compare to the performance penalty from an incresed number of calls between unmanaged and managed runtimes?
    \item Does \gls{AoT}-compilation in the il2cpp runtime actually provide a speed up?
    \item Does the use of F\# (and \gls{FRP}) introduce an additional overhead?
    \item Unity introduced a new incremental \gls{GC} in Unity 2019.1. Does this new \gls{GC} provide a speed-up when using either C\# or \gls{FRP}?
\end{itemize}

\subsection{Results}
\metasheep

\subsubsection{Performance Penalty from Extensive Garbage Generation}
The results are listed in \tableref{unity:ai} and plotted in \figref{ai:benchmark}. The results indicate that F\# adds a small overhead, which increases as the number of units grow. This can be seen by comparing C\# Normal and F\# Normal. Furthermore, the C\# Inverse also adds a small overhead compared to C\# Normal. This could indicate that Unity has optimised the calls between native and managed code since v5.2.2. We also observe that the inverse FRP state machine performs notably worse than the other approaches as the number of units grow. We will go into greater depth as to why in the following section.

\begin{table}[H]
    \sisetup{round-mode=places}
    \rowcolors{1}{}{lightgray}
    \makebox[\textwidth][c]{
    \begin{tabular}{P{4cm} | S[round-precision=2] | S[round-precision=2] | S[round-precision=2] | S[round-precision=2]}
        \textbf{Number of Units} & \textbf{C\#} & \textbf{C\# Inverse} & \textbf{F\#} & \textbf{FRP Inverse}
        \csvreader[head to column names]{00-data/ai-benchmark.csv}
        {1=\strategy, 2=\csharp, 3=\cinverse, 4=\fsharp, 5=\frp} % <Column number>=<Macro>
        {\\\hline \strategy & \csharp & \cinverse & \fsharp & \frp}
    \end{tabular}}
    \caption{Average framerate when simulating the given number of units in Unity's Mono runtime.}
    \label{tab:unity:ai}
\end{table}

\barChart*[12][\symbolic{Strategy,500,1000,1500,2000,2500}][Average FPS][Number of Units]{Average FPS in Unit Management benchmark}{ai:benchmark}{
    \plotData{Csharp Normal}{\aiBenchmarkData}
    \plotData{Csharp Inverse}{\aiBenchmarkData}
    \plotData{Fsharp Normal}{\aiBenchmarkData}
    \plotData{FRP Inverse}{\aiBenchmarkData}
}

\subsubsection{Performance of Runtimes}
The results, listed in \tableref{unity:ai:runtime} and plotted in \figref{ai:benchmark:runtime}, show that il2cpp does not necessarily provide a speed up. We deem this as C\# Normal is faster in Mono, whereas il2cpp provides roughly 2 more \gls{FPS} in C\# Inverse and F\# Normal.

Another interesting observation we made during the test is that there is a very large spike in the time it takes to generate the third frame. This spike is around 20 times the time it takes to generate the other frames. One could explain this spike in Mono as runtime-optimisation, but as it is also present in il2cpp, which is \gls{AoT}, that cannot be the case. We do therefore not know what causes the spike.

\begin{table}[H]
    \sisetup{round-mode=places}
    \rowcolors{1}{}{lightgray}
    \makebox[\textwidth][c]{
    \begin{tabular}{P{4cm} | S[round-precision=2] | S[round-precision=2]}
        \textbf{Runtime} & \textbf{il2cpp} & \textbf{Mono}
        \csvreader[head to column names]{00-data/ai-benchmark-runtimes.csv}
        {1=\strategy, 2=\iltocpp, 3=\mono} % <Column number>=<Macro>
        {\\\hline \strategy & \iltocpp & \mono}
    \end{tabular}}
    \caption{Average framerate in Unity's two runtimes measured with 250 unites in the scene.}
    \label{tab:unity:ai:runtime}
\end{table}

\barChart[12][\symbolic{Strategy,Csharp Normal,Csharp Inverse,Fsharp Normal,FRP Inverse}][Average FPS][Strategy]{Average FPS in Unit Management benchmark}{ai:benchmark:runtime}{
    \plotData{Mono}{\aiBenchmarkRuntimesData}
    \plotData{il2cpp}{\aiBenchmarkRuntimesData}
}

\subsubsection{Performance of the FRP-system}
The results are plotted in \figref{ai:benchmark:overhead}. The results show that \gls{FRP} introduces additional overhead. This overhead results in a decrease of ten \gls{FPS} on average over the 900 frames. On the other hand, the \gls{FRP}-system yields a smoother curve, which means that the game will be subject to less stuttering and fewer lag spikes.

\lineChart[xtick={100, 200, 300, 400, 500, 600, 700, 800}][FPS][Frame No.]{FPS for each frame in FRP and C\# Inverse}{ai:benchmark:overhead}{
    \plotUnmarkedData{Csharp}{\aiBenchmarkOverheadData}
    \plotUnmarkedData{FRP}{\aiBenchmarkOverheadData}
}

The problems with the \gls{FRP}-system is that each \ttt{FRPBehaviour} is actually a full-blown \gls{FRP}-system, which could very well be refactored into a singleton.

Solving this problem would require a larger refactoring, as this relates to the Unity lifecycle of \ttt{GameObject}s. First and foremost some Unity methods are required to be tied to the \ttt{GameObject} they belong to. Examples of such methods are \ttt{OnCollisionEnter} and \ttt{OnTriggerExit}. Other methods, such as \ttt{Update} and reacting to keyboard strokes could, on the other hand, be tied to a \ttt{FRPEngine}. The problem here arises when \ttt{GameObject}s are destroyed. We have not added support to remove \ttt{FRPBehaviour}s from the \gls{FRP}-system, as the current version \dquote{cleans} up after itself when \ttt{GameObject}s are destroyed. This is to be understood in the sense that the whole system is deallocated and thus never risks invoking event handlers on objects that have been destroyed.

In order to truely determine whether or not \gls{FRP} comes with a performance penalty, these changes would have to be incorporated.

\subsection{Unity's Incremental Garbage Collector}
The results from running the C\# Normal implementation with the two different runtimes and \glspl{GC} are plotted in \figref{ai:benchmark:csharp:gc}. These results show that the two \glspl{GC} perform more or less equivalently. It might even seem that the incremental \gls{GC} performs worse than the original after the curve stabilises after the 350th frame.

In general, the incremental \gls{GC} has lower variation, except from Mono after frame 650 where the \gls{FPS} varies wildly.

\barChart[12][xtick={0, 100, 200, 300, 400, 500, 600, 700, 800, 900}][FPS][Frame No.]{FPS for each frame in C\# using the two different Unity GC}{ai:benchmark:csharp:gc:bar}{
  \plotUnmarkedData{il2cpp}{\aiBenchmarkCsharpGCData}
  \plotUnmarkedData{Mono}{\aiBenchmarkCsharpGCData}
  \plotUnmarkedData{Incremental il2cpp}{\aiBenchmarkCsharpGCData}
  \plotUnmarkedData{Incremental Mono}{\aiBenchmarkCsharpGCData}
}

\lineChart[xtick={100, 200, 300, 400, 500, 600, 700, 800}][FPS][Frame No.]{FPS for each frame in F\# \gls{FRP} using the two different Unity \glspl{GC}}{ai:benchmark:frp:gc}{
      \plotUnmarkedData{il2cpp}{\aiBenchmarkCsharpGCData}
      \plotUnmarkedData{Mono}{\aiBenchmarkCsharpGCData}
      \plotUnmarkedData{Incremental il2cpp}{\aiBenchmarkCsharpGCData}
      \plotUnmarkedData{Incremental Mono}{\aiBenchmarkCsharpGCData}
}

We also tested the two \glspl{GC} with the \gls{FRP}-system we developed in F\#. The results are plotted in \figref{ai:benchmark:frp:gc}. These results show that the incremental \gls{GC} performs slightly better than the original one. However, it comes at the cost of must higher variation in the framerate, especially in the Mono runtime. The il2cpp incremental curve has large spikes up until around the 200th frame, after which point it stabilises within 5-8 \gls{FPS} variation. The Mono curve continues to variate with nearly 30 \gls{FPS}.

\lineChart[xtick={0, 100, 200, 300, 400, 500, 600, 700, 800}][FPS][Frame No.]{FPS for each frame in F\# \gls{FRP} using the two different Unity \glspl{GC}}{ai:benchmark:frp:gc}{
    \plotUnmarkedData{il2cpp}{\aiBenchmarkFsharpGCData}
    \plotUnmarkedData{Mono}{\aiBenchmarkFsharpGCData}
    \plotUnmarkedData{Incremental il2cpp}{\aiBenchmarkFsharpGCData}
    \plotUnmarkedData{Incremental Mono}{\aiBenchmarkFsharpGCData}
}


\section{Cognitive Dimensions}
In this experiment we first and foremost wish to test how well-suited \fsh is in the context of game development. In the side-by-side cognitive dimensions analysis we compare the participants' solutions in \fsh with those in \csh, as they were given test cases from the same category. This allow us to use a well-established vocabulary to discuss whether or not functional programming is suitable for game development.\tmcc{Rewrite metatext}

% Do we need this table? It is part of the Champagne testing, but not really related to side-by-side analysis
% \begin{table}[H]
% 	\alignCenter{
% 	\begin{tabular}{| l | c |}\hline
% 		\multicolumn{2}{|c|}{\textbf{Cognitive Dimensions}} \rowEnd
% 	 	Abstract Gradient & \mns \rowEnd
% 		Closeness of Mapping & \mns \rowEnd
% 		Consistency & \mns \rowEnd
% 		Diffuseness/Terseness & \mns \rowEnd
% 		Error-proneness & \mns \rowEnd
% 		Hard Mental Operations & \mns \rowEnd
% 		Hidden Dependencies & \mns \rowEnd
% 		Premature Commitment & \mn \rowEnd
% 		Progressive Evaluation & \mn \rowEnd
% 		Role-expressiveness & \mn \rowEnd
% 		Secondary Notation and Escape from Formalism & \mn \rowEnd
% 		Viscosity & \rowEnd
% 		Visibility and Juxtaposability & \rowEnd
% 	\end{tabular}}
% 	\caption{Cognitive Dimensions Findings}
% 	\label{tab:cog-dim-findings}
% \end{table}

\subsubsection{Abstract Gradient}
The abstract gradient is measured from abstraction hating, through abstraction tolerant, to abstraction loving. The abstractions measured are the notations' ability to group elements and refer to them as a single entity. Most modern textual-programming languages make extensive use of abstraction and functional languages even more so\cite{hudak1989conception}. \fs is a functional-programming language with object-oriented features allowing for extensive abstractions.

On the other hand \cs is an object-oriented language which supports functional features. This means that \cs also supports extensive abstraction. The main difference lies in the fact that \cs is object-oriented programming first and \fs is functional programming first. Functional programming tends to make use of abstraction more frequently, however that does not mean that other paradigms do not use abstraction\needcite.

Considering the high level of abstraction in both languages they will both be considered abstraction loving in this report. An important detail is that while \fs generally relies on finer grain abstraction and \cs on broader abstraction, they each support the other's abstraction level\tmc{Kan vi finde noget om data abstraction vs. functional abstraction som Bent snakkede om?}.

\subsubsection{Closeness of Mapping}
The measure of how close to the problem domain a language can get is called the closeness of mapping. In order to solve a real problem, that problem must be expressible in the language and the closer the language the easier it is to express\cite{green1996usability}. Textual programming languages are abstractions over the real problem domain and therefore often does not map directly to the domain.

Both languages have mechanisms to model the problem domain. In object-oriented programming, the world is represented as objects and the objects are abstracted over via classes\cite{kindler2011object}. Functional programming models the problem as behaviour (functions) which are applied to data\cite{hughes1989functional}. The advantage of the object oriented approach is that the object abstraction comes quite close the problem domain.

\quoteWithCite{The object expresses the user's view of reality [...]}{Object Oriented Analysis \& Design}{mathiassen2000object}

This approach is in contrast to the functional paradigme which models reality mathematically. This approach is not as close as the object model, however mathematical modelling of the world is widespread in many different fields of study.

\quoteWithCite{Typically the main function is defined in terms of other functions, which in turn are defined in terms of still more functions, until at the bottom level the functions are language primitives. These functions are much like ordinary mathematical functions [...]}{John Hughes}{hughes1989functional}

The abstraction models of the languages are the tools used by the programmers to model the world. \cs uses a model, which lends itself more to closeness of mapping, but both languages make use of custom types and naming which allow programmers to mold their programs in accordance with the problem domain. Additionally both languages support each others modelling approach. An example of this can be seen in \lstref{tree-imps}, where a recursive class in \cs can be implemented via custom types in \fs.

\begin{listing}[H]
  \begin{minted}{fsharp}
type Talent(strength, intellect, agility) =
  member val Strength = strength with get, set
  member val Agility = agility with get, set
  member val Intellect = intellect with get, set

type Tree =
| Node of TalentValue:Talent * Picked:bool * Children:Tree list
| Leaf of TalentValue:Talent * Picked:bool
  \end{minted}
  \begin{minted}{csharp}
public class MyTalent
{
  public int Strength;
  public int Agility;
  public int Intelligence;

  public bool Picked;

  public List<MyTalent> SubTalents = new List<MyTalent>();
}
  \end{minted}
  \caption{Talent Tree Implementations}
  \label{lst:tree-imps}
\end{listing}

The example in \lstref{tree-imps} implements a class in \fs and a discriminated union which together implement the behaviour of the \cs class. The \fs approach has separated the tree from the talent, where in the  \cs solution the tree emerges from the recursive nature of the type. The \cs solution is closer to the problem domain, but the \fs solution is closer to the concept. \tmcc{Det er måske ikke så klart.}\tmc{Nej det er det ikke. Jeg går ud fra at vi taler om data abstraction vs. functional abstraction?}

\subsubsection{Consistency}
% par 2 43:0 32:10, pipe operator
In the cognitive dimensions framework consistency is the coherence between the language designer's understanding and the language user's intuition of the language\cite{green1996usability}. This does not mean that consistency is the difference in language knowledge, but rather the difficulty of extrapolating behaviour and syntax of language features based on knowledge of a subset of the language or other language features.

\fs uses a strict type system which infers types. This feature allows the programmer to omit explicit typing while still gaining the benefits of it. In some cases the type inference can cause confusion or in an unexpected way, as when a \ttt{int16} value is used in the declaration of a \ttt{int} value. In \lstref{type-incompat} an example of this can be seen. An error is thrown because on line 2 an \ttt{int} literal and a \ttt{int16} variable are being multiplied. This behaviour is consistent with \fs's rules, but is surprising for programmers who are versed in C-style languages.

\begin{listing}[H]
\begin{minted}{fsharp}
let x = 10s
let y = 2 * x
\end{minted}
\caption{Type Incompatibility}
\label{lst:type-incompat}
\end{listing}

Naming conventions can present consistency difficulties for some languages. An example of this are the type modules, such as \ttt{List}. These modules supply helper functions for working with a particular type. These functions would be static methods on type class in Java or \cs, which may cause some confusion for C-family programmers\tmc{?}.

In addition to naming conventions causing confusion, \ttt{list}s have another problem. \fs and \cs share a platform and can therefore use each other's language features. While this is an advantage it also presents some disadvantages, most notably that \fs lists and \cs lists are not the same type. This clashes with programmer expectations and converting to the correct list type can be unexpectedly difficult, see \lstref{list-conv}.

\begin{listing}[H]
\begin{minted}{csharp}
private static List<object> GetParams(Microsoft.FSharp.Collections.List<object> parameters)
{
    return new List<LiteralType>(parameters);
}
\end{minted}
\caption{Conversion from \fs List to \cs List}
\label{lst:list-conv}
\end{listing}

In functional programming languages function signatures can often be specified by the programmer, to help the compiler catch unexpected behaviour. This is also possible in \fs, however in a unexpected manner. Function signatures, in \fs, are reported using the Hindly-Milner type system's syntax\cite{fsharp:type:inference}. However, when the programmer attempts to declare the function signature manually, they cannot use the same syntax. Instead a Python-like syntax is used. An example can be seen in \lstref{fun-sig}.

\begin{listing}[H]
\begin{minted}{fsharp}
// reported function signature
val add: x:int -> y:int -> int

// function definition and signature
let add x y = x + y
// or alternatively with explicit types
let add (x:int) (y:int) : int = x + y
\end{minted}
\caption{Function Signatures}
\label{lst:fun-sig}
\end{listing}

In \fs lambda expressions are denoted using the \ttt{fun} keyword and \ttt{-\textgreater} operator. The use of \ttt{fun} vs. the use of \ttt{func} may initially be confusing for programmers, but is quickly learned. After the initial confusion, the feature is consistent with the rest of \fs, as lambda expressions are defined like functions. This is not the case in \cs, where lambda expressions are defined using the \ttt{=\textgreater} operator. This is not consistent with the rest of \cs, because \cs does not use a function signature similar to the Hindly-Milner type system.

\fs has two primary collection types: lists and arrays. The array-collection type can provides some benefits when looking up elements by index.  However, when looking up an element by index dot notation is used to call the \ttt{[]} function, thus a lookup looks becomes \fsinline{array1.[0]}. This clashes with expectations of a C-family programmer where \csinline{array1[0]} is the norm.

While \fs presents some consistency problems, they're are consistent within the language. This indicates that the problems may be experienced more by novice programmers and that they dissipate with experience.

\subsubsection{Diffuseness/Terseness}
%Antal af paranteser: Participant 6, 1:01:00 cirka
\subsubsection{Error-proneness}
% Participant 2: 32:10, pipe operator
%Antal af paranteser: Participant 6, 1:01:00 cirka
% Participant 6: Copy-pasting input parameters into expressions 1:03:30-ish, participant changes order of arguments
\subsubsection{Hard Mental Operations}
The hard mental operations dimension defines how often incomprehensible expressions occur in the code. Hard mental operations often occur in conjunction with boolean expressions\cite{green1996usability}.

Boolean expressions are expressed similarly in C\# and F\#, with the only exception that C\# uses \ttt{!} to negate expressions, whereas F\# uses the \ttt{not} function. We argue that the use of the \ttt{not}-function reduces the perceived ambiguety, because it forces the programmer to parenthesise the expression. We have illustrated an example in \lstref{hard:mental:operations}. In this example the reader may incorrectly assume that the \ttt{\&\&} operator is evaluated before \ttt{!}, and that the expression is evaluated as  \csinline{!(expr1 && expr2)}. That is not the case in F\#, where the parantheses indicate the order of evaluation.

\begin{listing}[H]
    \begin{minipage}{.45\textwidth}
        \begin{minted}{fsharp}
if(!expr1 && expr2) {
    //[...]
}
        \end{minted}
    \end{minipage}
    \hfill
    \begin{minipage}{.45\textwidth}
        \begin{minted}{csharp}
if (not expr1) && expr2 then
    //[...]
        \end{minted}
    \end{minipage}
\caption{Hard mental operations illustrated using boolean expressions in C\# and F\#.}
\label{lst:hard:mental:operations}
\end{listing}

The F\# compiler uses type inference with the Damas-Hindley W algorithm\cite{fsharp:type:inference}, meaning that it is optional for the programmer to indicate types when he is writing functions. Sometimes it may be necessary to verify that the compilers inferance is correct by looking at the deduced function signature. We argue that this is a hard mental operation in F\#. The function signatures will quickly get incomprehensible as the number of arguments grow. Take for example the function signature of the \ttt{ReactTo} function that we wrote as part of the \gls{FRP} plugin for Unity: \fsinline{member FRPBehaviour.ReactTo : event:FRPEvent * condition:('T0 -> bool) * handler:('T0 -> unit) -> unit}. We should underline that this function uses tupled arguments, because we wanted to overload it with a function to unconditionally react to events. But even without tupled arguments the definition would have been: \fsinline{member FRPBehaviour.ReactTo : event:FRPEvent -> condition:('T0 -> bool) -> handler:('T0 -> unit) -> unit}. In C\# the equivalent would have been \csinline{public void ReactTo(FRPEvent event, Func<T, bool> condition, Action<T> handler)}, which is also lengthy, but in our opinion easier to comprehend.
\subsubsection{Hidden Dependencies}
Hidden dependencies discuss how many relationships there are between components, that are not visible from atleast one of the components. We discuss several problems in this section, but we must underline that some of these problems can be mitigated by using an \gls{IDE}, as they often allow programmers to jump to implementations.

The problem of hidden dependencies is prominent in both C\# and F\#, though in two different flavours. We have listed code in \lstref{hidden:dependencies} that showcase the problems. In F\# the problem is present on the function level. This is because F\# programmers are allowed to write functions that are directly nested in a module. These functions may be imported into another module or namespace with the \ttt{open} keyword. Given that a name of a function does not collide, the function may be refered to without the use of its fully qualified name. In C\# the problem occurs because programmers are allowed to reference code in base classes without using their fully qualified name. One example of this in Unity is that any \ttt{MonoBehaviour} may directly call \ttt{Destroy}, which is actually a static method on the \ttt{GameObject}-class. This may wash out the distinction between static and non-static methods. In F\# programmers are required to give fully qualified names when dealing with classes.

\begin{listing}[H]
    \begin{minipage}{.50\textwidth}
        \begin{minted}{csharp}
class Base {
    protected static string Method() {
        return "Method";
    }
}

class Inherited : Base {
    private string GetString() {
        return Method() + " is the string";
    }
}
        \end{minted}
    \end{minipage}
    \hfill
    \begin{minipage}{.40\textwidth}
        \begin{minted}{fsharp}
//Imagine this is in another file
module X =
    let function () =
        "function"

open X
module Y =
    let function2 () =
        function() + "2"
        \end{minted}
    \end{minipage}
\caption{Hidden dependencies in function/method calls in C\# and F\#.}
\label{lst:hidden:dependencies}
\end{listing}

In both languages the problem of shadowing may occur. An example of this is if a base class defines a \ttt{virtual} method in C\# or an \ttt{abstract} method in F\#. This method method may be shadowed further down the inheritance tree by implementing a function with the same name without using the \ttt{override} keyword. This will give a warning on compile-time, which can be removed by supplying the \ttt{new} keyword in front of the shadowing method.

% goto statements in C#
C\# has a modified version of the \ttt{goto}-statement, which traditionally allow programmers to jump to labels anywhere in the source code. In C\#, however, the jumps are restricted to either a label in the same scope or one in an enclosing scope\cite{csharp:goto}. Nevertheless, it may be hard to comprehend the exact target of a \ttt{goto} if it is deeply nested in multiple loops. According to a StackOverflow discussion\cite{goto:stack:overflow}, it seems that the \ttt{goto} statement sees limited use in practise.

% mutability
% Unity Editor reference assignment
The problem of hidden dependencies may also occur if a component is dependant on a global variable. In games it is common to see such types of dependency\cite{blow2004game, guana2015building, nystrom2014game}. This problem is easier to mitigate in F\# because the global variable is per default immutable and the developer has to explicitly indicate if he wants a mutable variable. In C\# it's easier to make slips, as everything is mutable per default. In Unity this problem is also present inbetween components, as it's a common pattern to declare a field on a class that references some component and thereby assign that field from Unity's Inspector\cite{unity:inspector:assignment}. This has the consequence that there is no way of knowing which components depend on each other by inspecting the source code.
\subsubsection{Premature Commitment}
Premature commitment describes how much guess-ahead the programmers has to make when he is programming in a given language. 

In F\# there is a fixed ordering when defining types, that enforces all namebindings (\ttt{let}) to be declared before members. This is similar to the problem of \textit{Commitment to layout} presented in \cite{green1996usability}. Luckily, these declarations may quickly be moved arround in F\# source code by cutting and pasting. This problem is not present in C\#, where the programmer is free to choose any ordering he likes when declaring methods, fields and properties on classes.

In C\# the problem of premature commitment may surface when dealing with class hierarchies. The problem is also present in F\#, as it is also object-oriented, but we argue that C\# is more prone to the class-related problems \tmc{Skal vi have en argumentation her}. The problem of premature commitment arises when a programmer has to implement a base class, without being certain which other classes might inherit therefrom. This introduce guess-work and might result in missing functionality, uneeded class members and potentially the requirement to re-implement the base class. This problem is even more prominent when inheriting from third party code that contains multiple classes, as the programmer might choose to inherit from one class and later discover that he took the wrong and need to re-implement the entire class. The problem is more prominent in Unreal than Unity, as Unreal has multiple different base classes for game objects\needcite, where Unity has one.

Another small-scale issue of premature commitment arises when using F\#'s collection functions (such as \ttt{List.map} or \ttt{Array.reduce}). These functions take as first argument a function and as second the collection to operate on. If they are used without the pipe operator (\ttt{|\textgreater}), the \gls{IDE} will be unable to aid the programmer when he is implementing the function until the second argument has been given. The correct order is thus to write the name of the function, add empty brackets, add the name of the collection and finally implement the function.
\subsubsection{Progressive Evaluation}

\subsubsection{Role-expressiveness}
Role-expressiveness defines how easily a program can be read and comprehended. The dimension is easily confused with hard mental operations or secondary notation, from which it should be kept apart\cite{green1996usability}. Role-expressiveness thus defines how self-explanatory a program is.

Being languages that run in the same platform, C\# and F\# share many constructs and all libraries. This means that a discussion of the standard library is of little interest. There are, however, some differences in the syntax that we will highlight. 

%let vs. var
First and foremost C\# uses either \ttt{var} or type names in variable declarations. In F\# the equivalent is \ttt{let}, possibly followed by \ttt{mutable} to indicate mutability. Depending on the programmer's background, these keywords may be more or less expressive. From a mathematical background it makes sense to create namebindings by using \ttt{let}, as that's common in proofs and similar mathematical lingo. The type of a namebinding may be inferred by how it's used. Furthermore, \ttt{let} indicates a namebinding and not a variable, which further underlines that F\# is pure until otherwise is expressed. This contrasts with the classic C-style way of defining variables; by using their type name. Some programmers that are less versed in mathematical notation may prefer this way, as it is more expressive of the variable's type. Finally, the \ttt{var} keyword is simply an abbreviation for \squote{variable}, which goes well hand-in-hand with its purpose; a variable which we do not care about the type of. Common for the last two types of declarations are that they do not indicate anything about mutability and thus require knowledge about the language at hand.

%F# fun-keyword
Lambda functions are available in both languages. In C\# they're expressed as \mintinline{csharp}|(a,b) => a + b|, where as in F\# they're expressed as \mintinline{fsharp}|fun a b -> a + b|. In this case F\# uses the keyword \ttt{fun} to indicate a lambda, which is quite literally an abbreviation of \squote{function}. One thing worth noting is that the \ttt{fun} keyword may easily be confused with the English word fun, another abbreviation such as \ttt{fn} or \ttt{func}, would probably have been better. C\# uses the \ttt{=\textgreater}, which is of limited expressiveness, especially because C\# does not use arrow-style function signatures anywhere else.

%Type vs. class
In order to declare custom data structures in C\# one can use \ttt{class}, \ttt{struct} or \ttt{enum}, depending purpose of the structure. In F\# all data structures are constructed using the \ttt{type} keyword. Depending on the symbols used in and around the definition, the outcome will change. We have illustrated this in \lstref{fsharp:type}. As a result this means that the \ttt{type} keyword in F\# have very limited role-expressiveness, compared to those of C\#.

\begin{listing}
    \begin{minted}{fsharp}
type Enum =
| B = 0
| C = 1

type Union =
| B
| C

type DiscriminatingUnion =
| B of bool
| C of char

type Record = { b: bool, c: char }

type Class() =
    let b = true
    let c = 'x'

[<Struct>]
type Struct(b:bool, c:char) =
    member this.B = b
    member this.C = c
    \end{minted}
    \caption{Different kinds of data structures defined using the \ttt{type}-keyword in F\#.}
    \label{lst:fsharp:type}
\end{listing}
\subsubsection{Secondary Notation and Escape from Formalism}
Secondary notation and escapte from formalism defines how well a programming environment supports conyeing information that is not part of the source code. Typical examples of such are comments, indentation and grouping code into paragraphs in textual languages\cite{green1996usability}.

C\# and F\# are very alike in this dimension. They both support the \ttt{//}-operator, which indicates that the rest of the line should be commented out and matching pairs of \ttt{/*} and \ttt{*/} which comments everything out between them. Furthermore functions, methods, classes and more or less any program construct may be annotated with \ttt{///}-comments, which allow the programmer to add \gls{XML}-documentation\cite{fsharp:xml:doc}. The programmer may use this to describe the intend of the construct along with its arguments and, if needed, link to other constructs in the program. Other developers may open this documentation in a pop-up box elsewhere in the program, whenever such construct is encountered.
\subsubsection{Viscosity}
Viscosity defines how much effort a developer has to put in to make a small change. \cite{green1996usability} notes that textual languages are less viscous than visual programming languages.

C\# and F\# are both textual languages and thus have relatively low viscosity. We argue that the primary difference between C\# and F\# are scope delimitation. C\# scopes are delimited by pairs of curly brackets, whereas in F\# they are delimited by indentation. If a programmer is to move code from one scope to another in C\#, he would have to either insert or delete pairs of curly brackets, whereas in F\# he would select the code that needs to be moved and press TAB or Shift+TAB.

% Participant 2, 40:40 samt 1:02:30
In our test cases, viscosity is particularly visible in the \dquote{concurrent}-update category. The reason for this is that the participants are asked to develop a sequential solution first, followed by a parallel implementation. Generally viscosity is low in both languages. In F\# we saw the magnetism task implemented using the pipe operator. Such an implementation can be extended to a parallel solution by piping into \ttt{Async.Parallel} and then \ttt{Async.RunSynchronously} (see \lstref{fsharp:pipe:async}). A similar solution can be achieved in C\# using \gls{LINQ}, albeit the change requires the programmer to delete a semicolon.

\begin{listing}
    \begin{minted}{fsharp}
let speed = 3f;

let moveBallForward (ball:GameObject) =
    ball.transform.Translate(ball.transform.forward * Time.deltaTime * speed)

let Update () =
    let balls = GameObject.FindGameObjectsWithTag("Magnetic")
    balls
    |> Array.map moveBallForward
    ()

let UpdateAsync () =
    let balls = GameObject.FindGameObjectsWithTag("Magnetic")
    balls
    |> Array.map (fun b -> async {moveBallForward})
    |> Async.Parallel
    |> Async.RunSynchronously
    ()
    \end{minted}
    \caption{Transforming from sequential to concurrent list operations in F\#.}
    \label{lst:fsharp:pipe:async}
\end{listing}

As with many other dimensions, viscosity can also be affected by the programmer's style of programming. This is explified in \lstref{csharp:viscous}, which is taken from one of the solutions in C\#. In order to make implement \dquote{concurrent} update, the participant had to construct a new list in the \ttt{Update}-method and wrap the calls to the state methods in \ttt{Task.Run} (e.g. \mintinline{csharp}|updateTasks.Add(Task.Run(() => Flee(fleeingShooter)))|). This change is manageable, but imagine how much effort it would take if we wanted to add an additional state to the shooter units. If such change was to be implemented, the programmer would have to:
\begin{itemize}
    \item Add an additional list and \ttt{foreach}-statement in \lstref{csharp:viscous}.
    \item Change the signature of the \ttt{TransferState}-method in \lstref{csharp:viscous:transfer} to accept an additional list and add an addtitional \ttt{else if} for the third list.
    \item Add an additional case to the switch in \lstref{csharp:viscous:transfer}, and update the call to \ttt{RemoveFromList} in all other cases.
\end{itemize}


\begin{listing}
    \begin{minted}{csharp}
class StateMachine : MonoBehaviour
{
    [...] //Pre-implemented code, such as JoinState

    private List<Shooter> fleeingShooters;
    private List<Shooter> movingShooters;
    private List<Shooter> attackingShooters;

    private void Update()
    {
        foreach(var fleeingShooter in fleeingShooters)
        {
            Flee(fleeingShooter);
        }
        foreach (var movingShooter in movingShooters)
        {
            Move(movingShooter);
        }
        foreach (var attackingShooter in attackingShooters)
        {
            Attack(attackingShooter);
        }
    }

    [...] //TransferState
    [...] //RemoveFromList

    [...] //methods for each unit state
}
    \end{minted}
    \caption{Example of viscous C\# implementation of the Unit Management Test.}
    \label{lst:csharp:viscous}
\end{listing}

\begin{listing}
    \begin{minted}{csharp}
public void TransferState(Shooter shooter, State state)
{
    switch (state)
    {
        case State.Fleeing:
            fleeingShooters.Add(shooter);
            RemoveFromList(shooter, ref movingShooters, ref attackingShooters);
            break;
        case State.Moving:
            movingShooters.Add(shooter);
            RemoveFromList(shooter, ref fleeingShooters, ref attackingShooters);
            break;
        case State.Attacking:
            attackingShooters.Add(shooter);
            RemoveFromList(shooter, ref movingShooters, ref fleeingShooters);
            break;
        default:
            break;
    }
}
    \end{minted}
    \caption{TransferState-method, which is part of the viscous Unit Management implementation from \lstref{csharp:viscous}.}
    \label{lst:csharp:viscous:transfer}
\end{listing}

\begin{listing}
    \begin{minted}{csharp}
private void RemoveFromList(Shooter shooter, ref List<Shooter> list1, ref List<Shooter> list2)
{
    if (list1.IndexOf(shooter) != -1) {
        list1.Remove(shooter);
        return;
    }
    if(list2.IndexOf(shooter) != -1) {
        list2.Remove(shooter);
        return;
    }
}
    \end{minted}
    \caption{RemoveFromList-method, which is part of the viscous Unit Management implementation from \lstref{csharp:viscous}.}
    \label{lst:csharp:viscous:remove}
\end{listing}
\subsubsection{Visibility and Juxtaposability}
Visibility and Juxtaposability determines whether required material is accessible without cognitive work\cite{green1996usability}. In textual languages this dimension is not necessarily determined by the language, but more so by the environment (\gls{IDE}).

During the sessions all the participants chose to use Visual Studio. Visual Studio supports numerous ways of making source code available. Examples of such are:
\begin{itemize}
    \item Splitting the text editor both horizontally and vertically, such that two files can be open side-by-side
    \item Jumping to implementations by control-clicking. 
    \item Hovering function or type names to read a description of what they're meant for. 
\end{itemize}
As Visual Studio was used in both C\# and F\#, the differences are minimal. It is interesting, however, that none of the Visibility and Juxtaposability enchancing features were used during the tests. This may have two possible explanations:
\begin{enumerate}
    \item The tests were relatively small-scale and required minimal interaction with existing code.
    \item When existing code was needed, the participants would refer to the cheat-sheet or task descriptions, where example code was given. Some participants actually chose to open those documents side-by-side with Visual Studio.
\end{enumerate}


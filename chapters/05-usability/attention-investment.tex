\subsubsection{Attention Investment Model}
The attention investment model, presented in \secref{attention-investment}, can be used to map out the programming steps undertaken by a user. This section will endeaver to do so, using the user test video material. Firstly, user comprehension of the feature being evaluated is measured. This feature is the use of \fsh and \gls{FRP} in game development. In order to measure comprehension, the instances where participants mention aspects of the feature where recorded and categorised, as can be seen in \tableref{comp-matrix}.

\begin{table}[H]
	\alignCenter{
	\begin{tabular}{| l | c | c | c |}\hline
		\multicolumn{1}{| l |}{\textbf{Recognised Aspects}}&\multicolumn{1}{| l |}{\textbf{Correct \& Unprompted}}&\multicolumn{1}{| l |}{\textbf{Correct \& Prompted}}&\multicolumn{1}{| l |}{\textbf{Incorrect}} \rowEnd
		Modularity 			& \mn\mn & & \mn \rowEnd
		ReactTo 				& \mns & \mn\mn\mns & \mn\mn\mn \rowEnd
		Types 					& \mn\mn & & \mn\mns\mns\mns\mns\mns \rowEnd
		List Operations	& \mn & \mn\mn\mns & \mn\mns \rowEnd
	\end{tabular}}
	\caption{User Comprehension of the Feature}
	\label{tab:comp-matrix}
\end{table}

In addition to a measure of the participants comprehension, the attention investment model also provides a quantification of their efforts in the programming activity. This consists of four metrics, mentioned in \secref{attention-investment}. This can be seen in \tableref{att-inv-findings}. The risk metric measures the amount of times participants mentioned or discussed things that could go wrong, which includes increased difficulty of some tasks using \fsh. The cost metric is the attention and time required to switch to \fsh. Any musing over the difficulties of switching over is included. The payoff is the reduced cost of game development after switching to \fsh. The imperative alternative metric is the number of times participants mentioned the problems in \csh, or other imperative languages, that form the basis for switching to \fsh.

\begin{table}[H]
	\alignCenter{
	\begin{tabular}{| l | c |}\hline
		\multicolumn{2}{|c|}{\textbf{Attention Investment}} \rowEnd
		Investment Risk & \mn\mn\mn\mn\mns  \rowEnd
		Investment Cost & \mn\mns\mns\mns\mns \rowEnd
		Investment Payoff & \mn\mns\mns\mns\mns \rowEnd
		Imperative Alternative & \mn\mn\mn \rowEnd
	\end{tabular}}
	\caption{Attention Investment Findings}
	\label{tab:att-inv-findings}
\end{table}

The participants were able to correctly use and describe \fsh and \gls{FRP} behaviour in some instances and struggled in other instances. As can be seen in \tableref{comp-matrix}, not all participants where cognisant of all feature aspects, e.g. all participants misunderstood or struggled with the type system. Functional programming claims a greater degree of modularity than imperative languages\cite{hughes1989functional}, however, less than half of the participants expressed cognisance of this. Some participants wrote much more modular code in \fsh than in \csh, but did not mention it.

As can be seen in \tableref{att-inv-findings} the participants noted a high cost with a high payoff. Participants could see the usefulness of \gls{FRP}, but several participants expressed uncertainty of any benefit provided by \fsh. Another point is that very few participants noted the problems with existing solutions, even when compared to \fsh. The high cost is attributed to loss of productivity while a developer learns to use \fsh.

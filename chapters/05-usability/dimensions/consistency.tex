\subsubsection{Consistency} % par 2 43:0
In the cognitive dimensions framework consistency is the coherence between the language designer's understanding and the language user's intuition of the language\cite{green1996usability}. This does not mean that consistency is the difference in language knowledge, but rather the difficulty of extrapolating behaviour and syntax of language features based on knowledge of other language features.

All participants had prior \cs experience which may have affected their expectations. This was apparent when participants applied \cs methodology in the \fs code. An example of this is the confusion of types some participants experienced. Several participants noted that they preferred strict typing or specifying types manually. 

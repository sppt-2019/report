\subsubsection{Viscosity}
Viscosity defines how much effort a developer has to put in to make a small change. \cite{green1996usability} notes that textual languages are less viscous than visual programming languages.

C\# and F\# are both textual languages and thus have relatively low viscosity. We argue that the primary difference between C\# and F\# are scope delimitation. C\# scopes are delimited by pairs of curly brackets, whereas in F\# they are delimited by indentation. If a programmer is to move code from one scope to another in C\#, he would have to either insert or delete pairs of curly brackets, whereas in F\# he would select the code that needs to be moved and press TAB or Shift+TAB.

% Participant 2, 40:40 samt 1:02:30
In our test cases, viscosity is particularly visible in the \dquote{concurrent}-update category. The reason for this is that the participants are asked to develop a sequential solution first, followed by a parallel implementation. Generally viscosity is low. Given an implementation in F\# using the pipe operator, one can implement a parallel solution by piping into \ttt{Async.Parallel} and then \ttt{Async.RunSynchronously} (see \lstref{fsharp:pipe:async}).

\begin{listing}
    \begin{minted}{fsharp}
let speed = 3f;

let moveBallForward (ball:GameObject) =
    ball.transform.Translate(ball.transform.forward * Time.deltaTime * speed)

let Update () =
    let balls = GameObject.FindGameObjectsWithTag("Magnetic")
    balls
    |> Array.map moveBallForward
    ()

let UpdateAsync () =
    let balls = GameObject.FindGameObjectsWithTag("Magnetic")
    balls
    |> Array.map (fun b -> async {moveBallForward})
    |> Async.Parallel
    |> Async.RunSynchronously
    ()
    \end{minted}
    \caption{Transforming from sequential to concurrent list operations in F\#.}
    \label{lst:fsharp:pipe:async}
\end{listing}
\section{Results Analysis} \label{sec:test-results}
The result of the tests was six git branches with source code written by the participants along with five video-files. Sadly the video-recording program broke down during one of the tests and we were unable to recover the file. We took notes during the tests and will refer to those instead. Whenever we quote the notes rather than the participant, we will clearly indicate that.

In general the participants were able to complete roughly one test case, some didn't and some started the second as well. This was the case for both \fsh and \csh.

\subsection{Questionnaire}
All participants were asked to estimate their own skill levels in these categories and give a ballpark estimate of how many Unity applications they had developed. The results can be seen in \tableref{participant-scores}.

\begin{table}[H]
\begin{tabular}{| c | r | r | r | r | r |}
	\hline
	\textbf{Participant}&\textbf{\csh}&\textbf{Unity}&\textbf{Game Dev}&\textbf{Functional}&\textbf{Unity Apps} \\ \hline
	1 & 9 & 9 & 5 & 3 & 25 \\ \hline
	2 & 8 & 8 & 7 & 2 & 10 \\ \hline
	3 & 8 & 8 & 2 & 1 & 12 \\ \hline
	4 & 10 & 10 & 8 & 1 & 10 \\ \hline
	5 & 9 & 9 & 9 & 2 & 10 \\ \hline
	6 & 8 & 8 & 9 & 6 & 5 \\ \hline
\end{tabular}
\caption{Participants Self Evaluations}
\label{tab:participant-scores}
\end{table}

\newcommand{\mn}{\newmoon}
\newcommand{\mns}{\fullmoon}

\subsection{Data Processing}
In the following section we will process the data from the tests, using methodology presented in \secref{prog-usability}. The tests here constructed using the Champagne Prototyping methodology and the analysis of the data will also follow this approach. Champagne Prototyping uses two methodologies for data processing; Attention Investment Model and Cognitive Dimensions.

\subsubsection{Attention Investment Model}
The attention investment model, presented in \secref{attention-investment}, can be used to map out the programming steps undertaken by a user. This section will endeaver to do so, using the user test video material. Firstly, user comprehension of the feature being evaluated is measured. This feature is the use of \fsh and \gls{FRP} in game development. In order to measure comprehension, the instances where participants mention aspects of the feature where recorded and categorised, as can be seen in \tableref{comp-matrix}.

\begin{table}[H]
	\alignCenter{
	\begin{tabular}{| l | c | c | c |}\hline
		\multicolumn{1}{| l |}{\textbf{Recognised Aspects}}&\multicolumn{1}{| l |}{\textbf{Correct \& Unprompted}}&\multicolumn{1}{| l |}{\textbf{Correct \& Prompted}}&\multicolumn{1}{| l |}{\textbf{Incorrect}} \rowEnd
		Modularity 			& \mn\mn & & \mn \rowEnd
		ReactTo 				& \mns & \mn\mn\mns & \mn\mn\mn \rowEnd
		Types 					& \mn\mn & & \mn\mns\mns\mns\mns\mns \rowEnd
		List Operations	& \mn & \mn\mn\mns & \mn\mns \rowEnd
	\end{tabular}}
	\caption{User Comprehension of the Feature}
	\label{tab:comp-matrix}
\end{table}

A closed bullet (\mn) means that a participant mentioned an aspect of the feature, when the participant mentioned it multiple times a open bullet is used (\mns). The categories, \textit{Correct \& Unprompted} and \textit{Correct \& Prompted} are instances where the participants mention or explain a feature correctly and/or positively. The last category are instances where features are mentioned negatively, incorrectly or is a cause for confusion.

In addition to a measure of the participants comprehension, the attention investment model also provides a quantification of their efforts in the programming activity. This consists of four metrics, mentioned in \secref{attention-investment}. This can be seen in \tableref{att-inv-findings}. The risk metric measures the amount of times participants mentioned or discussed things that could go wrong, which includes increased difficulty of some tasks using \fsh. The cost metric is the attention and time required to switch to \fsh. Any musing over the difficulties of switching over is included. The payoff is the reduced cost of game development after switching to \fsh. The imperative alternative metric is the number of times participants mentioned the problems in \csh, or other imperative languages, that form the basis for switching to \fsh.

\begin{table}[H]
	\alignCenter{
	\begin{tabular}{| l | c |}\hline
		\multicolumn{2}{|c|}{\textbf{Attention Investment}} \rowEnd
		Investment Risk & \mn\mn\mn\mn\mns  \rowEnd
		Investment Cost & \mn\mns\mns\mns\mns \rowEnd
		Investment Payoff & \mn\mns\mns\mns\mns \rowEnd
		Imperative Alternative & \mn\mn\mn \rowEnd
	\end{tabular}}
	\caption{Attention Investment Findings}
	\label{tab:att-inv-findings}
\end{table}

The participants were able to correctly use and describe \fsh and \gls{FRP} behaviour in some instances and struggled in other instances. As can be seen in \tableref{comp-matrix}, not all participants where cognisant of all feature aspects, e.g. all participants misunderstood or struggled with the type system. Functional programming claims a greater degree of modularity than imperative languages\cite{hughes1989functional}, however, less than half of the participants expressed cognisance of this. Some participants wrote much more modular code in \fsh than in \csh, but did not mention it.

As can be seen in \tableref{att-inv-findings} the participants noted a high cost with a high payoff. Participants could see the usefulness of \gls{FRP}, but several participants expressed uncertainty of any benefit provided by \fsh. Another point is that very few participants noted the problems with existing solutions, even when compared to \fsh. The high cost is attributed to loss of productivity while a developer learns to use \fsh.

\subsubsection{Cognitive Dimensions}
In this experiment we first and foremost wish to test how well-suited \fsh is in the context of game development. In the side-by-side cognitive dimensions analysis we compare the participants' solutions in \fsh with those in \csh, as they were given test cases from the same category. This allow us to use a well-established vocabulary to discuss whether or not functional programming is suitable for game development.

% Do we need this table? It is part of the Champagne testing, but not really related to side-by-side analysis
% \begin{table}[H]
% 	\alignCenter{
% 	\begin{tabular}{| l | c |}\hline
% 		\multicolumn{2}{|c|}{\textbf{Cognitive Dimensions}} \rowEnd
% 	 	Abstract Gradient & \mns \rowEnd
% 		Closeness of Mapping & \mns \rowEnd
% 		Consistency & \mns \rowEnd
% 		Diffuseness/Terseness & \mns \rowEnd
% 		Error-proneness & \mns \rowEnd
% 		Hard Mental Operations & \mns \rowEnd
% 		Hidden Dependencies & \mns \rowEnd
% 		Premature Commitment & \mn \rowEnd
% 		Progressive Evaluation & \mn \rowEnd
% 		Role-expressiveness & \mn \rowEnd
% 		Secondary Notation and Escape from Formalism & \mn \rowEnd
% 		Viscosity & \rowEnd
% 		Visibility and Juxtaposability & \rowEnd
% 	\end{tabular}}
% 	\caption{Cognitive Dimensions Findings}
% 	\label{tab:cog-dim-findings}
% \end{table}

\subsubsection{Abstract Gradient}
The abstract gradient is measured from abstraction hating, through abstraction tolerant, to abstraction loving. The abstractions measured are the notations ability to group elements and refer to them as a single entity. Most modern textual-programming languages make extensive use of abstraction and functional languages even more so \cite{hudak1989conception}. \fs is a functional-programming language with object-oriented features allowing for extensive abstractions.

On the other hand \cs is an object-oriented language which supports functional features. This means that \cs also supports extensive abstraction. The main difference lies in the fact that \cs is object-oriented programming first and \fs is functional programming first. Functional programming tends to make use of abstraction more frequently then conventional programming, however that does not mean conventional programming does not use abstraction often.

Considering the high level of abstraction in both languages they will both be considered abstraction loving in this report. An important detail is that \fs generally relies on finer grain abstraction and \cs on broader abstraction, they each support the others abstraction level.

\subsubsection{Closeness of Mapping}


\subsubsection{Consistency}


\subsubsection{Diffuseness/Terseness}


\subsubsection{Error-proneness}


\subsubsection{Hard Mental Operations}


\subsubsection{Hidden Dependencies}


\subsubsection{Premature Commitment}


\subsubsection{Progressive Evaluation}


\subsubsection{Role-expressiveness}


\subsubsection{Secondary Notation and Escape from Formalism}


\subsubsection{Viscosity}


\subsubsection{Visibility and Juxtaposability}

\subsection{Syntax Comparison}
Test participants encountered problems with various aspects of F\#'s syntax. Such issues are to be expected when developing in a new programming language, furthermore, syntactical errors are not the main focus of this inquiry. Therefore these issues are discussed in this section.

\subsubsection{Bindings and Operators}
In F\# the \ttt{=} symbol is used to denote two different operators; the value-binding operator, and the equality operator. Notably the symbol is not used for rebinding, F\# equivalent of assignments which it is in C\#. A comparison of the two different rebinding/assignment styles can be seen in \lstref{ass-comp}. Rebindings are only allowed on \ttt{mutable} variables, which should be limited to a scope.

\begin{listing}[H]
\begin{minipage}{.45\textwidth}
\begin{minted}{fsharp}
let mutable x = 0
x <- 1
\end{minted}
\end{minipage}
\hfill
\begin{minipage}{.45\textwidth}
\begin{minted}{csharp}
int x = 0;
x = 1;
\end{minted}
\end{minipage}
\caption{Assignment Comparison}
\label{lst:ass-comp}
\end{listing}

Furthermore, since \ttt{x = 1} is valid F\#, no errors where encountered immediately. Instead the program behaved unexpectedly and they received a warning, that the value of a boolean expression was being ignored. None of the participants where able to deduce the source of the error without assistance from the monitor. The degree of assistance required ranged from a hint to the monitor explaining the problem so that the test could continue.

Some participants encountered minor errors when declaring variables, because variables are implicitly immutable in F\#. This is the opposite of C\#, where the \ttt{const} keyword is used to explicitly declare a variable immutable. However, once the participants realised this, they had no issue using it.

\subsubsection{Scopes and Indentation}
Several of the participants did not connect indentation with scope. This meant that the scoping of variables at times presented a challenge. In most test cases the solutions are implemented in a class, therefore when test participants indented their function bodies incorrectly the code would work initially. This was because the function body would be part of the class instead of the function and could therefore still be called within the scope, this can be seen in \lstref{scope-prob}.

\begin{listing}[H]
\begin{minted}{fsharp}
type FRP_FPSController() =
    inherit FRPBehaviour()

    [...]

    member this.HandleMoveForward() =
    let newPosition = this.transform.position + new Vector3(0.0f, 0.0f, _velocity)
    this.transform.position <- newPosition
\end{minted}
\caption{Incorrect Indentation}
\label{lst:scope-prob}
\end{listing}

Lines 7 and 8 are indented incorrectly, but the code compiles and behaves as expected. The \gls{IDE} gives a warning on both lines, but no errors are thrown until new \ttt{member} functions are defined.

\subsubsection{Types and Type Inference}
Many participants had problems with the type system and type inference. The functions provided in the sample sheets had explicitly typed parameters, which the participants partially mimicked when declaring their own functions, however participants did not specify function return types nor did they utilise typing of their variables. An example of such a function can be seen in \lstref{part-func}.

\begin{listing}[H]
\begin{minted}{fsharp}
[...]
let getTotalWithMod (items:Item list) (attribute:Item->int) (attributeMod:Item->float32) =
[...]
\end{minted}
\caption{Participant Function with Type Annotations}
\label{lst:part-func}
\end{listing}

In addition several participants remarked that they preferred strictly typed languages, even though they had encountered several typing errors. Some of the participants had Python experience which may explain this assumption, because Python 3.x annotates function parameters similarly to F\# and Python is dynamically typed.

\subsubsection{Function Definitions}
In F\# classes all functions must be declared before the first method. This caused some initial confusion for participants, but it was quickly overcome. The reason for the strict order was not intuitive for the participants. The difference between the use of \ttt{let} and \ttt{member} is the accessibility of the method or function in question. The \ttt{let} keyword denotes a function in the class instance's scope, effectively a private function. The \ttt{member} keyword denotes a method which is similar to methods in C\#.

In addition to the confusion surrounding type definitions, participants also expressed frustration with how functions are defined. The issue stems from a problem with our sample sheet which did not contain an example of a simple function definition, plenty of functions where defined, but only ever as part of examples of other language features. Furthermore functions share the \ttt{let} keyword with variables, which contributed to the confusion during the test. Some participants also had some problems with lambda expressions. Some users where initially thrown off by the \ttt{fun} keyword, while others where new to the concept.


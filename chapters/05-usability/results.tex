\section{Test Results} \label{sec:test-results}
The result of the tests was six git branches with source code written by the participants along with five video-files. Sadly the video-recording program broke down during one of the tests and we were unable to recover the file. We took notes during the tests and will refer to those instead. Whenever we quote the notes rather than the participant, we will clearly indicate that.


In general the participants were able to complete roughly one test case, some didn't and some started the second as well. This was the case for both \fsh and \csh.

\subsection{Questionnaire}
All participants where asked to estimate their own skill levels in these categories and give a ballpark estimate of how many Unity applications they had developed. The results can be seen in \tableref{participant-scores}.

\begin{table}[H]
\begin{tabular}{| c | r | r | r | r | r |}
	\hline
	\textbf{Participant}&\textbf{\csh}&\textbf{Unity}&\textbf{Game Dev}&\textbf{Functional}&\textbf{Unity Apps} \\ \hline
	1 & 9 & 9 & 5 & 3 & 25 \\ \hline
	2 & 8 & 8 & 7 & 2 & 10 \\ \hline
	3 & 8 & 8 & 2 & 1 & 12 \\ \hline
	4 & 10 & 10 & 8 & 1 & 10 \\ \hline
	5 & 9 & 9 & 9 & 2 & 10 \\ \hline
	6 & 8 & 8 & 9 & 6 & 5 \\ \hline
\end{tabular}
\caption{Participants Self Evaluations}
\label{tab:participant-scores}
\end{table}

\newcommand{\mn}{\newmoon}
\newcommand{\mns}{\fullmoon}

\subsection{Data Processing}
In the following section we will process the data from the tests. The tests where constructed using the Champagne Prototyping methodology and the analysis of the data will also follow this approach. Champagne Prototyping uses two methodologies for data Processing; Attention Investment Model and Cognitive Dimensions. In addition the code written by test participants will be microbenchmarked to estimate the execution speed of their code in relation to our reference implementations.

\subsubsection{Cognitive Dimensions}
In this experiment we first and foremost wish to test how well-suited \fsh is in the context of game development. In the side-by-side cognitive dimensions analysis we compare the participants' solutions in \fsh with those in \csh, as they were given test cases from the same category. This allow us to use a well-established vocabulary to discuss whether or not functional programming is suitable for game development.

\begin{table}[H]
	\alignCenter{
	\begin{tabular}{| l | c |}\hline
		\multicolumn{2}{|c|}{\textbf{Cognitive Dimensions}} \rowEnd
	 	Abstract Gradient & \mns \rowEnd
		Closeness of Mapping & \mns \rowEnd
		Consistency & \mns \rowEnd
		Diffuseness/Terseness & \mns \rowEnd
		Error-proneness & \mns \rowEnd
		Hard Mental Operations & \mns \rowEnd
		Hidden Dependencies & \mns \rowEnd
		Premature Commitment & \mn \rowEnd
		Progressive Evaluation & \mn \rowEnd
		Role-expressiveness & \mn \rowEnd
		Secondary Notation and Escape from Formalism & \mn \rowEnd
		Viscosity & \rowEnd
		Visibility and Juxtaposability & \rowEnd
	\end{tabular}}
	\caption{Cognitive Dimensions Findings}
	\label{tab:cog-dim-findings}
\end{table}
\tmcc{Do we need this table? It is part of the Champagne testing, but not really related to side-by-side analysis}

% The writability parameter is evaluated by examining the participants' reactions during the test and the readability by reading through the code in the subsequent analysis. Modularity is included because functional programming allows the participants to write generalised code to carry out multiple different actions when treating data collections \cite{hughes1989functional}. For instance, in the test cases of this experiment a generalised tree walker could be used in the talents task to both calculate the user's current bonus as well as the maximum achievable bonus.

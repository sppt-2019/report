\subsection{Champagne Prototyping}
In this section we analyse the video files from the usability experiment using the champagne prototyping. This method uses two different analysis strategies; attention investment and cognitive dimensions. In both of these analyses we count mentions of each feature. We present these counts using bullets in tables, which is recommended by \cite{blackwell2004champagne}. A closed bullet (\mn) means that a participant mentioned an aspect of the feature, when the participant mentioned it multiple times an open bullet is used (\mns). The categories, \textit{Correct \& Unprompted} and \textit{Correct \& Prompted} are instances where the participants mention or explain a feature correctly and/or positively. The last category are instances where features are mentioned negatively, incorrectly or is a cause for confusion.

\subsection{Attention Investment Model}
The attention investment model, presented in \secref{attention-investment}, can be used to map out the programming steps undertaken by a user. This section will endeavor to do so, using the user test video material. Firstly, user comprehension of the feature being evaluated is measured. This feature is the use of \fsh and \gls{FRP} in game development. In order to measure comprehension, the instances where participants mention aspects of the feature were recorded and categorised, as can be seen in \tableref{comp-matrix}.

\begin{table}[H]
	\alignCenter{
	\begin{tabular}{| l | l | l | l |}\hline
		\multicolumn{1}{| l |}{\textbf{Recognised Aspects}}&\multicolumn{1}{| l |}{\textbf{Correct \& Unprompted}}&\multicolumn{1}{| l |}{\textbf{Correct \& Prompted}}&\multicolumn{1}{| l |}{\textbf{Incorrect}} \rowEnd
		Modularity 			& \mn\mn & & \mn \rowEnd
		ReactTo 				& \mns & \mn\mn\mns & \mn\mn\mn \rowEnd
		Types 					& \mn\mn & & \mn\mns\mns\mns\mns\mns \rowEnd
		List Operations	& \mn & \mn\mn\mns & \mn\mns \rowEnd
	\end{tabular}}
	\caption{User Comprehension of the Feature}
	\label{tab:comp-matrix}
\end{table}

In addition to a measure of the participants comprehension, the attention investment model also provides a quantification of their efforts in the programming activity. This consists of four metrics, mentioned in \secref{attention-investment}. This can be seen in \tableref{att-inv-findings}. The risk metric measures the amount of times participants mentioned or discussed things that could go wrong, which includes increased difficulty of some tasks using \fsh. The cost metric is the attention and time required to switch to \fsh. Any musing over the difficulties of switching over is included. The payoff is the reduced cost of game development after switching to \fsh. The imperative alternative metric is the number of times participants mentioned the problems in \csh, or other imperative languages, that form the basis for switching to \fsh.

\begin{table}[H]
	\alignCenter{
	\begin{tabular}{| l | c |}\hline
		\multicolumn{2}{|c|}{\textbf{Attention Investment}} \rowEnd
		Investment Risk & \mn\mn\mn\mn\mns  \rowEnd
		Investment Cost & \mn\mns\mns\mns\mns \rowEnd
		Investment Payoff & \mn\mns\mns\mns\mns \rowEnd
		Imperative Alternative & \mn\mn\mn \rowEnd
	\end{tabular}}
	\caption{Attention Investment Findings}
	\label{tab:att-inv-findings}
\end{table}

The participants were able to correctly use and describe \fsh and \gls{FRP} behaviour in some instances and struggled in other instances. As can be seen in \tableref{comp-matrix}, not all participants were cognisant of all feature aspects, e.g. all participants misunderstood or struggled with the type system. Functional programming claims a greater degree of modularity than imperative languages\cite{hughes1989functional}, however, less than half of the participants expressed cognisance of this. Some participants wrote much more modular code in \fsh than in \csh, but did not mention it.

As can be seen in \tableref{att-inv-findings} the participants noted a high cost with a high payoff. Participants could see the usefulness of \gls{FRP}, but several participants expressed uncertainty of any benefit provided by \fsh. Another point is that very few participants noted the problems with existing solutions, even when compared to \fsh. The high cost is attributed to loss of productivity while a developer learns to use \fsh.


\subsubsection{Cognitive Dimensions}
\subsection{Cognitive Dimensions}
In this section we use the cognitive dimensions framework to aid the analysis of the video from the usability test. In this analysis we count the number of times the participants experience or mention problems with each dimension. We use the same notation that was presented in the previous section.

\begin{table}[H]
	\alignCenter{
	\begin{tabular}{| l | c |}\hline
		& \textbf{F\#} & \textbf{C\#} \rowEnd
	 	Abstract Gradient &  \rowEnd
		Closeness of Mapping &  \rowEnd
		Consistency &  \rowEnd
		Diffuseness/Terseness &  \rowEnd
		Error-proneness & \rowEnd
		Hard Mental Operations &  \rowEnd
		Hidden Dependencies &  \rowEnd
		Premature Commitment &  \rowEnd
		Progressive Evaluation &  \rowEnd
		Role-expressiveness &  \rowEnd
		Secondary Notation and Escape from Formalism &  \rowEnd
		Viscosity & \rowEnd
		Visibility and Juxtaposability & \rowEnd
	\end{tabular}}
	\caption{Cognitive Dimensions Findings}
	\label{tab:cog-dim-findings}
\end{table}

\subsection{Analysis}
\metasheep

\subsubsection{Consistency} % par 2 43:0
In the cognitive dimensions framework consistency is the coherence between the language designer's understanding and the language user's intuition of the language\cite{green1996usability}. This does not mean that consistency is the difference in language knowledge, but rather the difficulty of extrapolating behaviour and syntax of language features based on knowledge of other language features.

All participants had prior \cs experience which may have affected their expectations. This was apparent when participants applied \cs methodology in the \fs code. An example of this is the confusion of types some participants experienced. Several participants noted that they preferred strict typing or specifying types manually. 

\subsubsection{Secondary Notation}
In the test we saw very limited use of secondary notation. This is likely caused by the relatively small tasks, along with the fact that the code was not meant to be used in production nor expanded upon. The programmers, however, would often open said pop-up boxes to read about the functions and methods before putting them to use. Generally, our intuition was that the programmers found it more easy to understand the C\# documentation than the F\# documentation. We suspect that there two reasons:
\begin{enumerate}
    \item The programmers were more experienced in C\#.
    \item F\# uses the arrow function signatures (e.g. \mintinline{fsharp}|val map : mapping : ('T -> 'U) -> list:'T list -> 'U list|, which is the signature of \ttt{List.map}) in its documentation. This notation indicates that a function of multiple parameters can be curried. Curring is not supported in C\# and is thus a concept that the programmers were unfamiliar with.
\end{enumerate}
\subsubsection{Viscosity}
Viscosity defines how much effort a developer has to put in to make a small change. \cite{green1996usability} notes that textual languages are less viscous than visual programming languages.

C\# and F\# are both textual languages and thus have relatively low viscosity. We argue that the primary difference between C\# and F\# are scope delimitation. C\# scopes are delimited by pairs of curly brackets, whereas in F\# they are delimited by indentation. If a programmer is to move code from one scope to another in C\#, he would have to either insert or delete pairs of curly brackets, whereas in F\# he would select the code that needs to be moved and press TAB or Shift+TAB.

% Participant 2, 40:40 samt 1:02:30
In our test cases, viscosity is particularly visible in the \dquote{concurrent}-update category. The reason for this is that the participants are asked to develop a sequential solution first, followed by a parallel implementation. Generally viscosity is low. Given an implementation in F\# using the pipe operator, one can implement a parallel solution by piping into \ttt{Async.Parallel} and then \ttt{Async.RunSynchronously} (see \lstref{fsharp:pipe:async}).

\begin{listing}
    \begin{minted}{fsharp}
let speed = 3f;

let moveBallForward (ball:GameObject) =
    ball.transform.Translate(ball.transform.forward * Time.deltaTime * speed)

let Update () =
    let balls = GameObject.FindGameObjectsWithTag("Magnetic")
    balls
    |> Array.map moveBallForward
    ()

let UpdateAsync () =
    let balls = GameObject.FindGameObjectsWithTag("Magnetic")
    balls
    |> Array.map (fun b -> async {moveBallForward})
    |> Async.Parallel
    |> Async.RunSynchronously
    ()
    \end{minted}
    \caption{Transforming from sequential to concurrent list operations in F\#.}
    \label{lst:fsharp:pipe:async}
\end{listing}

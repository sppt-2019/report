\chapter{Concurrency and Programming in C\# and F\#}
In the previous section we uncovered that even though experienced gameplay programmers write better code in F\#, they were reluctant to switch. In order to introduce functional gameplay programming in the industry, we therefore need stronger incentive. One such incentive could be implicit (or at least simpler) concurrency. In \cite{DBLP:journals/cl/Tremblay-parallel} the author argues that the lenient evaluation strategy, which was also proposed by Sweeney, provides high implicit parallelisability. This evaluation strategy may be implemented in the \gls{FRP} system that we prototyped during the usability test.

In this section we research how the lenient evaluation strategy compares to classic concurrency strategies. During this research we found that there is a certain threshold of task-sizes, which must be exceeded before concurrency has a positive effect on execution time. We attempt to estimate this threshold in \secref{crit:work} using two tests; a busy-wait delay estimation and matrix summation.

In the final section of this chapter we measure the performance of said system to ascertain whether or not the use of functional programming in Unity results in a performance penalty. 

\section{Benchmarks}\label{sec:benchmarks}
In this section we explore the performance of the lenient evaluation strategy (see \secref{implicit-para}) and how well it parallelises. The reason for our study is that interest in the lenient evaluation strategy has been lacking to say the least. We could not find a reason as to why and decided to explore whether it was because \cite{DBLP:journals/cl/Tremblay-parallel} gave false promises of the high implicit parallelisability of leniently evaluated programs.

\subsection{Test cases}
\cite{DBLP:journals/cl/Tremblay-parallel} presents two test cases that we reuse in this experiment:

\begin{labeling}{\quad\quad}
    \item[Binary Tree Sum] where a tree-walker sums the values of all leaves of a binary tree.
    \item[Binary Tree Accumulation and Sort] where a tree-walker flattens all values of a tree's leaves into a list and sorts it.
\end{labeling}

The difference between the two test cases is that there are no data-dependencies in the summation test, i.e. we can calculate the results of the left subtree and the right subtree in parallel. In the Binary Tree Accumulation and Sort benchmark the tree must be traversed in a left-to-right order and thus there is a data-dependency between the results from the left and right subtree.

\subsection{Implementations}\label{sec:bench-impls}
We implemented the benchmark suite in both F\# and C\#. In this test F\# will be the primary test subject, as we wish to examine how well it parallelises. The C\# implementation will be used as control, in the sense that we will compare, for the different strategies, which of the languages is fastest. We are well aware that F\# is not lenient, but we wanted to compare the strategies within the same language runtime and therefore attempted to map the lenient evaluation strategy onto F\# using its async workflows. \lstref{lenient:to:task} gives a code example how to do so (a similar mapping could also be made to .NET's \m{Task}s).

\begin{listing}[H]
\begin{minted}{fsharp}
//Lenient function, body evaluated asynchronously from caller
//parameters evaluated asynchronously from function body
let CalculateC a b =
    let c = a * 50 //Implicit synchronisation with a-thread
    c + b;   //Implicit synchronisation with b-thread

//F# lenient mapping of the CalculateC-function
//async function that runs asynchronously from the caller
let CalculateCLenient a b = async {
    let! aResult = a
    let c = aResult * 50
    let! bResult = b
    return c + b
}
\end{minted}
\caption{Lenient evaluation mapping using F\# Async Workflows.} \label{lst:lenient:to:task}
\end{listing}
The mapping shown in \lstref{lenient:to:task} involves wrapping all arguments to methods in Async Workflows and explicitly synchronise using the \ttt{let!} keyword whenever there is a data-dependency between them. This means that all values of arguments are calculated asynchronously from the method body. The method is marked as \ttt{async}, meaning that an Async Workflow is spawned for every invocation of the method, i.e. it runs asynchronously from the caller. This leaves the majority of the footwork to F\#'s scheduler, which must figure out in which order to execute them.

We implemented the test cases in four different synchronization models:
\begin{labeling}{\quad\quad}
    \item[Sequential] which uses divide and conquer on one unit of execution to do all the calculations. This provides the baseline speed for the computation on a single core.
    \item[Async Workflows] which uses also uses divide and conquer, but in each recursion step, two Async Workflows are spawned to compute the results of the left and right subtree. The Workflows are then synchronised at the end of each recursive call. This provides a \textit{\dquote{classic parallel}} baseline speed.
    \item[Task] which uses C\#'s \ttt{Task}-library and divide and conquer. It is similar to the previous, except that it spawns \ttt{Task}s.
    \item[Lenient] using the mapping displayed in \lstref{lenient:to:task}, i.e. wrapping everything in Async Workflows and have the .NET Core task scheduler figure out the computation order.
\end{labeling}

\subsection{Test Setup}
According to \cite{sestoft2013microbenchmarks} the most reliable results are obtained when the tests are repeated multiple times and the average and standard deviation calculated for the results. We therefore decided to take the average execution speed over 100 repetitions for varying problem sizes, starting with 1 leaf node and gradually doubling the number up to a total of 65536 nodes.

The tests were run on a laptop, of which the specifications are listed in \tableref{sys-specs}.

\makeTable{
{| l | R{6em} | p{3em} |}
\hline
\multicolumn{3}{| c |}{\textbf{Processor}} \\ \hline
Model & \multicolumn{2}{| c |}{Intel Core i7 4702HQ} \\ \hline
Clock Frequency & 2.2 & GHz \\ \hline
Max Turbo & 3.2 & GHz \\ \hline
Physical & 4 & Cores \\ \hline
Logical\footnotemark & 8 & Cores \\ \hline
\multicolumn{3}{|c|}{\textbf{Memory}} \\ \hline
Memory Size & 16 & GiB  \\ \hline
Memory Speed & 1600 & MHz \\ \hline
Memory Type &  \multicolumn{2}{| c |}{DDR3L 1600} \\ \hline
\multicolumn{3}{|c|}{\textbf{Software}} \\ \hline
Operating System & \multicolumn{2}{| c |}{Ubuntu 18.04 64bit}  \\ \hline
C\# runtime & \multicolumn{2}{| c |}{dotnet 2.2.104} \\ \hline
}{System specifications of the test machine.}{sys-specs}\footnotetext{Logical cores are sometimes called threads. However logical cores is used here to avoid confusion with the software concept; threads, which is distinct from hardware threads.}

We wish to analyse the following research questions:
\begin{enumerate}
    \item Which of the four presented strategies handle increasing sizes of trees best?
    \item Does F\# have worse performance than C\# when running equivalent (concurrent) code?
\end{enumerate}

\subsection{Results}\label{sec:bench-res}
In this section we analyse and discuss the results based on the two research questions presented in the previous section.

\subsubsection{Parallel Strategy and Performance}
The results are plotted in \figref{binary-accumulation} and \figref{binary-summation} (and listed in \tableref{accumulation:res} and \tableref{summation:res} in \apxref{benchmark:data}).

\newcommand{\binAcumSymbolics}{\symbolic{Problem,1,4,8,16,32,64,128,256,512,1024,2048,4096,8192,16384,32768,65536}}
\barChart*[5][\binAcumSymbolics][Run Time (ns)][Number of Nodes]{Binary accumulation benchmark results in F\# (lower is better).}{binary-accumulation}{
    \plotDataWithError{Sequential}{\accumulationData}
    \plotDataWithError{Async Workflow}{\accumulationData}
    \plotDataWithError{Task}{\accumulationData}
    \plotDataWithError{Lenient}{\accumulationData}
}
\barChart*[5][\binAcumSymbolics][Run Time (ns)][Number of Nodes]{Binary summation benchmark results in F\# (lower is better).}{binary-summation}{
    \plotDataWithError{Sequential}{\summationData}
    \plotDataWithError{Async Workflow}{\summationData}
    \plotDataWithError{Task}{\summationData}
    \plotDataWithError{Lenient}{\summationData}
}

Much to our surprise, the sequential implementation was actually the fastest in all cases. It seems that the overhead of spawning and synchronising \ttt{Task}s outweighs the performance gain of concurrency when the problem sizes are in the magnitude of additions and list appending. Furthermore, the Accumulation test case presented in \cite{DBLP:journals/cl/Tremblay-parallel} is a poor choice when it comes to parallelism, as it must traverse the tree in a left-to-right manner, meaning that the only things that can be executed in parallel is the recursive calls down the tree. We suspect that the advantages of parallel programming will be more prominent, as the amount of work in each unit of execution increases. We will research this hypothesis in greater depth in the following section.

The execution times of the lenient strategy grows faster with problem size than those of the task strategy. This means that the lenient approach handle increasing sizes of trees worse. It is, however, not as bad as async workflows. We have only obtained data for async workflows up to trees containing 256 nodes, as the running times grew so rapidly that it was not feasible to continue. Strangely enough, the lenient strategy scales much better. The only difference between the two implementations are that workflows are started with \ttt{Async.StartChild} in the lenient implementation and \ttt{Async.Parallel} in the other. These types of results underline how small and seemingly irrelevant details may have a huge impact on the performance of parallel programs.

\subsubsection{Language Performance with C\# Tasks}
\figref{seq-binary-accumulation} and \figref{seq-binary-summation} shows the running time of the sequential implementation in C\# and F\#. These results show that F\# is faster in the accumulation benchmark up to tree sizes of roughly 256 nodes. In the summation benchmark C\# is faster all the way. Both data sets seem to decrease in running time up to tree sizes of 128 nodes. This is strange, as it means that the implementation handles more computations in less time. We are unsure what causes this.

\barChart*[10][\binAcumSymbolics][Run Time (ns)][Number of Nodes]{Binary Accumulation in F\# and C\# using the sequential solutions (lower is better).}{seq-binary-accumulation}{
    \plotDataWithErrorAndLegend{Sequential}{\accumulationData}{F\#}
    \plotDataWithErrorAndLegend{Sequential}{\accumulationDataCsharp}{C\#}
}
\barChart*[10][\binAcumSymbolics][Run Time (ns)][Number of Nodes]{Binary Summation in F\# and C\# using the sequential solutions (lower is better).}{seq-binary-summation}{
    \plotDataWithErrorAndLegend{Sequential}{\summationData}{F\#}
    \plotDataWithErrorAndLegend{Sequential}{\summationDataCsharp}{C\#}
}

In \figref{task-binary-accumulation} and \figref{task-binary-summation} we have plotted the running times for F\# and C\# in the two test cases. Both the C\# and F\# implementation uses the Task strategy (i.e. divide-and-conquer that spawns two tasks in each recursion step). These results align with the previous in that F\# is fastest when there is a small number of nodes in the tree (up to 128 nodes in accumulation and 8 nodes in summation). After that point C\# is faster, and it seems that C\# handle increasing number of nodes better. The strange curve we observed in the sequential data sets are also present in C\# in the binary summation benchmark. Again, we're unsure what causes this behaviour.

\barChart*[10][\binAcumSymbolics][Run Time (ns)][Number of Nodes]{Binary Accumulation in F\# and C\# using \ttt{Task} parallelisation (lower is better).}{task-binary-accumulation}{
    \plotDataWithErrorAndLegend{Task}{\accumulationData}{F\#}
    \plotDataWithErrorAndLegend{Fork Join}{\accumulationDataCsharp}{C\#}
}
\barChart*[10][\binAcumSymbolics][Run Time (ns)][Number of Nodes]{Binary Summation in F\# and C\# using \ttt{Task} parallelisation (lower is better).}{task-binary-summation}{
    \plotDataWithErrorAndLegend{Task}{\summationData}{F\#}
    \plotDataWithErrorAndLegend{Fork Join}{\summationDataCsharp}{C\#}
}

\section{Parallel Overhead \& Performance}\label{sec:crit:work}
In this section we present results from an experiment that estimates how much work need to be done in a task to outweigh the performance penalty of \ttt{Task} synchronisation. We also implement a matrix summation benchmark to determine how different parallelisation strategies handle matrices of increasing sizes.

\subsection{Estimating Minimum Concurrent Workload}
In this experiment we estimate the minimum concurrent workload of C\#'s \ttt{Task}-system. By minimum concurrent workload we mean how much time each task must execute before it is worthwhile to spawn it, compared to a sequential solution. The reason for this exploration is that we found the sequential solution to be faster in the binary tree benchmarks presented in the previous section.

\subsubsection{Test Setup}
We use the Binary Tree Summation benchmark presented in the previous section with a minor modification: Every time the algorithm finds a \ttt{Node}, it busy-waits for a given amount of time to simulate work. The busy-wait was implemented with a loop, whose number of iterations is gradually halved until the sequential solution executes faster than the parallel. The hypothesis here is that the parallel solutions will be faster, because it is capable of busy-waiting multiple tasks at the same time.

We implemented the test in two variations, which are both listed in \lstref{benchmark:strategies}. The first variation emulates a data dependency between the wait and the results the subtrees, i.e. the wait is intended to emulate a computation that must be carried out after the results of both subtrees have been computed. The other variation emulates a situation where the left and right subtree can be computed in parallel with the wait, i.e. no data dependency between the delay and the subtrees. The tree has a total of 60 leaf nodes. In addition to the binary tree summation, we also implemented a N-ary tree summation in the same variations as that of binary.

\begin{listing}
    \begin{minted}{csharp}
public static int DoFakeWork(int workBias) {
    //Start the 'work bias' before blocking wait on child computation, i.e. we're waiting while the children
    //are computing
    var fakeSum = 0;
    for(var i = 0; i < workBias / 2; i++) {
        fakeSum += i;
    }
    for (var i = 0; i < workBias / 2; i++) {
        fakeSum -= i;
    }
    return fakeSum;
}

public static async Task<int> SumLeaves(Tree<int> tree, int workBias)
{
    if (tree is Leaf<int> leaf)
        return leaf.Value;

    var sums = tree.Children.Select(c => SumLeaves(c, workBias)).ToList();
    
#if !DELAY_DEPENDS_ON_LR
    var wb = Task.Run(() => DoFakeWork(workBias));
    await Task.WhenAll(sums);
    var fakeSum = await wb;
#else
    await Task.WhenAll(sums);
    var fakeSum = DoFakeWork(workBias);
#endif
    
    return sums.Sum(t => t.Result) + fakeSum;
}
    \end{minted}
    \caption{Implementation of the two different data dependency strategies with an N-ary tree. The strategy may be selected by either defining or undefining the \ttt{DELAY_DEPENDS_ON_LR} preprocessor flag.}
    \label{lst:benchmark:strategies}
\end{listing}

\subsubsection{Results}
The results are plotted in \figureref{crit-work-dep} and \figureref{crit-work-no-dep} (and listed in \tableref{binary:tree:with:bias:dependency} and \tableref{binary:tree:with:bias:no:dependency} in \apxref{crit:work:data}).

\newcommand{\workBiasSymbolics}{\symbolic{Work Bias (iterations),134217728,67108864,33554432,16777216,8388608,4194304,2097152,1048576,524288,262144,131072,65536,32768,16384,8192,4096,2048,1024}}
\lineChart{Critical workload with data dependency.}{crit-work-dep}{
    \plotData{Sequential}{\workWithDependencyData}
    \plotData{Fork Join}{\workWithDependencyData}
    \plotData{Lenient}{\workWithDependencyData}
}
\lineChart{Critical workload without data dependency.}{crit-work-no-dep}{
    \plotData{Sequential}{\workWithoutDependencyData}
    \plotData{Fork Join}{\workWithoutDependencyData}
    \plotData{Lenient}{\workWithoutDependencyData}
}
\lineChart{Critical workload with data dependency, N-ary tree.}{crit-work-dep-nary}{
    \plotData{Sequential}{\workWithDependencyDataNary}
    \plotData{Fork Join}{\workWithDependencyDataNary}
    \plotData{Lenient}{\workWithDependencyDataNary}
}
\lineChart{Critical workload with data dependency, N-ary tree.}{crit-work-no-dep-nary}{
    \plotData{Sequential}{\workWithoutDependencyDataNary}
    \plotData{Fork Join}{\workWithoutDependencyDataNary}
    \plotData{Lenient}{\workWithoutDependencyDataNary}
}

\makeTable{
    { c | c | c }
    & No data dependency & Data dependency \\\hline
    Binary & 1024 & 4096 \\
    N-ary & 2048 & 2048
}{Iterations of the busy-wait loop before the sequantial solution becomes the fastest.}{crit:work:iterations}

The number of iterations in the busy-wait loop before the sequential solution is faster than the concurrent is listed in \tableref{crit:work:iterations}. The graphs also underline that our hypothesis was correct. The parallel solutions grows slower than the sequential because they are capable of executing multiple busy-waits concurrently. Finally we notice that the lenient and fork join strategy lie very close in execution speed. This is a promising result for the lenient evaluation strategy, as it shows that it may be as fast as a traditional concurrency strategy.

Some conccurency models batch smaller jobs together to form larger jobs\needcite. Such batching may reduce the time spent context switching and thus increase the execution speed of the concurrent solutions. Such strategy is employed by Unity's C\# job system\cite{unity:csharp:job:system}.

\subsection{Matrix Summation}
In this section we execute a matrix summation benchmark. This benchmark measures the time it takes to sum all indices of a random $N x N$ matrix. This benchmark was implemented in different parallelisation strategies to explore how well they scale to increasing sizes of $N$:

\begin{labeling}{\quad\quad}
    \item[Sequential] utilises a double-nested for-loop to iterate over the matrix and sum the values. This benchmark provides a baseline value for running the computation on one thread.
    \item[Map Reduce] maps a function that sums each column over the matrix. The resulting list of column sums is then reduced to the overall sum of the matrix. In C\# we utilise the \gls{LINQ}-methods \ttt{Select}, \ttt{Sum} and \ttt{Aggregate}.
    \item[Parallel Foreach] uses a parallel loop to iterate over the columns of the matrix that may execute the summation of each column in parallel.
    \item[Tasks] is similar to parallel foreach, with the only exception that we manually spawn a \ttt{Task} that calculates the sum of each column.
\end{labeling}

We have not included a lenient-variation in this experiment, as an implementation in our C\# mapping would be largely equivalent to the Tasks-implementation (see \lstref{matrix-sum-csharp}). The most notable difference being that a lenient-evaluation strategy would most likely also construct the matrix in parallel with the summation. As the time it takes to construct a matrix is not included in the results here, this should have no effect on the validity of the results.
\begin{listing}
\begin{minted}{csharp}
public static async Task<long> SumTask(long[,] matrix)
{
    //Create an enumerable over the columns of the matrix
    var columns  = Enumerable.Range(0, matrix.GetLength(0));
    //Sum each column in parallel
    var sums = columns.Select(c => Task.Run(() =>
    {
        var sum = 0L;
        for(var i = 0; i < matrix.GetLength(1); i++)
        {
            sum = unchecked(sum + matrix[c, i]);
        }

        return sum;
    })).ToList();

    //Join the resuls and sum the sums of each column
    await Task.WhenAll(sums);
    return sums.SumUnchecked();
}
\end{minted}
\caption{Tasks implementation of Matrix Sum, largely equal to a lenient C\# mapping.} \label{lst:matrix-sum-csharp}
\end{listing}

As the matrices are of size $N x N$, they contain a total of $N^2$ elements with random values between \ttt{Int64.Minvalue} and \ttt{Int64.MaxValue}. When running the test with large matrices we found that the result would overflow, which throws an exception because C\# is a managed language. In order to avoid this, we used the \ttt{unchecked}-keyword, which disables bounds-checking on an integral arithmetic operation\cite{csharp:unchecked}. \cite{csharp:unchecked} states that using \ttt{unchecked} \textit{\dquote{might improve performance}}, compared to checked integral arithmetic operations.

\subsubsection{Results}
The results are plotted in \figureref{linpack-summation}. The first thing to notice is that Map Reduce seems to be roughly equal to the sequential in running time. This could indicate that the \ttt{Select}-method of C\#'s \gls{LINQ}, which was used to implement Map Reduce, does not parallelise its iterations. We will thus treat Map Reduce as a sequential solution for the rest of this result discussion.

\newcommand{\linpackSymbolics}{\symbolic{Problem Size,2,4,8,16,32,64,128,256,512,1024,2048,4096}}
\barChart*[7][\linpackSymbolics]{Matrix Summation}{linpack-summation}{
    \plotDataWithError{Sequential}{\linpackData}
    \plotDataWithError{Map Reduce}{\linpackData}
    \plotDataWithError{Parallel Foreach}{\linpackData}
    \plotDataWithError{Tasks}{\linpackData}
}
In general, the results from this experiment is in alignment with those of the previous, in that there is an initial overhead associated with concurrency. In this case, it seems the sequential and concurrent solutions evens out at job sizes of around 256 summations, after which point the concurrent solutions are faster.

After overcoming the initial overhead, the concurrent solutions handle increasing matrix sizes much better than their sequential counterparts. This is even more notable in \figureref{linpack-summation-line}, which plots the same data as a line and without logarithmic y-axis. As the matrix sizes continue to grow, it may be possible to split the columns in multiple separate tasks, possibly making the concurrent implementations faster yet.

\lineChart{Matrix Summation}{linpack-summation-line}{
    \plotData{Sequential}{\linpackData}
    \plotData{Map Reduce}{\linpackData}
    \plotData{Parallel Foreach}{\linpackData}
    \plotData{Tasks}{\linpackData}
}

\subsection{Discussion}
Our results show, first and foremost, that there is an initial overhead in concurrent programming. In order to speed up a computation the task sizes must exceed a certain threshold. Our results show that this threshold lies around the size of 1024-4096 iterations in a for-loop that increments a variable\tmc{Måske skal vi tælle nogle instruktioner her?}, but also that it varies with the type of problem. Furthermore, the results show that the concurrency of lenient evaluation strategy is similar to that of classic concurrency strategies, but that the choice of problems in \cite{DBLP:journals/cl/Tremblay-parallel} is not well suited to parallelisation. Finally C\#'s \ttt{Task}-model seems to handle the matrix summation problem well, after exceeding the threshold of roughly 256 summations in each column.

As part of the project we implemented a simple \gls{FRP} system in Unity, because this particular programming paradigm is well suited to gameplay programming in functional languages\needcite. We had initially decided that this experiment would use either lenient evaluation or Async Workflows under the hood to update \ttt{MonoBehaviour}s concurrently, but discovered too late that Unity uses a custom concurrency strategy called Unity C\# Job System\cite{unity:csharp:job:system} (see \secref{unity:concurrency}). Unity therefore does not allow \ttt{MonoBehaviour} updates from \ttt{Task}s and Async Workflows\cite{unity:async}. The resulting \gls{FRP} system therefore runs sequentially.
\section{Performance Benchmarking the FRP System}
In this chapter we examine the performance of F\# and our \gls{FRP} system in Unity. We first examine the aspect of \gls{GC} by looking at Unity best-practice guidelines, which suggests that garbage is to be avoided to the extend that it's possible. We were not aware of this when we implemented reference solutions to the eight usability test cases that we presented in \secref{usability:test:cases}. We therefore adapt one of the solutions to conform to Unity best-practice guidelines and benchmark that against a more \dquote{careless} implementation.

\subsection{Unity Garbage Collection}
In this section we first examine best practices for developing applications in Unity with a particular focus on garbage. We then list different \gls{GC} algorithms, briefly characterise them and investigate which algorithms are used in Mono, dotnet and Unity. Finally we measure the running times of F\# against C\# in Unity and a functional map-based approach against an imperative one.

\subsubsection{Best Practices}
Unity recommends careful memory management when writing in C\# and avoiding unnecessary heap allocations\cite{unity:optimisation}. The performance optimisation guideline in \cite{unity:optimisation} lists many common performance bottlenecks for Unity developers. The most notable of those are lack of caching and extensive use of boxing. Unity provides many methods and properties that allow developers to access collections of components, such as the \ttt{GameObject.FindObjectsWithTag} method and \ttt{Mesh.vertices} property\cite{unity:optimisation, unity:heap}. The implementation of those methods will allocate a new array for the objects behind the scenes every time they're invocated. We list an example of this from \cite{unity:heap} as \ttt{Wrong} in \lstref{unity:array:prop}. In the example \ttt{mesh.veritices} might seem like an innocent property access, but each time the property is accessed, a new array is allocated. This means that the code allocates four new arrays in every iteration of the loop. This puts a huge burden on the \gls{GC} and will, according to \cite{unity:heap}, result in noticeable performance degradation. Instead, developers should use the code listed as \ttt{Correct} in \lstref{unity:array:prop}, which does the exact same, but only allocates one array for all iterations, due to better use of caching.

The problems highlighted in \lstref{unity:array:prop} are an instance of common subexpression elimination, and one could speculate whether or not Unity's C\# compiler should be capable of performing such optimisations. Nevertheless, Unity's best practice guidelines list them as an example and explain how developers should transform their code manually\cite{unity:heap}.

\begin{listing}
\begin{minted}{csharp}
//sample implementation of mesh.vertices
class Mesh {
    public Vector3[] vertices {
        get {
            var verts = new Vector3[/*number of vertices*/]
            //find the vertices and put them into verts
            return verts;
        }
    }
}

//Wrong
for(var i = 0; i < mesh.vertices.Length; i++)
{
    float x, y, z;

    x = mesh.vertices[i].x;
    y = mesh.vertices[i].y;
    z = mesh.vertices[i].z;

    DoSomething(x, y, z);
}

//Correct
var verts = mesh.vertices;
for (var i = 0; i < verts.Length; i++) {
    DoSomething(verts[i].x, verts[i].y, verts[i].z);
}
\end{minted}
\caption{Common performance bottleneck in Unity \cite{unity:heap}. \ttt{mesh.vertices} should be cached. Example is taken from \cite{unity:heap}.} \label{lst:unity:array:prop}
\end{listing}

The problem of boxing occurs when a value-type should be used by reference, for instance when constructing a list of integers or appending a float to a string. This generates a small amount of garbage, which can quickly accumulate, e.g. during list iterations. Furthermore, \cite{unity:optimisation} underlines the importance of avoiding \gls{LINQ}-statements all together, due to the garbage generated under the hood. \cite{unity:heap} recommends avoiding coding styles that requires passing functions as arguments and to completely avoid closures, due to the amount of garbage generated by said language constructs. This is in conflict with many functional idioms, which we will explore later.

\subsubsection{Garbage Collection Algorithm}\label{sec:gc-strat}
Unity uses the Boehm–Demers–Weiser \gls{GC}, which is a conservative mark-sweep \gls{GC}\cite{unity:heap}, originally created for automatic memory management in C and C++\cite{boehm2007transparent}. Mark-sweep algorithms are the simplest type of \glspl{GC} and have the primary disadvantages that they halt computation while running, increase in execution time as more objects are allocated and may fragment memory\cite{sestoft2017programming}.

The dotnet runtime uses a generational \gls{GC} with three generations for smaller objects and a single generation for large objects\cite{dotnet:gc}. The younger generations are collected more often than the older and all surviving objects are moved to the older generations. Each time an older generation is collected, all younger generations are also collected. Generational \glspl{GC} have the advantage that short-lived object allocations have a smaller performance penalty, but the disadvantage that they introduce additional overhead if old objects contain references to young objects\cite{sestoft2017programming}. Depending on the system the dotnet runtime may use different \gls{GC} strategies, including concurrent versions\cite{dotnet:gc}. Concurrent \glspl{GC} can collect garbage concurrently with the computation, meaning that \gls{GC} pauses are minimised or entirely removed\cite{dotnet:gc}.

Mono has previously used the Boehm–Demers–Weiser \gls{GC}, but has since moved to a concurrent, generational \gls{GC} called sgen\cite{mono:gc}. We have previously mentioned that Unity uses the Mono runtime, which may cause some confusion, so a clarification is in order. Unity supports two different runtimes: Mono and IL2CPP. Unity's Mono runtime is a fork of the official Mono runtime\cite{unity:mono:github}, meaning that updates to the official Mono are not necessarily applied to Unity's Mono runtime. The IL2CPP runtime \gls{AoT} compiles code in \gls{IL} to C++, which also uses the Boehm–Demers–Weiser \gls{GC}\cite{il2cpp:gc}. However, as part of Unity's 2019.1.0 release an experimental \textit{\dquote{incremental garbage collector, which should reduce stutters and time spikes}} was added\cite{unity:roadmap}.

\subsubsection{Functional Programming and Garbage Collection}\label{sec:func-garbage}
All these recommendations stand in direct contrast to the common practices employed in the functional programming paradigm. In functional programming it's typical to map over collections, which has two problems compared to this Unity performance guideline:
\begin{enumerate}
    \item map allocates a new collection instead of mutating the existing collection.
    \item map requires a function as one of the arguments, which defines what should happen to each of the elements in the collection.
\end{enumerate}
This practice also extends to other generalised constructs, such as the tree-walker\cite{normark2008mapping}. These guidelines explain why Unity Technologies does not want to add F\# support, despite over 3500 votes on their feedback forums in April 2018\cite{unity:fsharp}. The vote was later closed by Unity, without any explanation\footnote{In previous work we have cited the Unity forums to support this claim\cite{p92018gameplay}, but as of February 2019 Unity has closed their feedback forums, meaning that this citation is no longer valid.}.

\subsubsection{Investigating Performance}
%Unity's performance guidelines regarding \gls{GC} seems to convey the message that functional-style programming should be avoided in Unity. However,
\gls{GC} is not the only thing that may affect performance in a managed language. There is also the problem of calling from the native (or unmanaged) code to the managed code. An investigation of Unity's integration with the managed runtime shows that a there is a considerable overhead in calling the pre-defined \ttt{MonoBehaviour}-methods (such as \ttt{Update}) in Unity 5.2.2\cite{unity:runtime:calls}. In \cite{unity:runtime:calls} the author sets up two different scenes:
\begin{enumerate}
    \item A scene containing 10,000 separate \ttt{MonoBehaviour}s with an \ttt{Update}-method that increments a variable.
    \item A scene containing one \ttt{MonoBehaviour}, which contains an array of 10,000 objects. Each time the \ttt{Update}-method is called, the \ttt{MonoBehaviour} iterates through the 10,000 objects and calls a custom \ttt{MyUpdate}-method.
\end{enumerate}
On an iPhone 6 the first approach took an average of 5.4ms to update the 10,000 objects, whereas the second took 0.22ms\cite{unity:runtime:calls}. In the first approach only 0.4\% of the time is spent actually executing the \ttt{Update}-code, the remaining 99.6\% is spent doing sanity checks, iterating \ttt{MonoBehaviours} and instrumenting calls from the native code into the runtime\cite{unity:runtime:calls}.

\subsubsection{Test Setup}
The question then arises if the (potentially) increased overhead from \gls{GC} can be outweighed by having a single \ttt{MonoBehaviour} manage several other behaviours in the same scene. In order to investigate, we reused the implementation of the Unit Management test case from the usability test (see \secref{usability:test:cases}). This solution is listed in \lstref{test:case:ai}. This test case may be solved by creating a collection of tuples: \ttt{(Unit, State)}. The state machine contains a series of unit management methods; one for each state. These methods take a unit as argument and returns a state. At each iteration the corresponding state's methods are mapped over the collection to create a new collection of game objects and their updated state, which is stored for the subsequent update. This approach avoids dealing with the problems of updating the list while iterating and potentially applying two updates to one \ttt{GameObject}. The advantage is that a single \ttt{MonoBehaviour} is in charge of updating all units in the \dquote{Realtime Strategy Game} and the disadvantage is that it generates substantially more garbage, as a new collection is allocated at each \ttt{Update}. We refer to this method as \dquote{Inverse} in the remainder of this section.

The other approach, here referred to as \dquote{Normal}, creates a \ttt{MonoBehaviour} for each unit, which contains it's own state machine. This has the advantage that we can exploit caching and generate less garbage, as suggested by Unity Technologies\cite{unity:optimisation}. It comes at the disadvantage that each unit must have its own \ttt{Update}-method, potentially introducing a large overhead\cite{unity:runtime:calls}.

\begin{listing}
\begin{minted}{csharp}
private List<(State state, Unit unit)> _stateList;
public void Update()
{
    //Apply updates and store the updated states in a list
    var newStates = _stateList.Select(s =>
    {
        switch (s.state)
        {
            case State.Fleeing:
                return Flee(s.unit);
            case State.Moving:
                return Move(s.unit);
            case State.Attacking:
                return Attack(s.unit);
            default: return (State.Moving, s.unit);
        }
    }).ToList();

    //zip the list with the old states to create tuples: (new state, old state)
    foreach (var statePair in newStates.Zip(_stateList, (sNew, sOld) => (sNew,sOld)))
    {
        //Compare old state and new state, initialise the unit for the new state if changed
        if (statePair.sNew.state != statePair.sOld.state)
        {
            var unit = statePair.sNew.unit;
            _initialiseState(statePair.sNew.state, statePair.sNew.unit);
            //Create a new list containing the updated unit
            _stateList = _stateList.Select(s => s.unit == unit ? (state, unit) ? s);
        }
    }
}
\end{minted}
\caption{Possible solution for the Unit Management test cases.}
\label{lst:test:case:ai}
\end{listing}

\subsubsection{Methodology}
We decided to implement the two approaches in both C\# and F\#. We run the tests in Unity 2019.1.0f2 and unless otherwise stated, the IL2CPP runtime is used. In all test cases we used a \ttt{MonoBehaviour} written in C\# to measure the time between each \ttt{Update}-call, i.e. the time it takes to generate a frame. We decided to run the test in five setups with 500, 1000, 1500, 2000 and 2500 units. For each setup we generated 900 frames, as that corresponds to 15 seconds of gameplay at 60 \gls{FPS}. Each measurement was added to a \ttt{HashSet}, which was written to a CSV file after the test. This means that the measurements include all game-related code, both including rendering, physics and so alike. However, as this system is ultimately going to be used to develop games, we conclude that delta time (or equivalently \gls{FPS}) is a sufficient metric, as that is of utmost importance to the player.

The following research questions outlines the intent of the experiment:
\begin{itemize}
    \item How does the performance penalty from extensive garbage generation compare to the performance penalty from an increased number of calls between unmanaged and managed runtimes?
    \item Does \gls{AoT}-compilation in the IL2CPP runtime actually provide a speed up?
    \item Does the use of F\# (and \gls{FRP}) introduce an additional overhead?
    \item Unity introduced a new incremental \gls{GC} in Unity 2019.1. Does this new \gls{GC} provide a speed-up when using either C\# or \gls{FRP}?
\end{itemize}

\subsection{Results}\label{sec:unity-garbage-res}
In this section we discuss the results from the benchmarks. We use the questions presented in the previous section as baseline for the discussion.

\subsubsection{Performance Penalty from Extensive Garbage Generation}
The results are listed in \tableref{unity:ai} and plotted in \figref{ai:benchmark}. The results indicate that F\# adds a small overhead, which increases as the number of units grow. This can be seen by comparing C\# Normal and F\# Normal. Furthermore, the C\# Inverse also adds a small overhead compared to C\# Normal. This could indicate that Unity has optimised the calls between native and managed code since v5.2.2. We also observe that the inverse \gls{FRP} state machine performs notably worse than the other approaches as the number of units grow. The reason is that each unit is in fact an entire \gls{FRP}-system with condition-checking and event-dispatching. We outline in \secref{fw:frp:optimisation} how a (potentially) more performant system could be structured.

\makeTable{{| P{4cm} | S[round-precision=2] | S[round-precision=2] | S[round-precision=2] | S[round-precision=2] |}
        \hline
        \bfseries Number of Units & \bfseries C\# & \bfseries C\# Inverse & \bfseries F\# & \bfseries FRP Inverse
        \csvreader[head to column names]{00-data/ai-benchmark.csv}%
        {1=\strategy, 2=\csharp, 3=\cinverse, 4=\fsharp, 5=\frp} % <Column number>=<Macro>
        {\\\hline\strategy & \csharp & \cinverse & \fsharp & \frp}%
        \\\hline}%
{Average framerate when simulating the given number of units in Unity's Mono runtime.}{unity:ai}%

\barChart*[12][\symbolic{Strategy,500,1000,1500,2000,2500}][Average FPS][Number of Units]{Average FPS in Unit Management benchmark using the Mono runtime (higher is better).}{ai:benchmark}{
    \plotData{Csharp Normal}{\aiBenchmarkData}
    \plotData{Csharp Inverse}{\aiBenchmarkData}
    \plotData{Fsharp Normal}{\aiBenchmarkData}
    \plotData{FRP Inverse}{\aiBenchmarkData}
}

\subsubsection{Performance of Runtimes}
The results, listed in \tableref{unity:ai:runtime} and plotted in \figref{ai:benchmark:runtime}, show that IL2CPP does not necessarily provide a speed up. We deem this as C\# Normal is faster in Mono, whereas IL2CPP provides roughly two more \gls{FPS} in C\# Inverse and F\# Normal.

Another interesting observation we made during the test is that there is a very large spike in the time it takes to generate the third frame. This spike is around 20 times the time it takes to generate the other frames. One could explain this spike in Mono as runtime-optimisation, but as it is also present in IL2CPP, which is \gls{AoT}, that cannot be the case. We do therefore not know what causes the spike.

\makeTable{{| P{4cm} | S[round-precision=2] | S[round-precision=2] |}
        \hline
        \textbf{Runtime} & \textbf{IL2CPP} & \textbf{Mono}%
        \csvreader[head to column names]{00-data/ai-benchmark-runtimes.csv}%
        {1=\strategy, 2=\iltocpp, 3=\mono} % <Column number>=<Macro>
        {\\\hline \strategy & \iltocpp & \mono}%
        \\\hline
        }%
{Average framerate in Unity's two runtimes measured with 1500 units in the scene.}%
{unity:ai:runtime}%


\barChart[12][\symbolic{Strategy,Csharp Normal,Csharp Inverse,Fsharp Normal,FRP Inverse}][Average FPS][Strategy]{Average FPS of the two different runtimes in the Unit Management benchmark (higher is better).}{ai:benchmark:runtime}{
    \plotData{Mono}{\aiBenchmarkRuntimesData}
    \plotData{il2cpp}{\aiBenchmarkRuntimesData}
}

\subsubsection{Performance of the FRP-system}
The results are plotted in \figref{ai:benchmark:overhead}. The results show that \gls{FRP} introduces additional overhead. This overhead results in a decrease of ten \gls{FPS} on average over the 900 frames. On the other hand, the \gls{FRP}-system yields a smoother curve, which means that the game will be subject to less stuttering and fewer lag spikes.

\lineChart[enlarge x limits=false,xtick={0,100, 200, 300, 400, 500, 600, 700, 800, 900}][FPS][Frame No.]{FPS for each frame in FRP and C\# Inverse using the Mono runtime (higher is better).}{ai:benchmark:overhead}{
    \plotUnmarkedData{Csharp}{\aiBenchmarkOverheadData}
    \plotUnmarkedData{FRP}{\aiBenchmarkOverheadData}
}

\subsubsection{Unity's Incremental Garbage Collector}
The results from running the C\# Normal implementation with the two different runtimes and \glspl{GC} are plotted in \figureref{ai:benchmark:csharp:gc:bar}. These results show that the two \glspl{GC} perform more or less equivalently. It might even seem that the incremental \gls{GC} performs worse than the original after the framerate stabilises after the 350th frame.

In general, the incremental \gls{GC} has lower variation, except from Mono after frame 650 where the \gls{FPS} varies wildly (which can be seen by the high standard deviations in \figref{ai:benchmark:csharp:gc:bar}).

\barChart[5][enlarge x limits={value=0.035,auto},width=1.3\textwidth,xtick=data,\symbolic{0-49,50-99,100-149,150-199,200-249,250-299,300-349,350-399,400-449,450-499,500-549,550-599,600-649,650-699,700-749,750-799,800-849,850-899}][FPS][Frame No.]{Average FPS over 50 frames in C\# using the two different Unity GCs (higher is better).}{ai:benchmark:csharp:gc:bar}{
  \plotDataWithError{il2cpp}{\aiCsharpGCDataCol}
  \plotDataWithError{Mono}{\aiCsharpGCDataCol}
  \plotDataWithError{Incremental il2cpp}{\aiCsharpGCDataCol}
  \plotDataWithError{Incremental Mono}{\aiCsharpGCDataCol}
}

\barChart[5][enlarge x limits={value=0.035,auto},width=1.3\textwidth,xtick=data,\symbolic{0-49,50-99,100-149,150-199,200-249,250-299,300-349,350-399,400-449,450-499,500-549,550-599,600-649,650-699,700-749,750-799,800-849,850-899}][FPS][Frame No.]{Average FPS over 50 frames in F\# using the two different Unity GCs (higher is better).}{ai:benchmark:fsharp:gc:bar}{
  \plotDataWithError{il2cpp}{\aiFsharpGCDataCol}
  \plotDataWithError{Mono}{\aiFsharpGCDataCol}
  \plotDataWithError{Incremental il2cpp}{\aiFsharpGCDataCol}
  \plotDataWithError{Incremental Mono}{\aiFsharpGCDataCol}
}

We also tested the two \glspl{GC} with the \gls{FRP}-system we developed in F\#. The results are plotted in \figref{ai:benchmark:fsharp:gc:bar}. These results show that the incremental \gls{GC} performs slightly better than the original one. However, it comes at the cost of much higher variation in the framerate, especially in the Mono runtime. The IL2CPP incremental curve has large spikes up until around the 200th frame, after which point it stabilises. The Mono curve continues to variate between an average of 10 \gls{FPS} to over 30 \gls{FPS}.



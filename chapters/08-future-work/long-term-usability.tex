\section{Longer Term Usability Evaluation}\label{sec:long-term}
The usability evaluation presented in this project gives only a small glimpse into the challenges faced by experienced C\# programmers when they start using F\#. We suspect that part of the reason why the participants were not keen on adopting F\# was that they only experienced the \dquote{initial frustration} of switching to a new language. Another interesting topic to explore is whether experienced gameplay programmers are more positive towards F\# after gaining more experience. This would require a longer-term usability evaluation to be conducted.

A longer-term usability evaluation presents the challenge that the programmers may not be monitored all the time, thus decreasing the amount of insight we can obtain on F\#'s learning curve. If we assume that it is not necessary to monitor the participants, we can draw inspiration from the expert-review method presented in \cite{nanz2013examining} and the idea of structuring exercises as game prototypes presented in \cite{bolhuis2019gameplay}. This combined method would formulate a series of game prototype exercises, and send them to participants at regular interval along with a questionnaire that examines how happy the participant was with using F\#. Alternatively several small usability test sessions could be arranged, where the participants solve similar exercises. Both these approaches require a huge time investment from the participants and the latter requires a lot of planning. Nonetheless, these approaches may provide another interesting point-of-view as to why functional programming has not been broadly accepted in the game development industry.

Alternatively a comparative experiment could draw inspiration from \cite{hanenberg2010experiment}, that studies the effect of dynamic and static types on programming languages. Such experiment would gather a larger groups of participants and have half of them implement a game prototype in C\# and half in F\#. The total development time should be roughly 27 hours\cite{hanenberg2010experiment}. Alternatively, online resources could be used to evaluate the usability, by having the participants complete an online course before the actual session.

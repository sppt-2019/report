\section{Lenient Evaluation in F\#}
In this section we present an avenue for future research into a lenient parallelisation system in \fs. The theoretical background for this system is rooted in the lenient evaluation strategy and the work/span work estimation system (see \secref{implicit-para} and \secref{work-span}). Two different pilot implementations were tested using Async Workflows from \fs and .NET \m{Task}s (see \secref{benchmarks} and \secref{bench-impls}). The two parallelisation systems promise fine grain parallelisation, which is suited for the lenient system.

The performance of Async Workflows was not up to par with other approaches (at least not when workflows are started without the explicit \ttt{Async.StartChild}), however \m{Task}s presented a promising system for lenient parallelisation in \fs. This can be seen in \figref{binary-accumulation} and \figref{binary-summation}. The pilot implementation is a naïve implementation of lenient evaluation in \fs. The system could benefit from using work/span and hardware information to make informed decisions about what and when to parallelise. The benefit of using work/span, is that the analysis can be made at compile time.

However, the above results are not concrete proof that such a system could provide a significant speed up in real world scenarios. Therefore a new implementation, using work/span or a different analysis tool, should be implemented and benchmarked. This benchmark should be compared to state of the art sequential and concurrent implementations to ascertain relative speed up. However, if the speed up is not significant, the system may still hold merit as a potentially easier to use parallelisation system.

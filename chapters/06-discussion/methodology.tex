\section{Methodology}
In this section we will discuss the chosen methodology for this project. A number of different methodologies were presented in \secref{prog-usability}, of these we opted for \champagne. None of the methodologies were exact matches for our case, therefore we modified \champagne to fit our case better. This modification consisted of including additional \attentions and \cognitive usability methodologies.

\subsection{Champagne Prototyping}
The \champagne method was selected because of its cheap deployment cost and the availability of a \fs plugin for unity\cite{fsharp2019plugin}. This meant that early in the project the necessities of the method were met (see \secref{champagne}), i.e. a fully operational prototype based on an existing product. However, \discount was also an option. This methodology only requires a set of problems and some participants to solve them. The main drawback of \discount in this case is the lack of an \gls{IDE}, which is considered a part of the programming language in this report, based on \cognitive.

This left \expert as an alternative to \champagne, however, as the name implies, this method requires an expert in each language under test. In addition, participants would be required to participate in multiple sessions. This was a concern, as finding qualified participants willing to spend an hour on the test already proposed a challenge. Given this challenge, \champagne was the method which provided the most benefit at the smallest cost. This freed up time to improve the test setup and explore features such as the \gls{FRP} system.

\subsubsection{Attention Investment Model}
The primary contribution of the \attention in this project is the overview of participant comprehension. This can be seen in \tabref{comp-matrix}, where an overview of what the participants understood of the \gls{FRP} system. The method was used as a broad-brush measure to participant comprehension, however it can be used as a finer grain tool. When used in this way, the method could be used to simulate participant behaviour and thus explain their decision making process\cite{blackwell2002first}. This approach was not used due to the significant effort required. In addition the authors did not provide clear direction on how to conduct such a test in the papers presenting the method\tmcc{Too aggressive?}.

Furthermore, the \attention method was designed to map the decisions made by the programmer during a programming activity. It can be used to explore why a programmer may opt for a manuel solution instead of a programmatic solution and provides a vocabulary to discuss this. However, in \champagne it is targeted at a certain feature. We used to methodology in accordance with \champagne and in a similar manner to \cognitive; as a vocabulary to discuss the decision-making  process of the participants (see \secref{att-inv-app}).

\subsubsection{Multipass Cognitive Dimensions}
The \cognitive framework is used in \champagne, but we also used the framework on its own. The reason for this was to discuss our experience with \fs and \cs separately from the participants experience. This allowed us to compare the experiences and make educated guesses on whether some issues are resolved with more practice. This means that \cognitive was applied twice, first as vocabulary to compare \cs and \fs, and second as part of \champagne to gain an overview of usability issues. The first application was in the same style as our previous study of gameplay programming languages\cite{p92018gameplay}.

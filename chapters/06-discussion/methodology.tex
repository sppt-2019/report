\section{Methodology}
In this section we will discuss the chosen methodology for this project. A number of different methodologies where presented in \secref{prog-usability}, of these we opted for \champagne. None of the methodologies where exact matches for our case, therefore we modified \champagne somewhat to fit our case better. This modification consisted of including additional \attentions and \cognitive usability methodologies.

\subsection{Champagne Prototyping}
The \champagne method was selected because of it's cheap deployment cost and the availability of a \fs plugin for unity\cite{fsharp2019plugin}. This meant that early in the project the necessities of the method where met (see \secref{champagne}), i.e. a fully functional prototype based on an existing product. However, \discount was also an option. This methodology only requires a set of problems and some test participants to solve them. The main drawback of \discount in this case is the lack of an \gls{IDE}, which is considered a part of the programming language in this report, based on \cognitive.

This left \expert as an alternative to \champagne, however \expert requires an expert in each language under test. In addition, a participants would need to participate in multiple sessions. This was a concern, since finding participants willing to spend an hour on the test was challenging\tmcc{Weak argument?}.

\subsubsection{Attention Investment Model}

\subsubsection{Multipass Cognitive Dimensions}

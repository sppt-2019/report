\section{F\# Adoption Potential}
While \fs code produced during the user test, was in general shorter than the \cs code (see \appref{terse-diff-comp}), the participants expressed that they would not personally choose \fs in a real project. Even participants who clearly identified these benefits of \fs, held this opinion. In this we will explore the stated reasons for this standpoint. The participants stated a number of rationals for this during the debriefing interview, which have been categorised below. Each category is, in turn, analysed and explained.

\subsection{Prevalence and Popularity of C\#}
Several participants stated that \cs was simply more prevalent, therefore additional documentation, forum posts, tools and libraries would be available. These are very important because they aid developer productivity, which more importance than performance in game development. This was pointed out by participant 5 and was consistent with what other participants expressed. In the quotes below the relevant sentences have been highlighted with bold text.

\quoteParticipant{If my boss told me to do it, then I would use it I wouldn't refuse. But I wouldn't use it in my spare time, \textbf{\cs is too well documented and familiar}. It would take a while to get into \fs and the functions I am familiar with from \cs [...]}{Participant 1}{p1-debrief}

Participant 1 states that they would not use \fs, unless specifically asked to do so. Learning a new language or tool takes too long time, even when the benefits of the tool or language are evident. This is echoed by participant 4.

\quoteParticipant{It was very difficult to look at a new language again after such a long time. And I have worked with \cs and \unity almost everyday and \textbf{I am as familiar with \cs as is needed in \unity}
[...]\\
Alright, so it isn't anything I would ever use, but that is because I have used this since 2011. \textbf{It is so integrated in me that I just do that, that, and that and I am happy that I am at the point where I don't need to think about syntax errors}, because it is usually logic errors I get.}{Participant 4}{p4-debrief}

This is inline with the literature, which states (among other things) that developers tend to select languages they feel they know\cite{meyerovich2013empirical}. However participant 4's sentiment; that new languages are mainly useful for larger development studios, is in contradiction with the literature where evidence points to the opposite. Finally, the domain specificity i.e. the \gls{FRP} system, is a factor for adoption. This is also inline with the research. 

\subsection{Learning Overhead \& Cost}


\subsection{Comfort Zone}

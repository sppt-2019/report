\section{F\# Adoption Potential}\label{sec:adoption}\label{sec:test-conc}
While the \fs code produced during the user test in general was shorter than the \cs code (see \appref{terse-diff-comp}), the participants expressed that they would not personally choose \fs in a real project. Even participants who clearly identified the benefits of \fs, held this opinion. In this section we will explore the participants' reasons for this standpoint. The participants stated a number of rationales for this during the debriefing interview, which we categorised in four categories; prevalence and popularity of C\#, learning overhead and cost, comfort zone and different of priority. Each category is, in turn, analysed and explained.

\subsection{Prevalence and Popularity of C\#}
Several participants stated that \cs was simply more prevalent, which means that documentation, forum posts, tools and libraries would be more available. These sources of information are very important because they aid developer productivity, which is of higher importance than performance in game development. This was pointed out by participant 5 and was consistent with what other participants expressed. In the quotes below the relevant sentences have been highlighted with bold text.

\quoteParticipant{If my boss told me to do it, then I would use it, I wouldn't refuse. But I wouldn't use it in my spare time, \textbf{\cs is too well documented and familiar}. It would take a while to get into \fs and the functions I am familiar with from \cs [...]}{Participant 1}{p1-debrief}

Participant 1 states that he would not use \fs, unless specifically asked to do so. Learning a new language or tool takes too long time, even when the benefits of the tool or language are evident. This is echoed by participant 4:

\quoteParticipant{And I have worked with \cs and \unity almost everyday and \textbf{I am as familiar with \cs as is needed in \unity}
[...]\\
Alright, so it isn't anything I would ever use, but that is because I have used this since 2011. \textbf{It is so integrated in me that I just do that, that, and that} and I am happy that I am at the point where I don't need to think about syntax errors, because it is usually logic errors I get.}{Participant 4}{p4-debrief}

This is inline with the literature, which states (among other things) that developers tend to select languages they feel they know\cite{meyerovich2013empirical}. However, participant 4 also argued that new languages are mainly useful for larger development studios. This is in contradiction with the literature, where evidence points to the opposite. Finally, the participants' agreed that the use of events is a highly positive. We argue that the use of events is a corner stone in the domain specificity of the \gls{FRP} system, echoing scientific research that says that domain specific languages are more likely to be adopted\cite{meyerovich2013empirical}.

\subsection{Learning Overhead \& Cost}
All participants struggled with \fs's syntax, which was expected. The syntax struggle was also a factor in participants' decision making process. While \fs was a new language, it also presented a new paradigm. This paradigm shift was difficult for the participants, but few mentioned the shift directly.

\quoteParticipant{I really like functional, but now that I tried programming for games I am not sure.}{Participant 6}{p6-debrief}

However, most participants did not take note of the paradigmatic changes directly. Instead their focus was on the syntactical differences as well as the new keywords and operators. This may be caused by the their limited experience with \fs and that the paradigm shift may become more apparent to the participants.

\quoteParticipant{It was very difficult to look at a new language again after such a long time.}{Participant 4}{p4-debrief}

\quoteParticipant{However syntax-wise I was quite lost.}{Participant 1}{p1-debrief}

All participants, except 6, had very limited experience with the functional paradigm. The five first participants had Medialogy educations, which does not teach functional programming. According to \cite{meyerovich2013empirical} only 15\% of computer science students learn functional programming languages if they are not taught during their education. This further increases the learning cost associated with \fs, as the participants must acquaint themselves with both a new paradigm and a new language.

Participants were made aware of a potential advantage of \fs; the higher parallelisability, which may afford developers greater utilisation of the system resources. However, participant 5 pointed out that \dquote{\textit{Performance is provided by the engine, not the game}}\footnote{The footage of the fifth test was lost, therefore any and all quotes from participant 5 are those noted by the test observer during the test.}, which explains why none of the participants were particularly interested in \fs's potential for simpler or even implicit parallelisation.

\subsection{Comfort Zone}
As earlier stated, developers tend to select programming languages that they are familiar with\cite{meyerovich2013empirical}. This is reflected by the participants as well, where both participant 1 and 4 noted that they were too used to \cs to switch to \fs. Furthermore, participant 1 stated that the languages he used tended to be the more popular languages.

\quoteParticipant{I think in the context of what I would be assigned to do, if someone assigned this to me I would do it, but typically it is the more popular languages you are assigned.}{Participant 1}{p1-debrief}

In addition most participants had developed habits from \cs that they would have to unlearn or manage to productively write \fs. Syntactically the habits consisted of writing semicolons, curly braces and explicitly writing types in front of variables. On the semantic side, the inversion of access modifier defaults, immutability and implicit returns posed the largest challenge.

\quoteParticipant{Yea, I mean it is only a question of time before you get used to not making semicolons and curly brackets.}{Participant 1}{p1-debrief}

\quoteParticipant{It would help a lot if I had longer time work with a project.}{Participant 6}{p6-debrief}

Participants largely agreed that the habitual issues would disappear with more practice. Given more experience with \fs and experience with actual projects and problems, the paradigmatic differences may become more apparent to the participants.

\subsection{Difference of Priority}\label{sec:diff-pri}
Some of the potential benefits offered by \fs are not as appealing to the participants as initially believed. Surprisingly, multiple  participants expressed that performance was not a concern, which is in direct conflict with literature on software developers in general\cite{meyerovich2013empirical}. In addition productivity was deemed more important than program correctness.

\quoteParticipant{As is, I can't see the advantage of it because I already use \unity which manages everything. So there isn't, as such, any thing [requirement] there.}{Participant 4}{p4-debrief}

Participant 5 expressed that \fs was more readable because it forced correct program indentation, but still maintained that productivity was more important. Due to internet distribution platforms, bugs can be fixed at a later date and meeting the initial deadline is the main challenge. Furthermore, we speculate that games have a shorter lifetime (usually only one larger release and a series of smaller patches) than other software products and therefore maintenance is of lower priority.

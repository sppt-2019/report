\chapter{Experimenting with Parallelism}
In this chapter we conduct a series of experiments with parallel programs. First and foremost \cite{DBLP:journals/cl/Tremblay-parallel} claims that the lenient evaluation strategy provides high implicit parallelism by evaluating function arguments in parallel with the function body. This claim is put to the test using two binary tree benchmarks that was presented in the same article.

The results from the benchmarks indicate that there is a certain threshold of task-sizes, which must be exceeded before parallelisation has a positive effect on execution time. We attempt to estimate this threshold in \secref{crit:work} using two tests; a busy-wait delay estimation and matrix summation.

\section{Benchmarks}
In this section we explore the performance of the lenient evaluation strategy and how well it parallelises. The reason for this is that the interest in it has been lacking to say the least. We could not find a reason as to why and decided to explore whether it was because \cite{DBLP:journals/cl/Tremblay-parallel} gave false promises.

\subsection{Test cases}
\cite{DBLP:journals/cl/Tremblay-parallel} presents two test cases that we reuse in this experiment:

\begin{labeling}{\quad\quad}
    \item[Binary Tree Sum] where a tree-walker sums the values of all leaves of a binary tree.
    \item[Binary Tree Accumulation and Sort] where a tree-walker flattens all values of a tree's leaves into a list and sorts it.
\end{labeling}

The difference between the two test cases is that there are no data-dependencies in the summation test, i.e. we can calculate the results of the left subtree and the right subtree in parallel. In the Binary Tree Accumulation and Sort benchmark the tree must be traversed in a left-to-right order and thus there is a data-dependency between the results from the left and right subtree.

\subsection{Implementations}
In this benchmark we decided to use C\#, as we were familiar with the language. Furthermore, we discovered in previous work that C\#'s performance in the .NET Core runtime in many cases was faster than that of C++ in the context of microbenchmarks \cite{p92018gameplay}. We are well aware that C\# is not lenient, but we wanted to compare the strategies within the same language runtime and therefore attempted to map the lenient evaluation strategy onto C\#'s task system. \lstref{lenient:to:task} gives a code example how to do so.

\begin{listing}[H]
\begin{minted}{csharp}
//Lenient function, body evaluated asynchronously from caller
//parameters evaluated asynchronously from function body
int CalculateC(int a, int b) {
    int c = a * 50; //Implicit synchronisation with a-thread
    return c + b;   //Implicit synchronisation with b-thread
}

//C# mapping of the CalculateC-function
//async method that runs asyncronously from the caller
async Task<int> CalculateC_Lenient(Task<int> a, Task<int> b) {
    int c = await a * 50;   //Explicit synchronisation with a-task
    return c + await b;     //Explicit synchronisation with b-task
}
\end{minted}
\caption{Lenient evaluation in C\#} \label{lst:lenient:to:task}
\end{listing}
The mapping shown in \lstref{lenient:to:task} involves wrapping all arguments to methods in \ttt{Task}s and explicitly synchronise two \ttt{Task}s whenever there is a data-dependency between them. This means that all values of arguments are calculated asynchronously from the method body. The method is marked as \ttt{async}, meaning that a \ttt{Task} is spawned for every invocation of the method, i.e. it runs asyncronously from the caller. This leaves the majority of the footwork to the .NET Core task scheduler, which must figure out in which order to execute them.

We implemented the test cases in three different synchronisation models:
\begin{labeling}{\quad\quad}
    \item[Sequential] which uses divide and conquer on one unit of execution to do all the calculations. This provides the baseline speed for the computation on a single core.
    \item[Fork-Join] which also uses divide and conquer, but in each recursion step, two \ttt{Task}s are spawned to compute the results of the left and right subtree. The \ttt{Task}s are then synchronised at the end of each recursive call. This provides a \textit{\dquote{classic parallel}} baseline speed.
    \item[Lenient] using the mapping displayed in \lstref{lenient:to:task}, i.e. wrapping everything in \ttt{Task}s and have the .NET Core task scheduler figure out the computation order.
\end{labeling}

\subsection{Test Setup}
According to \cite{sestoft2013microbenchmarks} the most reliable results are obtained when the tests are repeated multiple times and the average and standard deviation calculated for the results. We therefore decided to take the average execution speed over 100 repetitions for varying problem sizes, starting with 10 leaf nodes and gradually increasing the number up to a total of 10,000 leaves.

The tests were run on a laptop, of which the specifications are listed in \tableref{sys-specs}.

\makeTable{
{| l | R{6em} | p{3em} |}
\hline
\multicolumn{3}{| c |}{\textbf{Processor}} \\ \hline
Model & \multicolumn{2}{| c |}{Intel Core i7 4702HQ} \\ \hline
Clock Frequency & 2.2 & GHz \\ \hline
Max Turbo & 3.2 & GHz \\ \hline
Physical & 4 & Cores \\ \hline
Logical\footnotemark & 8 & Cores \\ \hline
\multicolumn{3}{|c|}{\textbf{Memory}} \\ \hline
Memory Size & 16 & GiB  \\ \hline
Memory Speed & 1600 & MHz \\ \hline
Memory Type &  \multicolumn{2}{| c |}{DDR3L 1600} \\ \hline
\multicolumn{3}{|c|}{\textbf{Software}} \\ \hline
Operating System & \multicolumn{2}{| c |}{Ubuntu 18.04 64bit}  \\ \hline
C\# runtime & \multicolumn{2}{| c |}{dotnet 2.2.104} \\ \hline
}{System specifications.}{sys-specs}\footnotetext{Logical cores are sometimes called threads. However logical cores is used here to avoid confusion with the software concept; threads, which is distinct from hardware threads.}

\subsection{Results}
The results are plotted in \figref{binary-accumulation} and \figref{binary-summation} (and listed in \tableref{accumulation:res} and \tableref{summation:res} in \apxref{benchmark:data}).

\newcommand{\binAcumSymbolics}{\symbolic{Problem,10,100,1000,10000,100000}}
\barChart*[15][\binAcumSymbolics]{Binary Accumulation}{binary-accumulation}{
    \plotDataWithError{Sequential}{\accumulationData}
    \plotDataWithError{Fork Join}{\accumulationData}
    \plotDataWithError{Lenient}{\accumulationData}
}\btc{?? -in reference to problem size 100}
\barChart*[15][\binAcumSymbolics]{Binary Summation}{binary-summation}{
    \plotDataWithError{Sequential}{\summationData}
    \plotDataWithError{Fork Join}{\summationData}
    \plotDataWithError{Lenient}{\summationData}
}

Much to our surprise, the sequential implementation was actually the fastest in all cases. It seems that the overhead of spawning and synchronising \ttt{Task}s outweighs the performance gain of parallelisation, when the problem sizes are in the magnitude of additions and list appending. Furthermore, the Accumulation test case presented in \cite{DBLP:journals/cl/Tremblay-parallel} is a poor choice when it comes to parallelism, as it must traverse the tree in a left-to-right manner, meaning that the only things that can be executed in parallel is the recursive calls down the tree. Another interesting result is that the execution time of the lenient approach grows much faster with the problem size compared to both Fork Join and Sequential. We suspect that the advantages of parallel programming will be more prominent, as the amount of work in each unit of execution increases. We will research this suspicion in greater depth in the following section.

\section{Parallel Overhead \& Performance}\label{sec:crit:work}
In this section we present results from an experiment that estimates the critical workload for each task. We also implement a matrix summation benchmark to determine how different parallelisation strategies handle matrices of increasing sizes.

\subsection{Estimating Critical Workload}
In this experiment we estimate the critical workload of C\#'s \ttt{Task}-system. By critical workload we mean the time each task must execute before it is worthwhile to spawn it, compared to a sequential solution. The reason for this exploration is that we found the sequential solution to be faster in the binary tree benchmarks presented in the previous section.

\subsubsection{Test Setup}
We use the Binary Tree Summation benchmark presented in the previous section with a minor modification: Every time the algorithm finds a \ttt{Node}, it busy-waits for a given amount of time to simulate work. The busy-wait was implemented with a loop, whose number of iterations is gradually halved until the sequential solution executes faster than the parallel. The hypothesis here is that the parallel solutions will be faster, because it is capable of busy-waiting multiple tasks at the same time.

The Binary Summation test case from the previous section was reused in this experiment, but implemented in two variations. The first variation emulates a data dependency between the wait and the results the subtrees, i.e. the wait is intended to emulate a computation that must be carried out after the results of both subtrees have been computed. The other variation emulates a situation where the left and right subtree can be computed in parallel with the wait, i.e. no data dependency between the delay and the subtrees. The tree has a total of 60 leaf nodes. In addition to the binary tree summation, we also implemented a N-ary tree summation in the same variations as that of binary.

\subsubsection{Results}
The results are plotted in \figureref{crit-work-dep} and \figureref{crit-work-no-dep} (and listed in \tableref{binary:tree:with:bias:dependency} and \tableref{binary:tree:with:bias:no:dependency} in \apxref{crit:work:data}).

\newcommand{\workBiasSymbolics}{\symbolic{Work Bias (iterations),134217728,67108864,33554432,16777216,8388608,4194304,2097152,1048576,524288,262144,131072,65536,32768,16384,8192,4096,2048,1024}}
\lineChart{Critical workload with data dependency.}{crit-work-dep}{
    \plotData{Sequential}{\workWithDependencyData}
    \plotData{Fork Join}{\workWithDependencyData}
    \plotData{Lenient}{\workWithDependencyData}
}
\lineChart{Critical workload without data dependency.}{crit-work-no-dep}{
    \plotData{Sequential}{\workWithoutDependencyData}
    \plotData{Fork Join}{\workWithoutDependencyData}
    \plotData{Lenient}{\workWithoutDependencyData}
}
\lineChart{Critical workload with data dependency, N-ary tree.}{crit-work-dep-nary}{
    \plotData{Sequential}{\workWithDependencyDataNary}
    \plotData{Fork Join}{\workWithDependencyDataNary}
    \plotData{Lenient}{\workWithDependencyDataNary}
}
\lineChart{Critical workload with data dependency, N-ary tree.}{crit-work-no-dep-nary}{
    \plotData{Sequential}{\workWithoutDependencyDataNary}
    \plotData{Fork Join}{\workWithoutDependencyDataNary}
    \plotData{Lenient}{\workWithoutDependencyDataNary}
}

\tmc{Vi skal lave lidt analyse her og snakke om vores fund.}

\subsection{Matrix Summation}
In this section we execute a matrix summation benchmark. This benchmark measures the time it takes to sum all indices of a random $N x N$ matrix. This benchmark was implemented in different parallelisation strategies to explore how well they scale to increasing sizes of $N$:

\begin{labeling}{\quad\quad}
    \item[Sequential] utilises a double-nested for-loop to iterate over the matrix and sum the values. This benchmark provides a baseline value for running the computation on one thread.
    \item[Map Reduce] maps a function that sums each column over the matrix. The resulting list of column sums is then reduced to the overall sum of the matrix. In C\# we utilise the \gls{LINQ}-methods \ttt{Select}, \ttt{Sum} and \ttt{Aggregate}.
    \item[Parallel Foreach] uses a parallel loop to iterate over the columns of the matrix that may execute the summation of each column in parallel.
    \item[Tasks] is similar to parallel foreach, with the only exception that we manually spawn a \ttt{Task} that calculates the sum of each column.
\end{labeling}

We have not included a lenient-variation in this experiment, as an implementation in our C\# mapping would be largely equivalent to the Tasks-implementation (see \lstref{matrix-sum-csharp}). The most notable difference being that a lenient-evaluation strategy would most likely also construct the matrix in parallel with the summation. As the time it takes to construct a matrix is not included in the results here, this should have little to no effect on the validity of the results.
\begin{listing}
\begin{minted}{csharp}
public static async Task<long> SumTask(long[,] matrix)
{
    //Create an enumerable over the columns of the matrix
    var columns  = Enumerable.Range(0, matrix.GetLength(0));
    //Sum each column in parallel
    var sums = columns.Select(c => Task.Run(() =>
    {
        var sum = 0L;
        for(var i = 0; i < matrix.GetLength(1); i++)
        {
            sum = unchecked(sum + matrix[c, i]);
        }

        return sum;
    })).ToList();

    //Join the resuls and sum the sums of each column
    await Task.WhenAll(sums);
    return sums.SumUnchecked();
}
\end{minted}
\caption{Tasks implementation of Matrix Sum, largely equal to a lenient C\# mapping.} \label{lst:matrix-sum-csharp}
\end{listing}

As the matrices are of size $N x N$, they contain a total of $N^2$ elements with random values between \ttt{Int64.Minvalue} and \ttt{Int64.MaxValue}. When running the test with large matrices we found that even 64-bit integers would overflow, which throws an exception because C\# and F\# are managed languages. In order to avoid this, we used the \ttt{unchecked}-keyword, which disables bounds-checking on an integral arithmetic operation \cite{csharp:unchecked}. \cite{csharp:unchecked} states that using \ttt{unchecked} \textit{\dquote{might improve performance}}, compared to checked integral arithmetic operations.

\subsubsection{Results}
The results are plotted in \figureref{linpack-summation}. The first thing to notice is that Map Reduce seems to be roughly equal to the sequential in running time. This could indicate that the \ttt{Select}-method of C\#'s \gls{LINQ}, which was used to implement Map Reduce, does not parallelise its iterations. We will thus treat Map Reduce as a sequential solution for the rest of this result discussion.

\newcommand{\linpackSymbolics}{\symbolic{Problem Size,2,4,8,16,32,64,128,256,512,1024,2048,4096}}
\barChart*[7][\linpackSymbolics]{Matrix Summation}{linpack-summation}{
    \plotDataWithError{Sequential}{\linpackData}
    \plotDataWithError{Map Reduce}{\linpackData}
    \plotDataWithError{Parallel Foreach}{\linpackData}
    \plotDataWithError{Tasks}{\linpackData}
}
In general, the results from this experiment is in alignment with those of the previous, in that there is an initial overhead associated with parallelisation. In this case, it seems the sequential and parallel solutions evens out at parallel job sizes of around 256 summations, after which point the parallel solutions are faster.

After overcoming the initial overhead, the parallel solutions handle increasing matrix sizes much better than their sequential counterparts. This is even more notable in \figureref{linpack-summation-line}, which plots the same data as a line and without logarithmic y-axis. As the matrix sizes continue to grow, it may be possible to split the columns in multiple separate tasks, possibly making parallel faster yet.

\lineChart{Matrix Summation}{linpack-summation-line}{
    \plotData{Sequential}{\linpackData}
    \plotData{Map Reduce}{\linpackData}
    \plotData{Parallel Foreach}{\linpackData}
    \plotData{Tasks}{\linpackData}
}

\section{Concurrency in Unity} \label{sec:unity:concurrency}
Traditionally Unity has been single-threaded, but over the last two years, Unity has made an effort to implement currency in the form of a C\# Job System\cite{unity:csharp:job:system} and more recently an \gls{ECS}\cite{unity:ecs}. Unity does not support the \ttt{async}/\ttt{await}-model when accessing \ttt{MonoBehaviours} that is usually seen on the .NET platform\cite{unity:async,csharp:async}.

\subsection{C\# Job System}
Unity's C\# Job System provides a \textit{"simple and safe"} way of writing multithreaded code in Unity. It can be used on top of the \dquote{traditional} way of writing Unity code, i.e. \ttt{MonoBehaviours} and \ttt{GameObjects}. In the job system, the developer expresses concurrent code using jobs rather than threads. Unity is in charge of scheduling the jobs on a group of worker threads, which is shared with the engine code\cite{unity:csharp:job:system}.

Jobs in Unity must be implemented as \ttt{struct}s that implement one of the \ttt{IJob}-interfaces. This interface defines an \ttt{Execute}-method, in which the concurrent code must be written. Jobs can be scheduled by calling one of the \ttt{Schedule}-methods, which returns a \ttt{JobHandle}, which can be used to manage dependencies between jobs. The developer is fully in charge of figuring out the dependencies between the jobs, as Unity does not provide any means of dependency management\cite{unity:csharp:job:system}.

\lstref{unity:job:example} shows an example that moves forward all bullets in a game using a job.
\begin{listing}[H]
\begin{minted}{csharp}
public class BulletJobSystem : MonoBehaviour {
    //This array holds all bullets that need to be moved
    private TransformAccessArray _transforms;
    //This job handle may be used to synchronise with previously scheduled move jobs
    private JobHandle _bulletMoveHandle;
    public GameObject BulletPrefab;

    //This struct implements the actual job.
    public struct MoveBulletsJob : IJobParallelForTransform {
        public float _moveSpeed;
        public float _deltaTime;

        //In the Execute-method we define how we want to apply an update to a single bullet.
        public void Execute(int index, TransformAccess transform) {
            transform.position += _moveSpeed * _deltaTime * (transform.rotation * new Vector3(0,0,1));
        }
    }

    void Start() {
        //Create an array with 0 elements and unlimited capacity
        _transforms = new TransformAccessArray(0, -1);
    }

    public void SpawnBullet(Transform shooter) {
        //Instantiate the bullet as usual in Unity
        var bullet = Instantiate(BulletPrefab);
        bullet.transform.position = shooter.position;
        bullet.transform.forward = shooter.forward;

        //Synchronise with the job and add the new bullet to the _transforms array.
        _bulletMoveHandle.Complete();
        _transforms.capacity++;
        _transforms.Add(bullet.transform);
    }

    void Update() {
        //Complete the job, i.e. only apply one update at a time
        _bulletMoveHandle.Complete();

        //And schedule a new bullet update
        var bulletMoveJob = new MoveBulletsJob (5f, Time.deltaTime);
        _bulletMoveHandle = bulletMoveJob.Schedule(_transform);
        JobHandle.ScheduleBatchedJobs();
    }
}
\end{minted}
\caption{Implementation of a job that moves bullets forward in Unity.} \label{lst:unity:job:example}
\end{listing}

\subsection{Entity Component Systems}
\gls{ECS} is a design-pattern, which presents an alternative way of representing objects in a game world. It is an alternative to the scene graph pattern\cite{scene:graph}, which is used in both Unity and Unreal today. \glspl{ECS} consist of three different components\cite{unity:ecs,ecs:general}:
\begin{labeling}{\quad\quad}
    \item[Entities] represent single objects in the game. It is a very simple data structure, e.g. an ID, which is used to look up components associated with the object.
    \item[Components] are data containers, which defines data that is needed in order to carry out a certain behaviour. It is very important in \gls{ECS} that each component is kept small and defines as few fields as possible.
    \item[Systems] define the behaviour of the objects. They consist of two parts; a filter, which defines the components that present for the system to take effect and an update method, which applies the system's behaviour to the entities at regular intervals.
\end{labeling}

The advantage of \glspl{ECS}, compared to scene graphs, are that they tend to increase code reusability, as systems may be used to control multiple different types of objects. Furthermore, \glspl{ECS} will attempt to group components together in so-called chunks, which increases spatial locality. In Unity, the \gls{ECS} may be used in conjunction with the C\# Job System to update chunks in parallel\cite{unity:ecs}. \lstref{unity:ecs:example} lists an example, which moves all bullets in the scene forward.

\begin{listing}
\begin{minted}{csharp}
//MoveForward component. Defines data that is needed to move forward
struct MoveForward {
    public float Speed;
}

//C# Job system that moves forward in parallel
public class MoveForwardSystem : JobComponentSystem {
    //The generic types of the interface implements the filter
    struct MoveForwardJob : IJobForEach<Translation, Rotation, MoveForward> {
        private readonly float _deltaTime;

        //Move the entity forward. We do not have access to "standard" Unity methods, so we use the dedicated math-library
        public void Execute(ref Translation trans, [ReadOnly]ref Rotation rotation, [ReadOnly]ref MoveForward moveForward) {
            var forward = math.forward(rotation);
            var step = math.mul(moveForward.Speed, _deltaTime);
            var moveVec = math.mul(forward, new float3(step));
            trans.Value += moveVec;
        }
    }
}

//On each update schedule the movement
protected void JobHandle OnUpdate(JobHandle inputDeps) {
    var job = new MoveForwardJob{_deltaTime = Time.deltaTime};
    return job.Schedule(this, inputDeps);
}
\end{minted}
\caption{Implementation of an \gls{ECS} that moves bullets forward in Unity.} \label{lst:unity:ecs:example}
\end{listing}

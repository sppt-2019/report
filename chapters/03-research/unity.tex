\section{Concurrency in Unity} \label{sec:unity:concurrency}
Initially Unity was single-threaded, but over the last two years, Unity Technologies has made an effort to implement concurrency in the form of a C\# Job System\cite{unity:csharp:job:system} and more recently an \gls{ECS}\cite{unity:ecs}. Unity does not support the \ttt{async}/\ttt{await}-model, that is usually seen on the .NET platform, when dealing with \ttt{MonoBehaviours}\cite{unity:async,csharp:async}.

\subsection{C\# Job System}
Unity's C\# Job System provides a \textit{\dquote{simple and safe}} way of writing multithreaded code in Unity. It can be used on top of the \dquote{traditional} way of writing Unity code, i.e. \ttt{MonoBehaviour}s. In the job system, the developer expresses concurrent code using jobs rather than threads. Unity is in charge of running the jobs on a group of worker threads, which is shared with the engine code\cite{unity:csharp:job:system}.

Jobs in Unity must be implemented as \ttt{struct}s that implement one of the \ttt{IJob}-interfaces. This interface defines an \ttt{Execute}-method, in which the concurrent code must be written. Jobs can be scheduled by calling one of the \ttt{Schedule}-methods, which returns a \ttt{JobHandle}, that can be used to manage dependencies between jobs. The developer is in charge of figuring out the dependencies between the jobs, as Unity does not provide any means of dependency management\cite{unity:csharp:job:system}.

\lstref{unity:job:example} in \appendixref{unity:conc} shows an example that moves forward all bullets in a game using an \ttt{IParallelForTransform}-job.

\subsection{Entity Component Systems}
\gls{ECS} is a design-pattern, which presents an alternative way of representing objects in a game world. It is an alternative to the scene graph pattern\cite{scene:graph}, which is used in both Unity and Unreal today. \glspl{ECS} consist of three different components\cite{unity:ecs,ecs:general}:
\begin{labeling}{\quad\quad}
    \item[Entities] represent single objects in the game. It is a very simple data structure, e.g. an ID, which is used to look up components associated with the object.
    \item[Components] are data containers, which defines data that is needed in order to carry out a certain behaviour. It is very important that each component is kept small and defines as few fields as possible.
    \item[Systems] define the behaviour of the entities. They consist of two parts; a filter, which defines the components that must be present for the system to take effect and an update method, which applies the system's behaviour to the entities at regular intervals.
\end{labeling}

The advantage of \glspl{ECS}, compared to scene graphs, are that they tend to increase code reusability, as systems may be used to control multiple different types of objects. Furthermore, \glspl{ECS} will attempt to group components together in so-called chunks, which increases spatial locality. In Unity, the \gls{ECS} may be used in conjunction with the C\# Job System to update chunks in parallel\cite{unity:ecs}. \lstref{unity:ecs:example} in \appendixref{unity:conc} lists an example, which moves all bullets in the scene forward.

\chapter{Conclusion}
This chapter serves as the conclusion of the report. Here the work undertaken during this project is summarised in detail, whereafter the research questions posed in the problem statement (see \secref{problem_statement}) are answered.

\section{Project Summary}
In this project we examined the claims of two game development gurus: John Carmack and Tim Sweeney. These claims suggested that increased use of functional programming in game development would be beneficial. These potential benefits of functional languages were examined. Tim Sweeney directly suggested two features; explicit effects typing and lenient evaluation strategy. The performance impact of these language features were examined as well as other related boons, such as \gsl{FRP}, implicit parallelism and other concurrency benefits.

We examined if their point-of-view was shared by game development professionals in Aalborg, by conducting a usability evaluation, where the participants were tasked with implementing gameplay code in \fs. The test consisted of gameplay programming tasks inspired by game development. The developers were asked to implement solutions to these tasks in \fs and \cs. We found that the programmers were reluctant to adopt \fs because they believed that the cost of learning a new language would out-weigh the benefits it could provide.

In need of a stronger incentive to promote \fs we decided to examine if \fs could provide more performant code, than \cs, via concurrency. We found that \fs introduces a performance penalty compared to \cs, which is especially noticeable as problem sizes grow. Furthermore, the Async Workflows concurrency strategy employed by \fs seems to be fragile, in the sense that small differences can severely impact performance (using \m{Async.StartChild} instead of \m{Async.Parallel}). In comparison, the Task model employed in \cs seems more robust and also results in more performant concurrent code. The same results were obtained when benchmarking \fs in Unity, where \ttt{MonoBehaviour}s implemented in \fs were a little less performant than those in \cs. The \gls{FRP} system was implemented suboptimally, where each \ttt{FRPBehaviour} is in fact a full-blown \gls{FRP} system, which resulted in poor performance. However, the \gls{FRP} and Async Workflows were well understood by the participants during the test.

\section{Research Questions}
In this section we will answer the exploratory questions, which have guided the research in this project. The research questions are listed here for convenience:
\begin{center}
    \begin{enumerate}
        \item \textbf{How well do experienced game developers express gameplay code in the functional paradigm, in comparison to object-oriented programming?}
        \item How can functional programming be incorporated in game development?
        \item What are the performance impacts of using functional idioms in a game engine?
        \item What is required of functional programming to be adopted by game developers?
    \end{enumerate}
\end{center}

\subsection{Expressing Gameplay Code in \fs}
The game developers struggled with various aspects of \fs and the \gls{FRP} system, however they managed to produce concise gameplay code (see \tabref{comp-matrix} and \secref{test-conc}). In some cases the \fs code was shorter than the \cs code written during the test (see \appref{terse-diff-comp}). In general, the \fs code produced during the test often had qualities\footnote{Qualities such as conciseness, readability and modularity.} that were lacking in the \cs code. Even still, developers were unsure of whether \fs was worth investing time in, which is illustrated in in \tabref{att-inv-findings}. Several developers recognised the benefits of \fs, but remained unsure if it was worth the investment risk, to switch to the language.

Of the findings from the usability test, the most surprising result was the difference of priorities (see \secref{diff-pri}). It was expected that developers would be mindful of risks associated with learning a new tool and that they may be reluctant to try something new, but developers disregarded the advantages of \fs because they were unnecessary. The participants did not explicitly state the concrete reason why these features were unnecessary, but we believe the rapid and short life cycle of modern games is partly the cause.

\subsection{Incorporating \fs in Game Development}
In this project we chose to add support for \fs in Unity by installing a 3rd party plugin. Using this plugin we implemented a simple \gls{FRP} system that allows gameplay programmers to employ the functional reactive programming paradigm. The participants of the usability evaluation agreed that event-driven programming is well-suited for game development and would improve code quality.

The \gls{FRP} system makes use of a paradigm within the functional paradigm, which forces developers to think about their code in a different way (see \secref{frp} and \secref{frp-sys}). This served as a segue into the functional paradigm. In contrast many of the traditional introductions to functional programming are not necessarily useful in real-world game development.

\subsection{Performance Impacts of F\#}
Using microbenchmarking techniques and a set of benchmarks, we examined the performance of \fs both standalone and in conjunction with \unity (see \secref{benchmarks}). We found that \fs introduces a small overhead compared to \cs. %, however this overhead is not statistically significant (see \secref{perf-diff})\tmc{Shit! Go back! Jeg vil helst ikke snakke om statistik!}.
We implemented three different benchmarks in this project: binary tree summation/accumulation, matrix summation and unit management (see \secref{bench-impls}). All the benchmarks showed that \fs was slightly slower than equivalent code in \cs. Furthermore, the binary tree summation/accumulation benchmark showed that the Async Workflow concurrency model in \fs was much slower than the other concurrent implementations (see \secref{bench-res}). As \cs and \fs run on the same platform, programmers may use \cs's Task model in \fs, which yields only slightly slower code in \fs.

\subsection{F\# Adoption Requirements}
The plausibility of the participants adopting \fs has been discussed in \secref{test-conc}. The participants list a number of problems that need to addressed before they would consider using \fs. The primary concerns raised by the participants are the cost of learning the language, and the mismatch between desired benefits and gained benefits.

Easing the learning process may also alleviate another issue with \fs: the availability of \unity/\fs documentation\footnote{While \fs has sample documentation, very limited documentation is available for using \fs with Unity.}. Additional documentation and assistance would allow developers to learn much more rapidly, which was a major concern for most participants.

The benefits provided by \fs were modularity and maintainability. Some participants noted that \fs is ideal if you know the game you want to implement well, however they concluded that this is not the case in most game development scenarios (see \secref{test-conc} and \appref{p4-debrief}). The modularity afforded by \fs was not considered an indicator for faster development, but rather an indicator of more maintainable code.

\section{Closing Remarks}
In conclusion, \fs was met with mixed enthusiasm from the participants, while the use of events in gameplay programming was met with almost unanimous approval. The primary reasons identified, for the participants stance on \fs, were the high perceived cost of learning the language, and the cultural shock of the functional paradigm. The benefits of \fs were clear to several participants, but many did not consider them relevant to game development, where productivity is of utmost importance. This schism, may be caused by participants not fully grasping the advantages, but this requires additional research. The .NET platform has reached a point where the performance impact of using functional programming in game development is minimal. This means that the remaining barriers are habitual and preferential. We envision that such barriers can be breached by teaching the functional paradigm early in game developers' education.

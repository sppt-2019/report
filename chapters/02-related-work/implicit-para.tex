\section{Implicit Parallelisation}\label{sec:implicit-para}
Functional programming has long claimed to be inherently parallelisable\cite{hudak1986functional, loidl1998granularity}. The claimed advantage of functional programs is the high degree of modularity, which simplifies the parallelisation process. A number of languages have attempted to implement programming systems that utilise this advantage, however few have reached mainstream use. In this section we will explore the theoretical background for implementing an implicitly parallelisable system in \fs.

First different evaluation strategies are explored, since \fs supports both strict and non-strict evaluation. In addition to these two dominant evaluation strategies, another promising strategy, called lenient evaluation, is examined. Once the strategies are outlined, the question of when to parallelise is addressed with the work/span methodology.

\subsection{Evaluation Strategies}
Programming language behaviour is heavily dependent on the evaluation strategy employed. Conventionally two such strategies are prominent in the literature\cite{DBLP:journals/cl/Tremblay-lenient}; \textit{strict} (also called eager) and \textit{non-strict} (commonly known as lazy, though this is not entirely accurate). \cite{DBLP:journals/cl/Tremblay-lenient} suggests a \textit{lenient} evaluated language. This strategy is non-strict, but not lazy, which places it in-between eager and lazy\cite{DBLP:journals/cl/Tremblay-lenient}. This intermediary position means that the lenient strategy benefits from both eager and lazy advantages. According to literature lenient evaluation lends itself to implicit parallelism\cite{DBLP:journals/cl/Tremblay-parallel}. This section will explore the three strategies and clarify the distinctions between them.

In Sweeney's discussion of the game-programming language of tomorrow, he underlines that it should make use of the lenient evaluation strategy\cite{theNextMainstreanProgrammingLanguage}. The reason for this is that eager evaluation strategy is too limiting and lazy evaluation is too costly. Sweeney also underlines that the compiler of this new language should be capable of optimising pieces of code to use eager evaluation, whenever it is more optimal.

\subsubsection{Strict Evaluation}
Strict evaluation, often called eager evaluation, is a prominent evaluation strategy in traditional programming languages. The defining feature of the strategy lies in the handling of function parameters. Eager evaluation requires fully evaluated parameters before a function may be evaluated\cite[p.~103]{huttel2010transitions}, however this is not the only requirement. Depending on the choice of parameter passing approach, the evaluation strategy may also be affected. Chief among eager parameter passing approaches are call-by-value and call-by-reference. In the case of call-by-value a parameter must be evaluated and its result passed to the function. On the other hand call-by-reference has the same requirement, but instead the memory address of the result is passed to the function.

\subsubsection{Non-strict Evaluation}
Contrary to strict evaluation, non-strict evaluation does not require parameter evaluation until they are needed\cite{hudak1989conception}. This strategy is often implemented using call-by-name or call-by-need parameter-passing approaches. This strategy affords the programmers more expressive power\cite{bird1997more} and allows for infinite data structures and non-terminating functions\cite[p.~103]{huttel2010transitions}. The parameter passing approaches state that parameter evaluation is delayed until they are called, therefore unused parameters need not be calculated at all.

\subsubsection{Lenient Evaluation}
The lenient evaluation strategy does not restrict the order of parameter evaluation. The only requirement is that the variables are available when they are needed, or in other words that the data dependencies are satisfied. Parameter passing approaches in this strategy often make extensive use of parallelism and concurrency techniques, due to the inherent parallelisability of lenient evaluation\cite{DBLP:journals/cl/Tremblay-parallel}.

The most notable instantiation of this evaluation strategy is call-by-future\cite{baker1977incremental}. In this implementation the main thread of the program executes the program and every time it encounters a function invocation, it spawns a future for each argument to the function. The main thread then continues to execute the function-body, synchronising with the futures as arguments are needed in the function-body. This has the result that functions and arguments are computed concurrently.\tmcc{Need an outro here, something about parallelisation cost and figuring out when to parallelise}


%\todo{use this \cite{glauert1993new}.}

\subsection{Formal Performance of Parallel Programs}\label{sec:work-span}
This section will detail a formal method of estimating performance of parallel programs called work and span. This method counts the number of primitive operations required to execute the entire program. This is called the sequential running time of the program and is denoted: $T(n)$ where $n$ is the problem size\cite{DBLP:books/daglib/para-algo}. The sped up running time, using additional processors, is denoted: $T_p(n)$. Here $p$ denotes the number of processors, Thus:
\begin{labeling}{\quad\quad}
    \item[$T(n)$] denotes the total sequential running time.
    \item[$T_p(n)$] denotes the total running time of the program when parallelised as much as possible.
\end{labeling}
In order to estimate $T_p(n)$ the \textit{work} and \textit{span} of the program must be identified. The work is the total running time of all processors, ignoring synchronisation overheads. This is equivalent to running the program on a single processor or sequentially. Therefore
\begin{equation}
    \text{work} = T(n)
\end{equation}
The span is the longest data dependent path in the program i.e. the longest path of strictly sequential computation. This is sometimes called the critical path or the computational depth\cite{Blelloch:1996:PPA:prog-para-algo}. The shorter the span, the more parallelisable the program. Finally the cost of the program can be calculated. This is the total running time across all processors including the time spent idling. The cost is denoted $pT_p$.

Given this information about a parallel program, the speed-up gain from parallelisation can be calculated. This calculation assumes an infinite number of processors, $T_\infty$. A number of different metrics for this gain exist, they are as follows.
\begin{labeling}{\quad\quad}
    \item[Speed-up] is the raw gain from running the program on multiple processors.
    \begin{equation*}
        S_p = T_1/T_p
    \end{equation*}
    \item[Efficiency] is the speed-up per processor.
    \begin{equation*}
        S_p/p
    \end{equation*}
    \item[Parallelism] is the maximum possible speedup given a number of processors.
    \begin{equation*}
        T_1/T_\infty
    \end{equation*}
    \item[Slackness] is a measure of the program's parallelisability. A slackness of less then one implies that perfect linear speedup is possible.
    \begin{equation*}
        T_1/(pT_\infty)
    \end{equation*}
\end{labeling}
Since no actual machine has an infinite number of processors, the above equations require slight modifications to simulate real machines. Any computation that can run on $N$ processors can be executed on smaller number of processors, $p < N$\cite{Gustafson2011}. This is achieved by dividing the work load onto the processors, instead of assigning one processor per task. Furthermore, running on fewer then $N$ processors the execution is bounded by:
\begin{equation*}
    T_p \leq T_N + \frac{T_1 - T_n}{p}
\end{equation*}
The bound $T_p$ can be expressed with upper and lower bound\cite{brent1974parallel}:
\begin{equation*}
    \frac{T_1}{p} \leq T_p \leq \frac{T_1}{p} + T_\infty
\end{equation*}
\tmcc{Outro needed here too}


\subsection{Lenient Parallelisation}
Using a highly granular parallelisation system, such as .NET \m{Task}s, each argument to a function could be evaluated as a task, unless the expression is too small. Determining the computational cost of an argument is calculated by estimating the work of the expression. This information, along with the span of the program, should provide enough information to the system to effectively parallelise a lenient program.
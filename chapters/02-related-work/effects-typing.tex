\section{Effects Typing}
In the problem statement we presented a quote from Tim Sweeney suggesting that \dquote{\textit{imperative constructs should be made available via explicit effects typing}} (see Chapter \ref{chap:introduction}). Effects typing i.e. type and effect systems, are type systems that track types and the changes made to them\cite{nielson1999type}. These changes are referred to as effects. Such changes could be modifying a mutable variable, opening a file handle, reading or writing to a shared resource. These systems are mainly seen in academia, where annotated type and effect systems have been implemented. The annotated approach has proven to be unwieldy to program\needcite, which may explain the small number of languages using such systems. Recently type and effect systems have been subject to renewed interest and the Rust programming language incorporates a similar system\cite{rust:lang}.

The effects tracked by type and effect systems can be extended to parallel changes. Thus the system can be used to manage concurrency and guarantee equivalent behaviour between sequential and concurrent implementations of the same problem\cite{krogh2017relational}. While type and effects systems have been around for a awhile, it has recently been proven to be able to calculate when it is sound to parallelise a task\cite{birkedal2012concurrent}. This information is discernible via static analysis of the source code, using an annotated type and effect system, at compile time.
